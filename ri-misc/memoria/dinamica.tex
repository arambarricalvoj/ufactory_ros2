\datos{color}{}{}

\section{Estudio dinámico}
El fabricante proporciona los parámetros clásicos y modificados de Denavit-Hartenberg en el manual del usuario. En las 
siguientes secciones se van a estudiar dichos parámetros.

\begin{comment}
    Recopilar parámetros dinámicos del fabricante

Masas de cada eslabón.

Posición del centro de masas de cada eslabón.

Tensores de inercia respecto a los ejes locales.

Parámetros DH (clásicos o modificados) para definir la geometría.

Elegir el modelo dinámico

Euler–Lagrange: parte de la energía cinética y potencial para derivar las ecuaciones de movimiento.

Newton–Euler recursivo: más eficiente computacionalmente, calcula fuerzas y pares propagando desde la base al efector y viceversa.

Formulación general La dinámica inversa se expresa como:

𝜏
(
𝑡
)
=
𝑀
(
𝑞
)
 
𝑞
¨
(
𝑡
)
+
𝐶
(
𝑞
,
𝑞
˙
)
 
𝑞
˙
(
𝑡
)
+
𝑔
(
𝑞
)
𝑀
(
𝑞
)
: matriz de inercia.

𝐶
(
𝑞
,
𝑞
˙
)
: términos de Coriolis y centrífugos.

𝑔
(
𝑞
)
: vector de gravedad.

𝜏
(
𝑡
)
: pares articulares.

Usar tus trayectorias calculadas

Ya tienes 
𝑞
(
𝑡
)
, 
𝑞
˙
(
𝑡
)
, 
𝑞
¨
(
𝑡
)
 de la interpolación quíntica.

Sustituye esos valores en las ecuaciones dinámicas para obtener 
𝜏
(
𝑡
)
.

Validar con software

MATLAB/Simulink tiene librerías de dinámica de robots (rigidBodyTree, inverseDynamics).

También puedes implementar las ecuaciones manualmente si el fabricante te da los parámetros.
\end{comment}