\datos{color}{}{}

\section{Estudio dinámico}
El fabricante proporciona las características físicas de cada eslabón, 
en particular la masa y la posición de su centro de masa (\textit{CoM}), figura \ref{fig:dinamica_param},
respecto de los sistemas de referencia de la figura \ref{fig:dinamica_sist} con la tabla de parámetros 
\ref{tab:dh_modificada_din}. También proporcionan en los modelos de simulación los momentos de inercia \cite{Spong2006} 
de cada eslabón, expuestos en el código \ref{lst:inercia}. El modelo del robot con el que se trabaja 
corresponde al modelo 1 del robot, que se alimenta a través de la \textit{AC Control Box}.

\begin{figure}[htb]
    \centering
    \begin{minipage}[t]{0.3\textwidth}
        \centering
        \includegraphics[scale=0.375]{Images/dinamica/sistemas.png}
        \caption{Sistemas de referencia para el estudio dinámico. Fuente: \cite{ufactory_xarm_manual}.}
        \label{fig:dinamica_sist}
    \end{minipage}
    \begin{minipage}[t]{0.65\textwidth}
        \centering
        \includegraphics[scale=0.315]{Images/dinamica/tabla.png}
        \caption{Parámetros de la dinámica del robot. Fuente: \cite{ufactory_xarm_manual}.}
        \label{fig:dinamica_param}
    \end{minipage}
\end{figure}

\vspace{-0.3cm}
\begin{table}[htb]
\centering
\begin{tabular}{c c c c c}
\hline
\textbf{$i$} & \textbf{$\theta_i$} ($^\circ$) & \textbf{$d_i$} (mm) & \textbf{$a_{i-1}$} (mm) & \textbf{$\alpha_{i-1} (^\circ)$} \\
\hline
1 & $\theta_1$         & $267$         & $0$         & $0$         \\
2 & $\theta_2$         & $0$         & $0$         & $-90$         \\
3 & $\theta_3 + (\textit{T2\_offset} = -79.34995)$         & $d_3$         & $a_2 = 289.48866$         & $0$         \\
4 & $\theta_4 + (\textit{T2\_offset} = 79.34995)$         & $342.5$         & $77.5$         & $-90$         \\
5 & $0$         & $0$         & $0$         & $90$         \\
6 & $0$         & $97$         & $76$         & $-90$         \\
\hline
\end{tabular}
\caption{Parámetros DH modificados del \textit{UFactory xArm6} para el estudio dinámico.}
\label{tab:dh_modificada_din}
\end{table}

\lstinputlisting[
    caption={Momentos de inercia del los eslabones del robot en estudio.},
    label={lst:inercia},
    breaklines=true,
    breakatwhitespace=true
]{Matlab/inercias.txt}


\newpage
\subsection{Ecuaciones de movimiento}

Las ecuaciones de movimiento de un robot manipulador son las expresiones matemáticas que describes cómo las
fuerzas y pares aplicados en las articulaciones producen el movimiento del sistema, teniendo en cuenta 
su masa, inercias, geometría y la acción de la gravedad. Se tendrá una ecuación de movimiento 
por cada eslabón y finalmente se obtiene el vector de pares articulares $\tau = [\tau_1, \tau_2, ..., 
\tau_6]\; N\cdot m$. La expresión general de la ecuación de movimiento es la siguiente:
\begin{equation} 
M(q)\,\ddot{q} + C(q,\dot{q})\,\dot{q} + g(q) = \tau 
\label{eq:movimiento_estandar} 
\end{equation}

donde:
\begin{itemize}
    \item \(M(q)\) es la matriz de inercia (simétrica definida positiva).
    \item \(C(q,\dot{q})\,\dot{q}\) agrupa términos centrífugos y de Coriolis.
    \item \(g(q)\) es el vector de gravedad.
    %\item \(F_v\,\dot{q}\) y \(F_c\,\mathrm{sgn}(\dot{q})\) modelan la fricción viscosa y de Coulomb, respectivamente.
    \item \(\tau\) son los pares actuados en las articulaciones.
    %\item \(J(q)\) es el jacobiano geométrico del efector final y \(f_{\text{ext}}\) fuerzas/momentos externos aplicados.
\end{itemize}

En \cite{siciliano2009robotics} se extiende la ecuación de movimiento para añadir las fricciones viscosa
\(F_v\,\dot{q}\) y de Coulomb \(F_c\,\mathrm{sgn}(\dot{q})\) y las fuerzas externas, donde 
\(J(q)\) es el jacobiano geométrico del efector final y \(f_{\text{ext}}\) las fuerzas externas aplicadas.
Esta versión extendida se presenta a continuación: 
\[
M(q)\,\ddot{q} + C(q,\dot{q})\,\dot{q} + g(q) + F_v\,\dot{q} + F_c\,\mathrm{sgn}(\dot{q}) = \tau + J^\top(q)\,f_{\text{ext}}
\]

Como el fabricante no proporciona los datos de friccioón y se va a trabajar en simulación, 
únicamente se estudia la expresión general. \\

Las ecuaciones de movimiento pueden calcularse mediante el balance de energías Lagrangiano
\cite{siciliano2009robotics} o por el método recursivo de Newton–Euler \cite{barrientos2007fundamentos}, siendo este último
el más habitual por ser computacionalmente más eficiente.

\subsubsection{Enfoque energético (Lagrangiano)}

En términos energéticos, los componentes pueden obtenerse a partir del lagrangiano \(\mathcal{L}=T-V\), 
donde \(T\) es la energía cinética total del manipulador y \(V\) la energía potencial gravitatoria.

La energía cinética se calcula como:


\[
T = \frac{1}{2}\sum_{i=1}^{n}\left( m_i\,\dot{p}_{c_i}^\top \dot{p}_{c_i} 
+ \omega_i^\top I_{c_i}\,\omega_i \right),
\]


donde \(m_i\) es la masa del eslabón \(i\), \(\dot{p}_{c_i}\) la velocidad del centro de masa del eslabón \(i\), 
\(\omega_i\) la velocidad angular del eslabón \(i\) e \(I_{c_i}\) su tensor de inercia respecto al centro de masa.\\

La energía potencial gravitatoria se expresa como:
\[
V = \sum_{i=1}^{n} m_i\,g^\top p_{c_i},
\]
donde \(g\) es el vector de aceleración de la gravedad y \(p_{c_i}\) la posición del centro de masa del eslabón \(i\). \\

A partir de estas expresiones, se obtienen los componentes de la ecuación de movimiento (ec. \ref{eq:movimiento_estandar}):
\[
M_{ij}(q) = \frac{\partial^2 T}{\partial \dot{q}_i \,\partial \dot{q}_j}, 
\qquad
g_i(q) = \frac{\partial V}{\partial q_i},
\]
y los elementos de \(C(q,\dot{q})\) mediante los símbolos de Christoffel:

\[
c_{ijk} = \frac{1}{2}\left(\frac{\partial M_{ij}}{\partial q_k} + \frac{\partial M_{ik}}{\partial q_j} - \frac{\partial M_{jk}}{\partial q_i}\right),
\qquad
[C(q,\dot{q})]_{ij} = \sum_{k=1}^{n} c_{ijk}\,\dot{q}_k.
\] \\


\subsubsection{Método recursivo de Newton--Euler}
Mientras que el enfoque Lagrangiano produce explícitamente \(M(q), C(q,\dot{q}), g(q)\),
Newton--Euler es más eficiente computacionalmente para evaluar \(\tau\) dado \(\{q,\dot{q},\ddot{q}\}\)
y es el más usado en simulación y control en tiempo real. \\

El método recursivo de Newton--Euler calcula directamente los pares articulares \(\tau_i\)
a partir de \(\{q,\dot{q},\ddot{q}\}\) (variables articulares) y los parámetros dinámicos \(\{m_i, r_{c_i}, I_{c_i}\}\)
(masas, centros de masa e inercias), mediante un barrido hacia adelante (cinemática y aceleraciones) 
y otro hacia atrás (acumulación de fuerzas y momentos).

\paragraph{Barrido hacia adelante}
Condiciones en la base:
\[
\omega_0 = 0,\quad \alpha_0 = 0,\quad a_0 = -g.
\]

Para \(i=1,\dots,n\), con \(R_i\) y \(p_i\) de las transformaciones homogéneas y \(z_i\) el eje articular:
\[
\omega_i = R_i^\top \omega_{i-1} + \dot{q}_i z_i,
\]

\[
\alpha_i = R_i^\top \alpha_{i-1} + \ddot{q}_i z_i + \dot{q}_i z_i \times (R_i^\top \omega_{i-1}),
\]

\[
a_i = R_i^\top \left( a_{i-1} + \alpha_{i-1} \times p_i + \omega_{i-1} \times (\omega_{i-1} \times p_i) \right),
\]

\[
a_{c_i} = a_i + \alpha_i \times r_{c_i} + \omega_i \times (\omega_i \times r_{c_i}).
\]

\paragraph{Fuerzas y momentos en el centro de masa}
\[
F_i = m_i a_{c_i}, \qquad
N_i = I_{c_i} \alpha_i + \omega_i \times (I_{c_i} \omega_i).
\]


\paragraph{Barrido hacia atrás}
Acumulación desde el efector hasta la base. Para \(i=n,\dots,1\):
\[
f_i = F_i + \sum_{k=i+1}^{n} R_k f_k,
\]

\[
n_i = N_i + r_{c_i} \times F_i + \sum_{k=i+1}^{n} \left( R_k n_k + r_{ik} \times (R_k f_k) \right),
\]

donde \(r_{ik}\) es el vector del origen \(i\) al origen \(k\).

\paragraph{Proyección sobre el eje articular}
Para articulación rotacional:
\[
\tau_i = n_i^\top z_i,
\]

y opcionalmente se añaden fricciones \cite{siciliano2009robotics}:
\[
\tau_i \leftarrow \tau_i + f_{v,i} \dot{q}_i + f_{c,i} \mathrm{sgn}(\dot{q}_i).
\] \\

\subsubsection{Ejemplo de una iteración para el eslabón 3 del \textit{UFactory xArm6}}

A modo ilustrativo, se muestra una iteración completa del método recursivo de Newton--Euler
para el eslabón $i=3$ del robot en estudio, utilizando la tabla \ref{tab:dh_modificada_din} y 
las matrices \(A_i\) calculadas en el apartado \ref{sec:dh-mod}.

\paragraph{Transformaciones homogéneas y descomposición en rotación y traslación}
\[
d_1 = 267,\qquad a_2 = 289.48866,\qquad a_3 = 77.5,\qquad d_4 = 342.5,\qquad a_5 = 76,\qquad d_6 = 97.
\]

Para el cálculo en el barrido hacia adelante del eslabón \(3\), necesitamos
\({}^{0}\!A_1\), \({}^{1}\!A_2\) y \({}^{2}\!A_3\). Denotando \(C_i=\cos(\cdot)\) y \(S_i=\sin(\cdot)\):

\[
A_1(q_1) =
\begin{bmatrix}
\cos q_1 & -\sin q_1 & 0 & 0 \\
\sin q_1 & \cos q_1 & 0 & 0 \\
0 & 0 & 1 & d_1 \\
0 & 0 & 0 & 1
\end{bmatrix}
\Rightarrow
{}^{0}\!R_1=
\begin{bmatrix}
\cos q_1 & -\sin q_1 & 0 \\
\sin q_1 & \cos q_1 & 0 \\
0 & 0 & 1
\end{bmatrix},\quad
{}^{0}\!p_1=\begin{bmatrix}0\\0\\d_1\end{bmatrix}.
\]

\[
A_2(\theta_2 = q_2+T2_{\text{offset}}) =
\begin{bmatrix}
\cos \theta_2 & -\sin \theta_2 & 0 & 0 \\
0 & 0 & 1 & 0 \\
-\,\sin \theta_2 & -\,\cos \theta_2 & 0 & 0 \\
0 & 0 & 0 & 1
\end{bmatrix},
\Rightarrow
{}^{1}\!R_2=
\begin{bmatrix}
\cos \theta_2 & -\sin \theta_2 & 0 \\
0 & 0 & 1 \\
-\,\sin \theta_2 & -\,\cos \theta_2 & 0
\end{bmatrix},\quad
{}^{1}\!p_2=\begin{bmatrix}0\\0\\0\end{bmatrix}.
\]

\[
A_3(\theta_3 = q_3+T3_{\text{offset}}) =
\begin{bmatrix}
\cos \theta_3 & -\sin \theta_3 & 0 & a_2 \\
\sin \theta_3 & \cos \theta_3 & 0 & 0 \\
0 & 0 & 1 & 0 \\
0 & 0 & 0 & 1
\end{bmatrix},
\Rightarrow
{}^{2}\!R_3=
\begin{bmatrix}
\cos \theta_3 & -\sin \theta_3 & 0 \\
\sin \theta_3 & \cos \theta_3 & 0 \\
0 & 0 & 1
\end{bmatrix},\quad
{}^{2}\!p_3=\begin{bmatrix}a_2\\0\\0\end{bmatrix}.
\]



La pose del marco \(3\) respecto a la base se obtiene con:


\[
{}^{0}\!T_3 = {}^{0}\!A_1\,{}^{1}\!A_2\,{}^{2}\!A_3,\qquad
{}^{0}\!R_3 = {}^{0}\!R_1\,{}^{1}\!R_2\,{}^{2}\!R_3,\qquad
{}^{0}\!p_3 = {}^{0}\!p_1 + {}^{0}\!R_1\,{}^{1}\!p_2 + {}^{0}\!R_1\,{}^{1}\!R_2\,{}^{2}\!p_3.
\]

\[
{}^{0}\!R_3 = {}^{0}\!R_1\,{}^{1}\!R_2\,{}^{2}\!R_3
=
\begin{bmatrix}
\cos q_1 & -\sin q_1 & 0 \\
\sin q_1 & \cos q_1 & 0 \\
0 & 0 & 1
\end{bmatrix}
\!
\begin{bmatrix}
\cos \theta_2 & -\sin \theta_2 & 0 \\
0 & 0 & 1 \\
-\,\sin \theta_2 & -\,\cos \theta_2 & 0
\end{bmatrix}
\!
\begin{bmatrix}
\cos \theta_3 & -\sin \theta_3 & 0 \\
\sin \theta_3 & \cos \theta_3 & 0 \\
0 & 0 & 1
\end{bmatrix}.
\]

\[
{}^{0}\!R_3 = {}^{0}\!R_2\,{}^{2}\!R_3
=
\begin{bmatrix}
\cos q_1 \cos \theta_2 & -\cos q_1 \sin \theta_2 & -\sin q_1 \\
\sin q_1 \cos \theta_2 & -\sin q_1 \sin \theta_2 & \cos q_1 \\
-\,\sin \theta_2 & -\,\cos \theta_2 & 0
\end{bmatrix}
\!
\begin{bmatrix}
\cos \theta_3 & -\sin \theta_3 & 0 \\
\sin \theta_3 & \cos \theta_3 & 0 \\
0 & 0 & 1
\end{bmatrix}.
\]

\[
{}^{0}\!R_3 =
\begin{bmatrix}
\cos q_1(\cos \theta_2 \cos \theta_3 - \sin \theta_2 \sin \theta_3)
&
\cos q_1(-\cos \theta_2 \sin \theta_3 - \sin \theta_2 \cos \theta_3)
&
-\sin q_1
\\
\sin q_1(\cos \theta_2 \cos \theta_3 - \sin \theta_2 \sin \theta_3)
&
\sin q_1(-\cos \theta_2 \sin \theta_3 - \sin \theta_2 \cos \theta_3)
&
\cos q_1
\\
-(\sin \theta_2 \cos \theta_3 + \cos \theta_2 \sin \theta_3)
&
\sin \theta_2 \sin \theta_3 - \cos \theta_2 \cos \theta_3
&
0
\end{bmatrix}.
\]





\[
{}^{0}\!p_3 = {}^{0}\!p_1 + {}^{0}\!R_1\,{}^{1}\!p_2 + {}^{0}\!R_1\,{}^{1}\!R_2\,{}^{2}\!p_3
= 
\begin{bmatrix}0\\0\\267\end{bmatrix}
+
\underbrace{{}^{0}\!R_1\,\begin{bmatrix}0\\0\\0\end{bmatrix}}_{=\,\mathbf{0}}
+
\left({}^{0}\!R_1\,{}^{1}\!R_2\right)\begin{bmatrix}289.48866\\0\\0\end{bmatrix}.
\]



Como \({}^{0}\!R_1\,{}^{1}\!R_2\) tiene primera columna
\(\big[\cos q_1 \cos \theta_2,\; \sin q_1 \cos \theta_2,\; -\sin \theta_2\big]^\top\),
se obtiene:


\[
{}^{0}\!p_3 =
\begin{bmatrix}
289.48866\,\cos q_1 \cos \theta_2 \\
289.48866\,\sin q_1 \cos \theta_2 \\
267 - 289.48866\,\sin \theta_2
\end{bmatrix}.
\]





\[
{}^{0}\!T_3 =
\begin{bmatrix}
{}^{0}\!R_3 & {}^{0}\!p_3 \\
\mathbf{0}_{1\times 3} & 1
\end{bmatrix}.
\]





\paragraph{Barrido hacia adelante}

Tomamos el eje articular local $z_3 = [0, 0, 1]^{T}$ y convertimos las longitudes a unidades 
del Sistema Internacional: $d_1 = 0.267 \;\text{m},\; a_2 = 0.28948866 \;\text{m}$.


\subparagraph{Velocidad angular}
\[
\omega_3 = R_3^\top\,\omega_2 + \dot{q}_3\,z_3
=
\begin{bmatrix}
\cos\theta_3 & \sin\theta_3 & 0 \\
-\sin\theta_3 & \cos\theta_3 & 0 \\
0 & 0 & 1
\end{bmatrix}
\begin{bmatrix}\omega_{2x}\\\omega_{2y}\\\omega_{2z}\end{bmatrix}
+
\dot{q}_3\begin{bmatrix}0\\0\\1\end{bmatrix}.
\]

\[
\Rightarrow
\begin{cases}
\omega_{3x} = \cos\theta_3\,\omega_{2x} + \sin\theta_3\,\omega_{2y},\\
\omega_{3y} = -\sin\theta_3\,\omega_{2x} + \cos\theta_3\,\omega_{2y},\\
\omega_{3z} = \omega_{2z} + \dot{q}_3.
\end{cases}
\]


\subparagraph{Aceleración angular}
\[
\alpha_3 = R_3^\top\,\alpha_2 + \ddot{q}_3\,z_3 + \dot{q}_3\,(z_3 \times (R_3^\top\,\omega_2)),
\]

con

\[
R_3^\top\,\omega_2 =
\begin{bmatrix}
\cos\theta_3\,\omega_{2x} + \sin\theta_3\,\omega_{2y}\\
-\sin\theta_3\,\omega_{2x} + \cos\theta_3\,\omega_{2y}\\
\omega_{2z}
\end{bmatrix}.
\]

\[
\Rightarrow
\begin{cases}
\alpha_{3x} = \cos\theta_3\,\alpha_{2x} + \sin\theta_3\,\alpha_{2y} - \dot{q}_3(-\sin\theta_3\,\omega_{2x} + \cos\theta_3\,\omega_{2y}),\\
\alpha_{3y} = -\sin\theta_3\,\alpha_{2x} + \cos\theta_3\,\alpha_{2y} + \dot{q}_3(\cos\theta_3\,\omega_{2x} + \sin\theta_3\,\omega_{2y}),\\
\alpha_{3z} = \alpha_{2z} + \ddot{q}_3.
\end{cases}
\]

\subparagraph{Aceleración lineal del origen del marco 3}
Con $p_3 = [0.28948866, 0, 0]\; \text{m}$ se obtiene

\[
a_3 = R_3^\top\left( a_2 + \alpha_2 \times p_3 + \omega_2 \times (\omega_2 \times p_3) \right).
\]


Expandiendo:
\[
\alpha_2 \times p_3 =
\begin{bmatrix}
0\\ \alpha_{2z}\,0.28948866\\ -\alpha_{2y}\,0.28948866
\end{bmatrix},\qquad
\omega_2 \times (\omega_2 \times p_3) =
\begin{bmatrix}
-0.28948866(\omega_{2y}^2+\omega_{2z}^2)\\
0.28948866\,\omega_{2x}\omega_{2y}\\
0.28948866\,\omega_{2x}\omega_{2z}
\end{bmatrix}.
\]

\[
\Rightarrow
a_3 = R_3^\top
\begin{bmatrix}
a_{2x} - 0.28948866(\omega_{2y}^2+\omega_{2z}^2)\\
a_{2y} + 0.28948866\,\alpha_{2z} + 0.28948866\,\omega_{2x}\omega_{2y}\\
a_{2z} - 0.28948866\,\alpha_{2y} + 0.28948866\,\omega_{2x}\omega_{2z}
\end{bmatrix}.
\]



\subparagraph{Aceleración del centro de masa}
Finalmente, con \(r_{c_3}=[r_x,r_y,r_z]^\top\) (coordenadas del centro de masa del eslabón 3,
figura \ref{fig:dinamica_param}):
\[
a_{c_3} = a_3 + \alpha_3 \times r_{c_3} + \omega_3 \times (\omega_3 \times r_{c_3}).
\]






\paragraph{Fuerzas y momentos en el centro de masa}
Con la masa \(m_3\) y el tensor de inercia en el \textit{CoM} \(I_{c_3}\):


\[
F_3 = m_3\,a_{c_3},
\qquad
N_3 = I_{c_3}\,\alpha_3 + \omega_3 \times (I_{c_3}\,\omega_3).
\]


\paragraph{Barrido hacia atrás}
Contribuciones de los eslabones posteriores (\(k=4,5,6\)) ya acumuladas y expresadas en el marco \(3\) mediante:
\[
f_3 = F_3 + \sum_{k=4}^{6} R_k\,f_k,
\]

\[
n_3 = N_3 + r_{c_3} \times F_3 + \sum_{k=4}^{6} \left( R_k\,n_k + r_{3k} \times (R_k\,f_k) \right),
\]
donde:

\begin{itemize}
  \item \(R_k\): matriz de rotación que transforma vectores desde el marco del eslabón \(k\) al marco del eslabón 3. 
  Matemáticamente, \(R_k = {}^{3}\!R_k = ({}^{0}\!R_3)^\top \, {}^{0}\!R_k\).
  \item \(r_{3k}\): vector de posición desde el origen del marco 3 hasta el origen del marco \(k\), expresado en el marco 3. 
  Se obtiene como \(r_{3k} = ({}^{0}\!R_3)^\top \left( {}^{0}\!p_k - {}^{0}\!p_3 \right)\).
\end{itemize}

Cada término \(R_k f_k\) representa la fuerza del eslabón \(k\) expresada en el marco 3, y 
\(r_{3k} \times (R_k f_k)\) es el momento adicional debido a la aplicación de esa fuerza a una distancia \(r_{3k}\) del origen del marco 3.



\paragraph{Proyección sobre el eje articular}
El par en la articulación \(3\) se obtiene proyectando el momento sobre su eje:


\[
\tau_3 = n_3^\top\,z_3,
\]


y, si se modela fricción:


\[
\tau_3 \leftarrow \tau_3 + f_{v,3}\,\dot{q}_3 + f_{c,3}\,\mathrm{sgn}(\dot{q}_3).
\]

\clearpage
\subsection{Control dinámico}
Para llevar a cabo el control dinámico del brazo robótico es necesario combinar el estudio
cinemático y dinámico presentado, tal que los pasos a seguir son:

\begin{enumerate}
    \item Calcular la cinemática directa y diferencial.
    \item Calcular la cinemática inversa y generar trayectorias articulares.
    \item Calcular los pares necesarios en cada articulación.
    \item Recalcular la trayectoria o escalar temporalmente si los pares superan los valores máximos.
\end{enumerate}

En este sentido, se ha programado en \textit{Matlab}, junto con la \textit{Robotics Toolbox} de Peter Corke, 
el control dinámico completo del robot \textit{UFactory xArm6}. El programa utiliza los parámetros dinámicos
del modelo (masas, centros de masa e inercias) extraídos del URDF oficial, y permite calcular tanto la
dinámica inversa (torques articulares $\tau$ dados $q,\dot{q},\ddot{q}$) como la dinámica directa
(aceleraciones articulares $\ddot{q}$ dadas $q,\dot{q},\tau$). \\

También es necesario introducir los siguientes parámetros dinámicos de cada link para que 
\textit{Matlab} pueda calcular el par necesario:
\begin{itemize}
    \item \textbf{$J_m$}: inercia del rotor del motor reflejada en el eje de la articulación.
    Este término representa la resistencia del motor a cambios de velocidad angular. En muchos modelos
    simplificados se toma $J_m = 0$ al ser despreciable frente a la inercia de los eslabones.

    \item \textbf{$G$}: relación de transmisión entre el motor y el eje de salida. Un valor
    elevado de $G$ corresponde a reductores armónicos o planetarios que multiplican el par y reducen la
    velocidad. En simulaciones básicas se suele fijar $G=1$.

    \item \textbf{$B$}: coeficiente de fricción viscosa en la articulación, modelado como
    un par proporcional a la velocidad angular ($\tau_B = B \cdot \dot{q}$). Representa pérdidas por rozamiento
    fluido o lubricación interna.

    \item \textbf{$T_c$}: par de fricción de tipo Coulomb, independiente de la velocidad
    y dependiente únicamente del sentido de giro. Se define como un vector $T_c = [T_c^+, T_c^-]$ que
    especifica el par de fricción en movimiento positivo y negativo respectivamente.
\end{itemize}

Estos parámetros permiten extender el modelo dinámico clásico añadiendo los términos de fricción
y transmisión \cite{siciliano2009robotics}, de modo que la ecuación completa se expresa como:
\[
\tau = M(q)\ddot{q} + C(q,\dot{q})\dot{q} + g(q) + B\dot{q} + T_c.
\]

El fabricante no aporta estos valores, y aunque se pueden obtener experimentalmente, 
se va a utilizar la ecuación dinámica simple al igual que se emplea en el simulador \textit{Gazebo}, 
que se implementar en el código \ref{lst:din_control}.  que 
calcula los pares de los motores necesarios a aplicar. En la salida del programa, \ref{lst:din_control2},
se observa que el par del tercer motor es superior a su par máximo: $|\tau_3(t)| > \tau_{3,\max}$. \\
\begin{comment}
    pero se puede realizar una estimación teórica, 
\ref{tab:dynparams}, a partir de la tabla \ref{tab:actuators} \cite{siciliano2009robotics, craig2005}:

\begin{itemize}
    \item \textbf{Inercia del motor ($J_m$)}: se estima a partir de la potencia y el par máximo
    de cada articulación. La relación básica es
    

\[
    J_m \approx \frac{\tau_{\max}}{\alpha_{\max}},
    \]


    donde $\tau_{\max}$ es el par máximo y $\alpha_{\max}$ la aceleración angular máxima.
    Con $\tau_{\max} = 8.4\,\text{Nm}$ y $\alpha_{\max} = 1500^\circ/\text{s}^2 \approx 26.2\,\text{rad/s}^2$,
    se obtiene $J_m \approx 3.2 \times 10^{-1}\,\text{kg·m}^2$ reflejado en el eje.

    \item \textbf{Relación de transmisión ($G$)}: los reductores armónicos suelen tener ratios
    elevados (50:1 -- 100:1). En ausencia de datos exactos, se asignan valores aproximados
    ($G=50$--$100$) según el par máximo de cada articulación. \\

    \item \textbf{Fricción viscosa ($B$)}: se aproxima como una fracción del par máximo dividido
    por la velocidad máxima, siendo este un caso límite:
    

\[
    B \approx \frac{\tau_{\max}}{\dot{q}_{\max}}.
    \]


    Para $\tau_{\max} = 8.4\,\text{Nm}$ y $\dot{q}_{\max} = 180^\circ/\text{s} \approx 3.14\,\text{rad/s}$,
    se obtiene $B \approx 2.7\,\text{N·m·s/rad}$. En la práctica, la fricción real es mucho menor,
    por lo que se usan valores reducidos ($10^{-3}$--$10^{-2}$) para simulación. \\

    \item \textbf{Fricción Coulomb ($T_c$)}: se estima como un pequeño porcentaje del par máximo
    (5--10\%), representando el par necesario para iniciar el movimiento:
    

\[
    T_c \approx 0.05 \cdot \tau_{\max}.
    \]


    Así, para $\tau_{\max} = 8.4\,\text{Nm}$ se obtiene $T_c \approx 0.42\,\text{Nm}$.
\end{itemize}

\begin{table}[htb]
\centering
\begin{tabular}{|c|c|c|c|c|c|c|}
\hline
\rowcolor[HTML]{000000} 
{\color[HTML]{FFFFFF} \textbf{Parámetro}} & {\color[HTML]{FFFFFF} \textbf{Joint 1}} & {\color[HTML]{FFFFFF} \textbf{Joint 2}} & {\color[HTML]{FFFFFF} \textbf{Joint 3}} & {\color[HTML]{FFFFFF} \textbf{Joint 4}} & {\color[HTML]{FFFFFF} \textbf{Joint 5}} & {\color[HTML]{FFFFFF} \textbf{Joint 6}} \\ \hline

\cellcolor[HTML]{000000}{\color[HTML]{FFFFFF} \textbf{$J_m$ (kg·m$^2$)}} 
& $3.2 \times 10^{-1}$ & $3.2 \times 10^{-1}$ & $1.6 \times 10^{-1}$ & $5 \times 10^{-2}$ & $5 \times 10^{-2}$ & $5 \times 10^{-2}$ \\ \hline

\cellcolor[HTML]{000000}{\color[HTML]{FFFFFF} \textbf{$G$ (ratio)}} 
& 80 & 80 & 60 & 100 & 100 & 100 \\ \hline

\cellcolor[HTML]{000000}{\color[HTML]{FFFFFF} \textbf{$B$ (N·m·s/rad)}} 
& 2.68 & 2.68 & 1.34 & 0.48 & 0.48 & 0.48 \\ \hline

\cellcolor[HTML]{000000}{\color[HTML]{FFFFFF} \textbf{$T_c$ (N·m)}} 
& $\pm 0.42$ & $\pm 0.42$ & $\pm 0.21$ & $\pm 0.075$ & $\pm 0.075$ & $\pm 0.075$ \\ \hline

\end{tabular}
\caption{Parámetros dinámicos aproximados aplicados al \textit{xArm 6} para simulación.}
\label{tab:dynparams}
\end{table}

NOTA: en la \textit{toolbox} de Peter Corke, los parámetros dinámicos de fricción e inercia
se definen en el lado del motor. Al existir una relación de transmisión $G$ entre el motor y la
articulación, dichos parámetros se reflejan en el eje de salida de la siguiente forma:

\begin{itemize}
    \item \textbf{Fricción viscosa:}
    

\[
        B_{\text{art}} = B_{\text{mot}} \cdot G^2
    \]


    donde $B_{\text{mot}}$ es el coeficiente de fricción viscosa en el rotor del motor, y $B_{\text{art}}$ es el valor efectivo en la articulación. 
    El factor $G^2$ aparece porque la velocidad angular se reduce por $G$, mientras que el par se amplifica por $G$.

    \item \textbf{Fricción Coulomb:}
    

\[
        T_{c,\text{art}} = T_{c,\text{mot}} \cdot G
    \]


    donde $T_{c,\text{mot}}$ es el par de fricción en el motor y $T_{c,\text{art}}$ el reflejado en la articulación. 
    Aquí sólo interviene un factor $G$, ya que el par se amplifica linealmente con la transmisión.

    \item \textbf{Inercia del rotor:}
    

\[
        J_{\text{ref}} = J_m \cdot G^2
    \]


    donde $J_m$ es la inercia del rotor del motor, y $J_{\text{ref}}$ es la inercia reflejada en el eje de salida. 
    El factor $G^2$ aparece porque la energía cinética depende cuadráticamente de la velocidad angular.
\end{itemize}
\end{comment} 

Esto se debe principalmente 
a que únicamente se han introducido valores arbitrarios de velocidades, aceleraciones y posiciones 
finales sin calcular correctamente la trayectoria como en el control cinemático. 
Por tanto, en este caso la trayectoria planificada no es físicamente realizable en el tiempo nominal,
pues el motor no puede generar el par necesario. La solución consiste en aplicar un
escalado temporal de la trayectoria, aumentando el tiempo total de ejecución y reduciendo
así las aceleraciones articulares. De este modo, el par requerido en cada articulación también
disminuye y se mantiene dentro de los límites físicos del motor. La articulación que alcanza antes
su límite de par es la que determina el nuevo tiempo mínimo necesario para completar el movimiento,
y el resto de articulaciones deben ajustarse para avanzar de manera coordinada, garantizando que
todas lleguen al objetivo simultáneamente. \\

Finalmente, es necesario diseñar un controlador, habitualmente PID \cite{barrientos2007fundamentos},
que aplicado a cada articulación combina el modelo cinemático y dinámico del robot para generar
las señales de mando adecuadas. Este tipo de control permite corregir errores de posición y
velocidad mediante la acción proporcional, integral y derivativa, garantizando que la trayectoria
planificada se ejecute de forma estable y coordinada. \\

No obstante, el diseño de estos controladores 
queda fuera del alcance de esta asignatura, ya que requiere un desarrollo más profundo con teoría 
de regulación automática. Además, el repositorio oficial del fabricante \cite{xarm_ros2}
ya incorpora estos controladores. \\

\subsection{Escalado temporal de la trayectoria para garantizar el cumplimiento de los límites de par}

En un manipulador industrial, los actuadores suelen estar dimensionados para generar el par necesario frente
a la dinámica del robot en condiciones nominales. Sin embargo, durante la planificación de trayectorias es
posible que el par teórico obtenido a partir del modelo dinámico supere el par máximo admisible por los motores.
Cuando esto ocurre, la trayectoria debe ser modificada para asegurar que se cumple la restricción:

\[
\forall i \in \{1,\dots,6\},\ \forall t \in [0,T], \qquad |\tau_i(t)| \le \tau_{i,\max}.
\]

Una estrategia clásica y ampliamente utilizada en la literatura consiste en aplicar un escalado temporal
de la trayectoria. Este enfoque, introducido originalmente en el contexto de la planificación dinámica por Bobrow
\cite{bobrow1985time} y posteriormente extendido por Shin y McKay~\cite{shin1986minimum}, se basa en
modificar la parametrización temporal de la trayectoria manteniendo su forma geométrica. Si la trayectoria
original se parametriza mediante un tiempo $t$, el escalado consiste en definir un nuevo tiempo
$\tilde{t} = k\,t$, con $k > 1$, lo que reduce las velocidades y aceleraciones articulares según

\[
\dot{q}(t) \mapsto \frac{1}{k}\,\dot{q}(t), 
\qquad
\ddot{q}(t) \mapsto \frac{1}{k^2}\,\ddot{q}(t),
\]

y, en consecuencia, disminuye el par requerido por la ecuación dinámica (ec. \ref{eq:movimiento_estandar}). \\

En este trabajo se ha implementado un algoritmo de retiming iterativo basado en búsqueda binaria sobre el factor
de escalado $k$. El procedimiento evalúa repetidamente el par resultante para distintos valores de $k$ hasta
encontrar el mínimo escalado que garantiza el cumplimiento de los límites de par. Este enfoque es sencillo,
robusto y adecuado para trayectorias suaves generadas mediante interpolación polinómica. \\

No obstante, tal como se indica en la tabla~\ref{tab:actuators}, el fabricante no proporciona los valores
reales de par máximo de cada articulación, ya que el controlador comercial opera como una ``caja negra''.
Por este motivo, en la práctica no es posible aplicar un retiming basado en límites reales del actuador.
Aun así, se ha implementado el algoritmo con fines didácticos, utilizando valores estimados de $\tau_{\max}$ para
ilustrar el funcionamiento del método. Si se desea observar el efecto del escalado, basta con modificar el
vector de pares máximos en el código mostrado en el código \ref{lst:main}. El escalado se ha programado 
en el código \ref{lst:dynamic_replanner}.



\lstinputlisting[
    caption={Cálculo de pares de los actuadores.},
    label={lst:din_control},
    breaklines=true,
    breakatwhitespace=true
]{Matlab/din_control.m}
\lstinputlisting[
    caption={Salida del cálculo de pares de los actuadores.},
    label={lst:din_control2},
    breaklines=true,
    breakatwhitespace=true
]{Matlab/din_control_salida.m}

