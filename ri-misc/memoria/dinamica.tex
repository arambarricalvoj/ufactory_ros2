\datos{color}{}{}

\section{Estudio dinámico}
El fabricante proporciona las características físicas de cada eslabón, 
en particular la masa y la posición de su centro de masa (\textit{CoM}), figura \ref{fig:dinamica_param},
respecto de los sistemas de referencia de la figura \ref{fig:dinamica_sist} con la tabla de parámetros 
\ref{tab:dh_modificada_din}. También proporcionan en los modelos de simulación los momentos de inercia \cite{Spong2006} 
de cada eslabón, expuestos en el código \ref{lst:inercia}. El modelo del robot con el que se trabaja 
corresponde al modelo 1 del robot, que se alimenta a través de la \textit{AC Control Box}.

\begin{figure}[htb]
    \centering
    \begin{minipage}[t]{0.3\textwidth}
        \centering
        \includegraphics[scale=0.375]{Images/dinamica/sistemas.png}
        \caption{Sistemas de referencia para el estudio dinámico. Fuente: \cite{ufactory_xarm_manual}.}
        \label{fig:dinamica_sist}
    \end{minipage}
    \begin{minipage}[t]{0.65\textwidth}
        \centering
        \includegraphics[scale=0.315]{Images/dinamica/tabla.png}
        \caption{Parámetros de la dinámica del robot. Fuente: \cite{ufactory_xarm_manual}.}
        \label{fig:dinamica_param}
    \end{minipage}
\end{figure}

\vspace{-0.3cm}
\begin{table}[htb]
\centering
\begin{tabular}{c c c c c}
\hline
\textbf{$i$} & \textbf{$\theta_i$} ($^\circ$) & \textbf{$d_i$} (mm) & \textbf{$a_{i-1}$} (mm) & \textbf{$\alpha_{i-1} (^\circ)$} \\
\hline
1 & $\theta_1$         & $267$         & $0$         & $0$         \\
2 & $\theta_2$         & $0$         & $0$         & $-90$         \\
3 & $\theta_3 + (\textit{T2\_offset} = -79.34995)$         & $d_3$         & $a_2 = 289.48866$         & $0$         \\
4 & $\theta_4 + (\textit{T2\_offset} = 79.34995)$         & $342.5$         & $77.5$         & $-90$         \\
5 & $0$         & $0$         & $0$         & $90$         \\
6 & $0$         & $97$         & $76$         & $-90$         \\
\hline
\end{tabular}
\caption{Parámetros DH modificados del \textit{UFactory xArm6} para el estudio dinámico.}
\label{tab:dh_modificada_din}
\end{table}

\lstinputlisting[
    caption={Momentos de inercia del los eslabones del robot en estudio.},
    label={lst:inercia},
    breaklines=true,
    breakatwhitespace=true
]{Matlab/inercias.txt}


\newpage
\subsection{Ecuaciones de movimiento}

Las ecuaciones de movimiento de un robot manipulador son las expresiones matemáticas que describes cómo las
fuerzas y pares aplicados en las articulaciones producen el movimiento del sistema, teniendo en cuenta 
su masa, inercias, geometría y la acción de la gravedad. Se tendrá una ecuación de movimiento 
por cada eslabón y finalmente se obtiene el vector de pares articulares $\tau = [\tau_1, \tau_2, ..., 
\tau_6]\; N\cdot m$. La expresión general de la ecuación de movimiento es la siguiente:
\begin{equation} 
M(q)\,\ddot{q} + C(q,\dot{q})\,\dot{q} + g(q) = \tau 
\label{eq:movimiento_estandar} 
\end{equation}

donde:
\begin{itemize}
    \item \(M(q)\) es la matriz de inercia (simétrica definida positiva).
    \item \(C(q,\dot{q})\,\dot{q}\) agrupa términos centrífugos y de Coriolis.
    \item \(g(q)\) es el vector de gravedad.
    %\item \(F_v\,\dot{q}\) y \(F_c\,\mathrm{sgn}(\dot{q})\) modelan la fricción viscosa y de Coulomb, respectivamente.
    \item \(\tau\) son los pares actuados en las articulaciones.
    %\item \(J(q)\) es el jacobiano geométrico del efector final y \(f_{\text{ext}}\) fuerzas/momentos externos aplicados.
\end{itemize}

En \cite{siciliano2009robotics} se extiende la ecuación de movimiento para añadir las fricciones viscosa
\(F_v\,\dot{q}\) y de Coulomb \(F_c\,\mathrm{sgn}(\dot{q})\) y las fuerzas externas, donde 
\(J(q)\) es el jacobiano geométrico del efector final y \(f_{\text{ext}}\) las fuerzas externas aplicadas.
Esta versión extendida se presenta a continuación: 
\[
M(q)\,\ddot{q} + C(q,\dot{q})\,\dot{q} + g(q) + F_v\,\dot{q} + F_c\,\mathrm{sgn}(\dot{q}) = \tau + J^\top(q)\,f_{\text{ext}}
\]

Como el fabricante no proporciona los datos de friccioón y se va a trabajar en simulación, 
únicamente se estudia la expresión general. \\

Las ecuaciones de movimiento pueden calcularse mediante el balance de energías Lagrangiano
\cite{siciliano2009robotics} o por el método recursivo de Newton–Euler \cite{barrientos2007fundamentos}, siendo este último
el más habitual por ser computacionalmente más eficiente.

\subsubsection{Enfoque energético (Lagrangiano)}

En términos energéticos, los componentes pueden obtenerse a partir del lagrangiano \(\mathcal{L}=T-V\), 
donde \(T\) es la energía cinética total del manipulador y \(V\) la energía potencial gravitatoria.

La energía cinética se calcula como:


\[
T = \frac{1}{2}\sum_{i=1}^{n}\left( m_i\,\dot{p}_{c_i}^\top \dot{p}_{c_i} 
+ \omega_i^\top I_{c_i}\,\omega_i \right),
\]


donde \(m_i\) es la masa del eslabón \(i\), \(\dot{p}_{c_i}\) la velocidad del centro de masa del eslabón \(i\), 
\(\omega_i\) la velocidad angular del eslabón \(i\) e \(I_{c_i}\) su tensor de inercia respecto al centro de masa.\\

La energía potencial gravitatoria se expresa como:
\[
V = \sum_{i=1}^{n} m_i\,g^\top p_{c_i},
\]
donde \(g\) es el vector de aceleración de la gravedad y \(p_{c_i}\) la posición del centro de masa del eslabón \(i\). \\

A partir de estas expresiones, se obtienen los componentes de la ecuación de movimiento (ec. \ref{eq:movimiento_estandar}):
\[
M_{ij}(q) = \frac{\partial^2 T}{\partial \dot{q}_i \,\partial \dot{q}_j}, 
\qquad
g_i(q) = \frac{\partial V}{\partial q_i},
\]
y los elementos de \(C(q,\dot{q})\) mediante los símbolos de Christoffel:

\[
c_{ijk} = \frac{1}{2}\left(\frac{\partial M_{ij}}{\partial q_k} + \frac{\partial M_{ik}}{\partial q_j} - \frac{\partial M_{jk}}{\partial q_i}\right),
\qquad
[C(q,\dot{q})]_{ij} = \sum_{k=1}^{n} c_{ijk}\,\dot{q}_k.
\] \\


\subsubsection{Método recursivo de Newton--Euler}
Mientras que el enfoque Lagrangiano produce explícitamente \(M(q), C(q,\dot{q}), g(q)\),
Newton--Euler es más eficiente computacionalmente para evaluar \(\tau\) dado \(\{q,\dot{q},\ddot{q}\}\)
y es el más usado en simulación y control en tiempo real. \\

El método recursivo de Newton--Euler calcula directamente los pares articulares \(\tau_i\)
a partir de \(\{q,\dot{q},\ddot{q}\}\) (variables articulares) y los parámetros dinámicos \(\{m_i, r_{c_i}, I_{c_i}\}\)
(masas, centros de masa e inercias), mediante un barrido hacia adelante (cinemática y aceleraciones) 
y otro hacia atrás (acumulación de fuerzas y momentos).

\paragraph{Barrido hacia adelante}
Condiciones en la base:
\[
\omega_0 = 0,\quad \alpha_0 = 0,\quad a_0 = -g.
\]

Para \(i=1,\dots,n\), con \(R_i\) y \(p_i\) de las transformaciones homogéneas y \(z_i\) el eje articular:
\[
\omega_i = R_i^\top \omega_{i-1} + \dot{q}_i z_i,
\]

\[
\alpha_i = R_i^\top \alpha_{i-1} + \ddot{q}_i z_i + \dot{q}_i z_i \times (R_i^\top \omega_{i-1}),
\]

\[
a_i = R_i^\top \left( a_{i-1} + \alpha_{i-1} \times p_i + \omega_{i-1} \times (\omega_{i-1} \times p_i) \right),
\]

\[
a_{c_i} = a_i + \alpha_i \times r_{c_i} + \omega_i \times (\omega_i \times r_{c_i}).
\]

\paragraph{Fuerzas y momentos en el centro de masa}
\[
F_i = m_i a_{c_i}, \qquad
N_i = I_{c_i} \alpha_i + \omega_i \times (I_{c_i} \omega_i).
\]


\paragraph{Barrido hacia atrás}
Acumulación desde el efector hasta la base. Para \(i=n,\dots,1\):
\[
f_i = F_i + \sum_{k=i+1}^{n} R_k f_k,
\]

\[
n_i = N_i + r_{c_i} \times F_i + \sum_{k=i+1}^{n} \left( R_k n_k + r_{ik} \times (R_k f_k) \right),
\]

donde \(r_{ik}\) es el vector del origen \(i\) al origen \(k\).

\paragraph{Proyección sobre el eje articular}
Para articulación rotacional:
\[
\tau_i = n_i^\top z_i,
\]

y opcionalmente se añaden fricciones \cite{siciliano2009robotics}:
\[
\tau_i \leftarrow \tau_i + f_{v,i} \dot{q}_i + f_{c,i} \mathrm{sgn}(\dot{q}_i).
\] \\

\subsubsection{Ejemplo de una iteración para el eslabón 3 del \textit{UFactory xArm6}}

A modo ilustrativo, se muestra una iteración completa del método recursivo de Newton--Euler
para el eslabón $i=3$ del robot en estudio, utilizando la tabla \ref{tab:dh_modificada_din} y 
las matrices \(A_i\) calculadas en el apartado \ref{sec:dh-mod}.

\paragraph{Transformaciones homogéneas y descomposición en rotación y traslación}
\[
d_1 = 267,\qquad a_2 = 289.48866,\qquad a_3 = 77.5,\qquad d_4 = 342.5,\qquad a_5 = 76,\qquad d_6 = 97.
\]

Para el cálculo en el barrido hacia adelante del eslabón \(3\), necesitamos
\({}^{0}\!A_1\), \({}^{1}\!A_2\) y \({}^{2}\!A_3\). Denotando \(C_i=\cos(\cdot)\) y \(S_i=\sin(\cdot)\):

\[
A_1(q_1) =
\begin{bmatrix}
\cos q_1 & -\sin q_1 & 0 & 0 \\
\sin q_1 & \cos q_1 & 0 & 0 \\
0 & 0 & 1 & d_1 \\
0 & 0 & 0 & 1
\end{bmatrix}
\Rightarrow
{}^{0}\!R_1=
\begin{bmatrix}
\cos q_1 & -\sin q_1 & 0 \\
\sin q_1 & \cos q_1 & 0 \\
0 & 0 & 1
\end{bmatrix},\quad
{}^{0}\!p_1=\begin{bmatrix}0\\0\\d_1\end{bmatrix}.
\]

\[
A_2(\theta_2 = q_2+T2_{\text{offset}}) =
\begin{bmatrix}
\cos \theta_2 & -\sin \theta_2 & 0 & 0 \\
0 & 0 & 1 & 0 \\
-\,\sin \theta_2 & -\,\cos \theta_2 & 0 & 0 \\
0 & 0 & 0 & 1
\end{bmatrix},
\Rightarrow
{}^{1}\!R_2=
\begin{bmatrix}
\cos \theta_2 & -\sin \theta_2 & 0 \\
0 & 0 & 1 \\
-\,\sin \theta_2 & -\,\cos \theta_2 & 0
\end{bmatrix},\quad
{}^{1}\!p_2=\begin{bmatrix}0\\0\\0\end{bmatrix}.
\]

\[
A_3(\theta_3 = q_3+T3_{\text{offset}}) =
\begin{bmatrix}
\cos \theta_3 & -\sin \theta_3 & 0 & a_2 \\
\sin \theta_3 & \cos \theta_3 & 0 & 0 \\
0 & 0 & 1 & 0 \\
0 & 0 & 0 & 1
\end{bmatrix},
\Rightarrow
{}^{2}\!R_3=
\begin{bmatrix}
\cos \theta_3 & -\sin \theta_3 & 0 \\
\sin \theta_3 & \cos \theta_3 & 0 \\
0 & 0 & 1
\end{bmatrix},\quad
{}^{2}\!p_3=\begin{bmatrix}a_2\\0\\0\end{bmatrix}.
\]



La pose del marco \(3\) respecto a la base se obtiene con:


\[
{}^{0}\!T_3 = {}^{0}\!A_1\,{}^{1}\!A_2\,{}^{2}\!A_3,\qquad
{}^{0}\!R_3 = {}^{0}\!R_1\,{}^{1}\!R_2\,{}^{2}\!R_3,\qquad
{}^{0}\!p_3 = {}^{0}\!p_1 + {}^{0}\!R_1\,{}^{1}\!p_2 + {}^{0}\!R_1\,{}^{1}\!R_2\,{}^{2}\!p_3.
\]

\[
{}^{0}\!R_3 = {}^{0}\!R_1\,{}^{1}\!R_2\,{}^{2}\!R_3
=
\begin{bmatrix}
\cos q_1 & -\sin q_1 & 0 \\
\sin q_1 & \cos q_1 & 0 \\
0 & 0 & 1
\end{bmatrix}
\!
\begin{bmatrix}
\cos \theta_2 & -\sin \theta_2 & 0 \\
0 & 0 & 1 \\
-\,\sin \theta_2 & -\,\cos \theta_2 & 0
\end{bmatrix}
\!
\begin{bmatrix}
\cos \theta_3 & -\sin \theta_3 & 0 \\
\sin \theta_3 & \cos \theta_3 & 0 \\
0 & 0 & 1
\end{bmatrix}.
\]

\[
{}^{0}\!R_3 = {}^{0}\!R_2\,{}^{2}\!R_3
=
\begin{bmatrix}
\cos q_1 \cos \theta_2 & -\cos q_1 \sin \theta_2 & -\sin q_1 \\
\sin q_1 \cos \theta_2 & -\sin q_1 \sin \theta_2 & \cos q_1 \\
-\,\sin \theta_2 & -\,\cos \theta_2 & 0
\end{bmatrix}
\!
\begin{bmatrix}
\cos \theta_3 & -\sin \theta_3 & 0 \\
\sin \theta_3 & \cos \theta_3 & 0 \\
0 & 0 & 1
\end{bmatrix}.
\]

\[
{}^{0}\!R_3 =
\begin{bmatrix}
\cos q_1(\cos \theta_2 \cos \theta_3 - \sin \theta_2 \sin \theta_3)
&
\cos q_1(-\cos \theta_2 \sin \theta_3 - \sin \theta_2 \cos \theta_3)
&
-\sin q_1
\\
\sin q_1(\cos \theta_2 \cos \theta_3 - \sin \theta_2 \sin \theta_3)
&
\sin q_1(-\cos \theta_2 \sin \theta_3 - \sin \theta_2 \cos \theta_3)
&
\cos q_1
\\
-(\sin \theta_2 \cos \theta_3 + \cos \theta_2 \sin \theta_3)
&
\sin \theta_2 \sin \theta_3 - \cos \theta_2 \cos \theta_3
&
0
\end{bmatrix}.
\]





\[
{}^{0}\!p_3 = {}^{0}\!p_1 + {}^{0}\!R_1\,{}^{1}\!p_2 + {}^{0}\!R_1\,{}^{1}\!R_2\,{}^{2}\!p_3
= 
\begin{bmatrix}0\\0\\267\end{bmatrix}
+
\underbrace{{}^{0}\!R_1\,\begin{bmatrix}0\\0\\0\end{bmatrix}}_{=\,\mathbf{0}}
+
\left({}^{0}\!R_1\,{}^{1}\!R_2\right)\begin{bmatrix}289.48866\\0\\0\end{bmatrix}.
\]



Como \({}^{0}\!R_1\,{}^{1}\!R_2\) tiene primera columna
\(\big[\cos q_1 \cos \theta_2,\; \sin q_1 \cos \theta_2,\; -\sin \theta_2\big]^\top\),
se obtiene:


\[
{}^{0}\!p_3 =
\begin{bmatrix}
289.48866\,\cos q_1 \cos \theta_2 \\
289.48866\,\sin q_1 \cos \theta_2 \\
267 - 289.48866\,\sin \theta_2
\end{bmatrix}.
\]





\[
{}^{0}\!T_3 =
\begin{bmatrix}
{}^{0}\!R_3 & {}^{0}\!p_3 \\
\mathbf{0}_{1\times 3} & 1
\end{bmatrix}.
\]





\paragraph{Barrido hacia adelante}

Tomamos el eje articular local $z_3 = [0, 0, 1]^{T}$ y convertimos las longitudes a unidades 
del Sistema Internacional: $d_1 = 0.267 \;\text{m},\; a_2 = 0.28948866 \;\text{m}$.


\subparagraph{Velocidad angular}
\[
\omega_3 = R_3^\top\,\omega_2 + \dot{q}_3\,z_3
=
\begin{bmatrix}
\cos\theta_3 & \sin\theta_3 & 0 \\
-\sin\theta_3 & \cos\theta_3 & 0 \\
0 & 0 & 1
\end{bmatrix}
\begin{bmatrix}\omega_{2x}\\\omega_{2y}\\\omega_{2z}\end{bmatrix}
+
\dot{q}_3\begin{bmatrix}0\\0\\1\end{bmatrix}.
\]

\[
\Rightarrow
\begin{cases}
\omega_{3x} = \cos\theta_3\,\omega_{2x} + \sin\theta_3\,\omega_{2y},\\
\omega_{3y} = -\sin\theta_3\,\omega_{2x} + \cos\theta_3\,\omega_{2y},\\
\omega_{3z} = \omega_{2z} + \dot{q}_3.
\end{cases}
\]


\subparagraph{Aceleración angular}
\[
\alpha_3 = R_3^\top\,\alpha_2 + \ddot{q}_3\,z_3 + \dot{q}_3\,(z_3 \times (R_3^\top\,\omega_2)),
\]

con

\[
R_3^\top\,\omega_2 =
\begin{bmatrix}
\cos\theta_3\,\omega_{2x} + \sin\theta_3\,\omega_{2y}\\
-\sin\theta_3\,\omega_{2x} + \cos\theta_3\,\omega_{2y}\\
\omega_{2z}
\end{bmatrix}.
\]

\[
\Rightarrow
\begin{cases}
\alpha_{3x} = \cos\theta_3\,\alpha_{2x} + \sin\theta_3\,\alpha_{2y} - \dot{q}_3(-\sin\theta_3\,\omega_{2x} + \cos\theta_3\,\omega_{2y}),\\
\alpha_{3y} = -\sin\theta_3\,\alpha_{2x} + \cos\theta_3\,\alpha_{2y} + \dot{q}_3(\cos\theta_3\,\omega_{2x} + \sin\theta_3\,\omega_{2y}),\\
\alpha_{3z} = \alpha_{2z} + \ddot{q}_3.
\end{cases}
\]

\subparagraph{Aceleración lineal del origen del marco 3}
Con $p_3 = [0.28948866, 0, 0]\; \text{m}$ se obtiene

\[
a_3 = R_3^\top\left( a_2 + \alpha_2 \times p_3 + \omega_2 \times (\omega_2 \times p_3) \right).
\]


Expandiendo:
\[
\alpha_2 \times p_3 =
\begin{bmatrix}
0\\ \alpha_{2z}\,0.28948866\\ -\alpha_{2y}\,0.28948866
\end{bmatrix},\qquad
\omega_2 \times (\omega_2 \times p_3) =
\begin{bmatrix}
-0.28948866(\omega_{2y}^2+\omega_{2z}^2)\\
0.28948866\,\omega_{2x}\omega_{2y}\\
0.28948866\,\omega_{2x}\omega_{2z}
\end{bmatrix}.
\]

\[
\Rightarrow
a_3 = R_3^\top
\begin{bmatrix}
a_{2x} - 0.28948866(\omega_{2y}^2+\omega_{2z}^2)\\
a_{2y} + 0.28948866\,\alpha_{2z} + 0.28948866\,\omega_{2x}\omega_{2y}\\
a_{2z} - 0.28948866\,\alpha_{2y} + 0.28948866\,\omega_{2x}\omega_{2z}
\end{bmatrix}.
\]



\subparagraph{Aceleración del centro de masa}
Finalmente, con \(r_{c_3}=[r_x,r_y,r_z]^\top\) (coordenadas del centro de masa del eslabón 3,
figura \ref{fig:dinamica_param}):
\[
a_{c_3} = a_3 + \alpha_3 \times r_{c_3} + \omega_3 \times (\omega_3 \times r_{c_3}).
\]






\paragraph{Fuerzas y momentos en el centro de masa}
Con la masa \(m_3\) y el tensor de inercia en el \textit{CoM} \(I_{c_3}\):


\[
F_3 = m_3\,a_{c_3},
\qquad
N_3 = I_{c_3}\,\alpha_3 + \omega_3 \times (I_{c_3}\,\omega_3).
\]


\paragraph{Barrido hacia atrás}
Contribuciones de los eslabones posteriores (\(k=4,5,6\)) ya acumuladas y expresadas en el marco \(3\) mediante:
\[
f_3 = F_3 + \sum_{k=4}^{6} R_k\,f_k,
\]

\[
n_3 = N_3 + r_{c_3} \times F_3 + \sum_{k=4}^{6} \left( R_k\,n_k + r_{3k} \times (R_k\,f_k) \right),
\]
donde:

\begin{itemize}
  \item \(R_k\): matriz de rotación que transforma vectores desde el marco del eslabón \(k\) al marco del eslabón 3. 
  Matemáticamente, \(R_k = {}^{3}\!R_k = ({}^{0}\!R_3)^\top \, {}^{0}\!R_k\).
  \item \(r_{3k}\): vector de posición desde el origen del marco 3 hasta el origen del marco \(k\), expresado en el marco 3. 
  Se obtiene como \(r_{3k} = ({}^{0}\!R_3)^\top \left( {}^{0}\!p_k - {}^{0}\!p_3 \right)\).
\end{itemize}

Cada término \(R_k f_k\) representa la fuerza del eslabón \(k\) expresada en el marco 3, y 
\(r_{3k} \times (R_k f_k)\) es el momento adicional debido a la aplicación de esa fuerza a una distancia \(r_{3k}\) del origen del marco 3.



\paragraph{Proyección sobre el eje articular}
El par en la articulación \(3\) se obtiene proyectando el momento sobre su eje:


\[
\tau_3 = n_3^\top\,z_3,
\]


y, si se modela fricción:


\[
\tau_3 \leftarrow \tau_3 + f_{v,3}\,\dot{q}_3 + f_{c,3}\,\mathrm{sgn}(\dot{q}_3).
\]

\clearpage
\subsection{Control dinámico}
Para llevar a cabo el control dinámico del brazo robótico es necesario combinar el estudio
cinemático y dinámico presentado, tal que los pasos a seguir son:

\begin{enumerate}
    \item Calcular la cinemática directa y diferencial.
    \item Calcular la cinemática inversa y generar trayectorias articulares.
    \item Propagar velocidades y aceleraciones mediante el barrido hacia adelante.
    \item Calcular fuerzas y momentos en cada eslabón.
    \item Acumular las contribuciones mediante el barrido hacia atrás.
    \item Proyectar los momentos sobre los ejes articulares para obtener los pares $\tau_i$.
\end{enumerate}

En este sentido, se ha programado en \textit{Matlab}, junto con la \textit{Robotics Toolbox} de Peter Corke, 
el control dinámico completo del robot \textit{UFactory xArm6}. El programa utiliza los parámetros dinámicos
del modelo (masas, centros de masa e inercias) extraídos del URDF oficial, y permite calcular tanto la
dinámica inversa (torques articulares $\tau$ dados $q,\dot{q},\ddot{q}$) como la dinámica directa
(aceleraciones articulares $\ddot{q}$ dadas $q,\dot{q},\tau$). \\

Al inicio, el programa solicita la posición articular de origen y la posición cartesiana a alcanzar,
junto con la duración en segundos de la trayectoria. La aplicación comprueba que la posición final es
alcanzable según el rango de trabajo del robot de $700$ mm, calcula las posiciones, velocidades y
aceleraciones articulares necesarias para alcanzar la posición cartesiana final mediante cinemática inversa
y genera la trayectoria correspondiente. A partir de estas trayectorias articulares, se aplica el algoritmo
de Newton--Euler para obtener los torques requeridos en cada articulación. \\

Las imágenes \ref{fig:din_control_tau}, \ref{fig:din_control_ddq} y \ref{fig:din_control_error}
muestran los torques articulares calculados, las aceleraciones articulares obtenidas por dinámica directa
y la comparación entre las aceleraciones de referencia y las reconstruidas. Los códigos 
\ref{lst:din_control} y \ref{lst:din_control2} recogen la implementación en \textit{Matlab} y la salida
por terminal. \\
