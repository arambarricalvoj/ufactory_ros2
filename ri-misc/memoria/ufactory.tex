\datos{color}{}{}

\section{\textit{UFactory xArm 6}}

El \textit{UFactory xArm 6}, figura \ref{fig:xarm6}, es un robot colaborativo de seis grados de libertad diseñado para aplicaciones de
investigación, formación y automatización industrial. Se trata de un manipulador compacto y versátil,
capaz de realizar movimientos complejos en tres dimensiones gracias a su configuración de seis articulaciones
rotacionales. %En el cuadro \ref{tab:resumen} se resumen las principales características del brazo y en las 
%siguiente secciones se detallan. \\
\vspace{-0.2cm}
\begin{figure}[htb]
    \centering
    \includegraphics[scale=0.15]{Images/ufactory/xarm6.jpg}
    \caption{\textit{UFactory xArm 6}. Fuente: \cite{ufactory_xarm}.}
    \label{fig:xarm6}
\end{figure}

\vspace{-0.5cm}
\subsection{Estructura mecánica: eslabones y articulaciones}
La estructura mecánica del robot se compone de un cuerpo ligero fabricado en aluminio y fibra de carbono, 
con un peso total cercano a los 12.5\,kg, lo que facilita su instalación en entornos de laboratorio o docencia.
Los distintos eslabones del brazo están formados por piezas huecas unidos mediante carcasas mecanizadas
que protegen los actuadores. El conjunto 
está diseñado para ofrecer rigidez estructural y al mismo tiempo mantener un peso reducido, lo que mejora la 
relación entre carga útil y masa propia del robot. \\

El manipulador consta de seis eslabones principales, 
correspondientes a las seis articulaciones rotacionales que le otorgan sus grados de libertad. Cada eslabón 
está conectado al siguiente mediante un eje motorizado con transmisión armónica y por tanto, cada articulación
está equipada con un motor eléctrico de corriente continua y un reductor armónico
de alta precisión. A pesar de tener los actuadores 
directamente en cada articulación, no se puede considerar de accionamiento directo debido al uso de
reductores armónicos en todas las articulaciones. \\%
%que aseguran un movimiento suave y preciso, además de minimizar el retroceso mecánico. \\

En cuanto a su apariencia, el robot se presenta en un acabado de color blanco 
con detalles en color oscuro en las juntas y carcasas de los motores, lo que le confiere un aspecto moderno y uniforme. \\

El rango de movimiento de las articulaciones, tabla \ref{tab:actuators}, abarca desde la base hasta la muñeca, 
con amplitudes que permiten cubrir un radio de trabajo de aproximadamente 700\,mm. En concreto, la articulación 
de la base ofrece un giro completo de $\pm 360^{\circ}$, mientras que las articulaciones intermedias permiten 
rotaciones de hasta $\pm 180^{\circ}$, y las articulaciones de muñeca alcanzan rangos de $\pm 360^{\circ}$, 
lo que proporciona gran flexibilidad para tareas de manipulación y orientación del efector final. \\

\subsection{Actuadores}
Cada articulación del \textit{xArm 6} está equipada con un motor eléctrico de corriente continua
combinado con un reductor armónico de alta precisión. Esta configuración permite alcanzar una elevada
rigidez torsional y eliminar prácticamente el retroceso mecánico, garantizando movimientos suaves y
precisos en tareas de manipulación y ensamblaje. Los motores, integrados directamente en cada eje,
proporcionan pares máximos que varían entre 1.5\,Nm en las articulaciones de la muñeca y hasta 8.4\,Nm
en las articulaciones de base y hombro, lo que asegura la capacidad de transportar cargas de hasta
5\,kg sin comprometer la repetibilidad del sistema. \\

La potencia nominal del conjunto es de 150\,W, distribuida entre las seis articulaciones, mientras que
la velocidad máxima de giro alcanza los 180º/s. Estos valores, junto con la repetibilidad de
$\pm 0.1$\,mm en sus trayectorias, hacen del \textit{xArm 6} un manipulador adecuado para aplicaciones de investigación,
docencia y procesos industriales ligeros que requieren gran precisión. En la tabla \ref{tab:actuators}
se resumen las principales características mecánicas y de actuadores, incluyendo el rango de trabajo
de cada articulación, el par máximo disponible y los parámetros globales de carga útil, potencia y
repetibilidad. \\

Los reductores armónicos \cite{articleReductor}, figura \ref{fig:reductor}, se basan en la deformación 
elástica controlada de un componente flexible para transmitir el movimiento. Este principio permite 
alcanzar relaciones de reducción muy elevadas en un volumen compacto, con una rigidez torsional superior 
a la de otros sistemas de engranajes. En el \textit{xArm 6}, su empleo asegura un movimiento suave y preciso, 
además de minimizar el retroceso mecánico (\textit{backlash}), lo que resulta fundamental para aplicaciones 
de ensamblaje, manipulación de piezas y tareas de investigación que requieren gran exactitud. 

Otra ventaja de los reductores armónicos es su capacidad para soportar cargas elevadas en relación con 
su tamaño, lo que contribuye a que el robot pueda mantener una carga útil de hasta 5\,kg sin comprometer 
la precisión. Asimismo, su diseño compacto permite integrar el actuador y el reductor dentro de cada 
articulación, reduciendo el volumen total del brazo y facilitando su instalación en espacios reducidos.
\vspace{-0.3cm}
\begin{figure}[htb]
    \centering
    \includegraphics[scale=0.313]{Images/ufactory/reductor2.png}
    \caption{\textit{Reductor armónico}. Fuente: \cite{articleReductor}.}
    \label{fig:reductor}
\end{figure}

\vspace{-0.3cm}
\begin{table}[htb]
\centering
\begin{tabular}{|c|c|c|c|c|c|c|}
\hline
\rowcolor[HTML]{000000} 
{\color[HTML]{FFFFFF} \textbf{Parámetro}} & {\color[HTML]{FFFFFF} \textbf{Joint 1}} & {\color[HTML]{FFFFFF} \textbf{Joint 2}} & {\color[HTML]{FFFFFF} \textbf{Joint 3}} & {\color[HTML]{FFFFFF} \textbf{Joint 4}} & {\color[HTML]{FFFFFF} \textbf{Joint 5}} & {\color[HTML]{FFFFFF} \textbf{Joint 6}} \\ \hline

\cellcolor[HTML]{000000}{\color[HTML]{FFFFFF} \textbf{Rango de trabajo}} 
& $\pm$ 360º & -117º -- 116º & -219º -- 10º & $\pm$ 360º & -97º -- 180º & $\pm$ 360º \\ \hline

\cellcolor[HTML]{000000}{\color[HTML]{FFFFFF} \textbf{Par máximo}} 
& 8.4 Nm & 8.4 Nm & 4.2 Nm & 1.5 Nm & 1.5 Nm & 1.5 Nm \\ \hline

\cellcolor[HTML]{000000}{\color[HTML]{FFFFFF} \textbf{Vel. y acel. máxima}} 
& \multicolumn{6}{c|}{180 º/s, 1500 º/s²} \\ \hline

\cellcolor[HTML]{000000}{\color[HTML]{FFFFFF} \textbf{Carga útil máxima}} 
& \multicolumn{6}{c|}{5 kg} \\ \hline

\cellcolor[HTML]{000000}{\color[HTML]{FFFFFF} \textbf{Potencia nominal}} 
& \multicolumn{6}{c|}{150 W} \\ \hline

\cellcolor[HTML]{000000}{\color[HTML]{FFFFFF} \textbf{Repetibilidad}} 
& \multicolumn{6}{c|}{$\pm$ 0.1 mm} \\ \hline

\end{tabular}
\caption{Características mecánicas y de actuadores del \textit{xArm 6}.}
\label{tab:actuators}
\end{table}

\subsection{Sensores integrados y opcionales}

El \textit{xArm 6} incorpora sensores básicos de posición en cada articulación mediante \textit{encoders} absolutos digitales,
que permiten conocer de forma directa y precisa la posición angular de cada eje sin necesidad de realizar un proceso
de referencia tras el encendido \cite{barrientos2007fundamentos}. En estos codificadores el disco transparente se divide en un número determinado de sectores,
siempre potencia de 2, cada uno codificado según un código binario cíclico, normalmente \textit{código Gray}, representado por
zonas transparentes y opacas dispuestas radialmente. De este modo, cada posición queda codificada de forma única y absoluta,
sin necesidad de contadores ni electrónica adicional para detectar el sentido de giro. \\

Además, dispone de funciones internas de detección de colisión por \textit{software} que
garantizan la seguridad durante la operación. Estas funciones monitorizan las corrientes de los motores y detienen el
movimiento si se detecta un contacto inesperado, lo que generaría un aumento brusco de corriente provocado por la
resistencia mecánica. Por tanto, el robot no anticipa la colisión, sino que la detecta una vez se ha producido. \\

El fabricante también ofrece una gama de sensores opcionales que amplían las capacidades del robot.
Entre ellos destaca el sensor de fuerza o torque de seis ejes, instalado en la brida del manipulador,
capaz de medir fuerzas y momentos en las tres direcciones espaciales (Fx, Fy, Fz, Mx, My, Mz). Este
sensor resulta fundamental en tareas de ensamblaje, manipulación delicada y control por contacto. \\

Otros sensores que se pueden incorporar son los sensores de presión analógicos en el \textit{gripper} de vacío, que permiten verificar la correcta
sujeción de piezas mediante succión, así como sensores digitales de fuerza en los \textit{gripper}s mecánicos y bio \textit{gripper}s,
que controlan la intensidad del agarre para evitar daños en los objetos manipulados, y la integración de cámaras
en el extremo del brazo, destinadas a aplicaciones de visión artificial e inspección. \\

En el cuadro \ref{tab:sensors_specs} se recogen las características resumidas de comunicación y alimentación de estos 
sensores.

\begin{table}[htb]
\centering
\begin{tabular}{|c|c|c|c|}
\hline
\rowcolor[HTML]{000000}{\color[HTML]{FFFFFF}\textbf{Sensor}} & 
{\color[HTML]{FFFFFF}\textbf{Tipo de señal}} & 
{\color[HTML]{FFFFFF}\textbf{Comunicación}} & 
{\color[HTML]{FFFFFF}\textbf{Alimentación}} \\ \hline

Encoder absoluto & Digital & Interna & Integrada en cada motor \\ \hline
Detección de colisión & Virtual & Interna & Alimentación del actuador \\ \hline
Fuerza o torque 6 ejes & Digital & \textit{USB} / \textit{Ethernet} & \textit{Hub} del robot \\ \hline
Sensor de presión & Analógica / Digital & Interna & \textit{Control box} / \textit{gripper} \\ \hline
Sensor de fuerza & Digital & Interna & \textit{Control box} / \textit{gripper} \\ \hline
Cámara & Digital & \textit{USB} / \textit{Ethernet} & \textit{Hub} del robot o fuente externa \\ \hline

\end{tabular}
\caption{Resumen de tipo de señal, comunicación y alimentación de los sensores del \textit{xArm 6}.}
\label{tab:sensors_specs}
\end{table}

\newpage
\subsection{Efector}
El efector final del \textit{xArm 6} corresponde a la brida situada en el extremo del manipulador,
diseñada con interfaces mecánicas estándar que permiten la conexión de una amplia variedad de herramientas.
Entre los efectores más habituales se encuentran las pinzas mecánicas, los \textit{grippers} de vacío y los
bio \textit{grippers}, todos ellos disponibles como accesorios oficiales del fabricante. Asimismo, el sistema
admite la instalación de cámaras en la brida para aplicaciones de visión artificial e inspección. \\

En la figura \ref{fig:efector} se observa la pinza mecánica junto con la cámara \textit{Intel RealSense D435} \cite{IntelRealSenseD435},
recomendada por el fabricante.

\begin{figure}[htb]
    \centering
    \includegraphics[scale=0.25]{Images/ufactory/pinza.jpg}
    \caption{Ejemplo de elemento terminal. Fuente: \cite{ufactory_xarm6}.}
    \label{fig:efector}
\end{figure}


La brida del robot cumple con el patrón de taladros indicado en la ISO 9409-1:2004 \cite{ISO9409-1}, lo que facilita la 
integración de herramientas de terceros y asegura la compatibilidad con dispositivos de medida como el sensor de fuerza
o torque de seis ejes. Este sensor se instala directamente en el efector y permite medir fuerzas y momentos en las
tres direcciones espaciales, ampliando las capacidades del manipulador en tareas de ensamblaje y
manipulación delicada. \\

El efector final del \textit{xArm 6} está diseñado para soportar una carga útil máxima de 5\,kg,
así como un momento máximo de 10\,Nm en la muñeca, garantizando un funcionamiento seguro dentro de
los límites especificados por el fabricante. \\

En el cuadro \ref{tab:end_effectors_specs} se recogen las características resumidas de comunicación y alimentación de estos 
elementos terminales.

\begin{table}[htb]
\centering
\begin{tabular}{|c|c|c|c|}
\hline
\rowcolor[HTML]{000000}{\color[HTML]{FFFFFF}\textbf{Elemento terminal}} & 
{\color[HTML]{FFFFFF}\textbf{Tipo de señal}} & 
{\color[HTML]{FFFFFF}\textbf{Comunicación}} & 
{\color[HTML]{FFFFFF}\textbf{Alimentación}} \\ \hline

Pinza mecánica & Digital & Interna & \textit{Control box} / \textit{gripper} \\ \hline
\textit{Gripper} de vacío & Analógica / Digital & Interna & \textit{Control box} / \textit{gripper} \\ \hline
Bio \textit{gripper} & Digital & Interna & \textit{Control box} / \textit{gripper}\\ \hline

\end{tabular}
\caption{Resumen de tipo de señal, comunicación y alimentación de los elementos terminales del efector del \textit{xArm 6}.}
\label{tab:end_effectors_specs}
\end{table}



\subsection{Sistema de control}

El \textit{xArm 6} se acompaña de dos unidades de \textit{hardware} externas que permiten su operación e integración
con accesorios: el \textit{control box} y el \textit{hub}. \\

El \textit{control box} constituye la unidad de control principal del robot, alojando la electrónica de potencia y
el controlador encargado de gestionar los actuadores y sensores y las funciones de seguridad, además de 
suministrar la energía necesaria al manipulador, incorporar puertos de comunicación \textit{Ethernet} y \textit{USB} y el botón de
parada de emergencia para garantizar un uso seguro. \\

Se presenta en dos versiones según el tipo de alimentación
eléctrica requerida. El \textit{control box AC}, figura \ref{fig:control_box_AC}, que se conecta directamente a la red eléctrica (100--240\,V AC),
lo que permite un uso inmediato en entornos de laboratorio o producción ligera sin necesidad de fuentes
externas adicionales, y el \textit{control box DC}, figura \ref{fig:control_box_DC}, que está diseñado para sistemas que operan con
corriente continua de 24\,V, siendo más compacto y ligero, lo que facilita su integración en plataformas
móviles o aplicaciones embebidas, pero requiere de una fuente externa de 24 V DC. \\

Por su parte, el \textit{hub} se instala en el extremo del brazo, junto a la brida, y actúa como módulo de expansión
para la conexión de accesorios que requieren estar en el efector final, como cámaras o el sensor de fuerza o torque
de seis ejes. Este dispositivo está conectado por cableado interno al \textit{control box}, del cual recibe tanto la
alimentación en 24\,V DC como la comunicación con el controlador del robot. De esta forma, los accesorios pueden
integrarse sin necesidad de cableado externo adicional. Cabe señalar que los grippers oficiales, por defecto, no se conectan al
\textit{hub}, ya que su control está integrado en el \textit{firmware} del \textit{control box}. \\

El modo de operación habitual del \textit{xArm 6} consiste en conectar el \textit{control box} a la red eléctrica
y establecer la comunicación con el ordenador de control a través de la interfaz \textit{\textit{Ethernet}}, utilizando
una dirección IP asignada en la red local. Aunque también es posible
la conexión directa por \textit{\textit{USB}}, la comunicación por \textit{IP} resulta más versátil y constituye el modo de operación
más extendido en aplicaciones industriales y colaborativas. \\

En la tabla \ref{tab:control_hub_specs} se recogen resumidamente las características de las unidades de control. \\

\begin{figure}[htb]
    \centering
    % Imagen izquierda
    \begin{minipage}[t]{0.45\textwidth}
        \centering
        \includegraphics[scale=0.1]{Images/ufactory/ac_controlBox.png}
        \caption{\textit{Control box} analógico. Fuente: \cite{ufactory_xarm6}.}
        \label{fig:control_box_AC}
    \end{minipage}
    \hfill
    % Imagen derecha
    \begin{minipage}[t]{0.45\textwidth}
        \centering
        \includegraphics[scale=0.25]{Images/ufactory/dc_controlBox.png}
        \caption{\textit{Control box} digital. Fuente: \cite{ufactory_xarm6}.}
        \label{fig:control_box_DC}
    \end{minipage}
\end{figure}


\begin{table}[htb]
\centering
\begin{tabular}{|c|c|c|}
\hline
\rowcolor[HTML]{000000}{\color[HTML]{FFFFFF}\textbf{Componente}} & 
{\color[HTML]{FFFFFF}\textbf{Alimentación}} & 
{\color[HTML]{FFFFFF}\textbf{Comunicación}} \\ \hline

Control Box (AC) & 100--240\,V AC & \textit{Ethernet} / \textit{USB} \\ \hline
Control Box (DC) & 24\,V DC & \textit{Ethernet} / \textit{USB} \\ \hline
Hub en la brida & 24\,V DC (desde el robot) & \textit{Ethernet} / \textit{USB} \\ \hline

\end{tabular}
\caption{Resumen de las especificaciones técnicas de las unidades de control del \textit{xArm 6}.}
\label{tab:control_hub_specs}
\end{table}

\newpage
\subsection{\textit{Software}}
%Gracias a la combinación de características mecánicas y de software, el \textit{xArm 6} constituye una plataforma versátil 
%y robusta para el estudio cinemático y dinámico, así como para el desarrollo de aplicaciones en robótica industrial 
%y colaborativa. \\
El robot dispone de interfaces de control que permiten su programación tanto en \textit{Python} como en \textit{C++},
además de una integración nativa con el \textit{framework} \textit{ROS2} a través del repositorio oficial del fabricante \cite{xarm_ros2}. La comunicación con el manipulador se 
realiza principalmente a través de los puertos \textit{\textit{Ethernet}} y \textit{\textit{USB}} del \textit{control box}, mediante los cuales 
se transmiten las órdenes de movimiento y se reciben datos de estado y retroalimentación de los sensores. \\

El fabricante proporciona un conjunto de librerías y \textit{APIs} que facilitan la programación de trayectorias, el control 
de efectores finales y la integración con sistemas externos. Asimismo, se incluye una interfaz gráfica denominada 
\textit{xArm Studio} \cite{ufactory_xarmstudio}, que permite la configuración inicial, la calibración y la ejecución de programas de manera intuitiva, 
sin necesidad de conocimientos avanzados de programación, así como la programación por bloques visuales de color. 
Estas herramientas convierten al \textit{xArm 6} en una plataforma abierta y flexible, apta tanto para entornos académicos
como industriales. \\

\begin{figure}[htb]
    \centering
    \includegraphics[scale=0.255]{Images/ufactory/gui.png}
    \caption{\textit{xArm Studio}. Fuente: \cite{ufactory_xarm6}.}
    \label{fig:efector}
\end{figure}
