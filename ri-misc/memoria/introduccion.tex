\datos{color}{}{}

\section{Introducción}
%\addcontentsline{toc}{section}{Tema 2 - Morfología del robot}

Este proyecto se enmarca en la asignatura de Robótica Industrial del Máster en Ingeniería de Sistemas y Control.
De entre las opciones planteada, este responde al tercero: \textit{\textcolor{gray}{``estudio sobre 
un tema libre.``}} \\

Para la realización del trabajo se ha utilizado como base el robot \textit{UFactory xArm 6} \cite{ufactory_xarm6} en simulación
mediante el \textit{framework} ROS2 \cite{doi:10.1126/scirobotics.abm6074} con el objetivo de realizar un estudio cinemático y
dinámico que lo permita programar. En otras palabras, este proyecto trabaja sobre un robot real. \\

\subsection{Objetivo y motivación del proyecto}
El objetivo del proyecto es trasladar los conceptos teóricos y prácticos estudiados en la asignatura a un robot 
real para aprender a trabajar con máquinas reales. \\

Este proyecto viene motivado por el trabajo que desempeña el estudiante en la Escuela de Ingeniería de Bilbao, 
donde se va a comenzar a trabajar con el citado brazo robótico para proyectos de investigación. \\

\subsection{Metodología} \label{met}
Dado que por el momento no se disponer del robot montado y configurado, se ha optado por trabajar en simulación
empleando el \textit{software} Gazebo \cite{gazebo_harmonic} y la distribución \textit{Jazzy Jalisco} \cite{ros2_jazzy} de ROS2.
Se ha utilizado la documentación oficial del fabricante, tanto el repositorio de ROS2 \cite{xarm_ros2} como el manual
de usuario del robot \cite{ufactory_xarm_manual}. \\

Los ejercicios o aplicaciones se han desarrollado en \textit{Python? C++?}...