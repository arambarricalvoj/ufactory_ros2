\datos{color}{}{}

\section{Evaluación y análisis de costes}

La adquisición e integración de un manipulador industrial como el \textit{xArm6}
de \textit{UFactory} implica una serie de costes directos e indirectos que deben ser
evaluados para determinar la viabilidad económica del sistema. En esta sección
se analizan los componentes principales del coste y la relación coste–beneficio en el
contexto de aplicaciones de manipulación robótica. 

\subsection{Costes directos}

Los costes directos corresponden a los elementos necesarios para disponer del
robot en funcionamiento básico. Para el caso del manipulador \textit{xArm6}, los
principales componentes son:

\begin{itemize}
    \item Robot UFactory xArm6: incluye el brazo robótico de 6 grados de
    libertad, la controladora integrada y el cableado principal. Su precio suele
    situarse alrededor de los 9.000 €.
    \item Pinza o efector final: el fabricante ofrece pinzas paralelas,
    pinzas neumáticas y herramientas adicionales como un motores lineales y cámaras de visión.
    El coste típico oscila entre
    600 y 3.000\,€.
    \item Fuente de alimentación y accesorios: soportes, bases de
    montaje, extensiones de cableado y elementos de fijación. Ya incluido con 
    el brazo robótico.
    \item \textit{Software} y licencias: el ecosistema de \textit{UFactory}, ROS2 y
    MoveIt2 es de código abierto, por lo que no existen costes de licencia. Sin
    embargo, pueden existir costes asociados a software de terceros o módulos
    adicionales.
\end{itemize}

El coste directo total estimado para un sistema funcional básico se sitúa alrededor
de 12.000\,€.

\subsection{Costes indirectos}

Además del coste del robot, existen costes asociados a su integración en un
entorno real o de laboratorio:

\begin{itemize}
    \item Integración con ROS2 y MoveIt2: aunque el fabricante proporciona
    paquetes ROS2 preconfigurados, la integración con la aplicación final puede
    requerir entre 20 y 60 horas de trabajo técnico.
    \item Infraestructura: mesa de trabajo, protecciones, sensores
    adicionales o cámaras externas.
    \item Formación: capacitación del personal en ROS2, MoveIt2 y
    programación del robot.
    \item Mantenimiento: aunque el \textit{xArm6} no requiere mantenimiento
    intensivo, se recomienda una revisión anual y posibles sustituciones de
    piezas menores.
\end{itemize}

\subsection{Relación coste–beneficio}

El \textit{xArm6} destaca por su relación coste–prestaciones en comparación con
manipuladores industriales tradicionales, cuyo precio suele situarse entre
25.000 y 60.000\,€. Para aplicaciones de investigación, docencia, prototipado
rápido o automatización ligera, el \textit{xArm6} ofrece un coste significativamente
inferior, integración nativa con ROS2 y MoveIt2, controladora cerrada y estable
proporcionada por el fabricante, facilidad de uso mediante interfaz gráfica o nodos
ROS2 y bajo coste de mantenimiento. \\

En consecuencia, el sistema resulta económicamente atractivo para entornos donde
la precisión extrema o la carga elevada no son requisitos críticos, pero sí lo
son la flexibilidad, la rapidez de integración y el coste reducido.