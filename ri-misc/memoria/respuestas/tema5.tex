\begin{table}[htb]
\centering
\begin{tabular}{|c|c|c|c|c|}
\hline
\rowcolor[HTML]{000000} 
{\color[HTML]{FFFFFF} 1} & {\color[HTML]{FFFFFF} 2} & {\color[HTML]{FFFFFF} 3} & {\color[HTML]{FFFFFF} 4} & {\color[HTML]{FFFFFF} 5} \\
\hline
\rowcolor[HTML]{FFFFFF} 
a & a, c & d & a & b \\
\hline
\end{tabular}
\caption{Tabla de respuestas de las cuestiones del quinto tema}
\label{table:tema1}
\end{table}


\begin{enumerate}
\item \textbf{¿Qué estudia la dinámica del robot?} (p. 215) 

La respuesta correcta es la a), que corresponde con la página 215 del libro: \\
``\textcolor{gray}{\textit{el modelo dinámico de un robot tiene por objetivo conocer 
la relación entre el movimiento del robot y las fuerzas implicadas en el mismo.
Esta relación se obtiene mediante el denominado modelo dinámico, que establece la relación matemática entre:
la localización del robot definida por sus variables articulares o 
por las coordenadas de localización de su extremo, y sus derivadas: velocidad y aceleración;
las fuerzas y pares aplicados en las articulaciones (o en el extremo del robot); y
los parámetros dimensionales del robot, como longitud, masas e inercias de sus elementos.}}`` 

La respuesta b) corresponde a la cinemática del robot y la d) no incluye los parámetros dimensionales del robot. \\

\item \textbf{El modelo dinámico es imprescindible para conseguir...} (p. 215, 2 respuestas)

Las respuestas correctas son la a) y la c), que corresponden con la página 215 del libro: \\
``\textcolor{gray}{\textit{el modelo dinámico es imprescindible para conseguir los siguientes fines:
simulación del movimiento del robot, diseño y evaluación de la estructura mecánica del robot,
dimensionamiento de los actuadores y diseño y evaluación del control dinámico del robot.}}`` \\

\item \textbf{¿Cómo se plantea el modelo cinemático?} (p. 216)

La única respuesta correcta es la d), que corresponde con la página 216 del libro: \\
``\textcolor{gray}{\textit{La obtención del modelo dinámico de un mecanismo, y en particular de un robot, se basa
fundamentalmente en el planteamiento del equilibrio de fuerzas establecido en la segunda ley de
Newton, o su equivalente para movimientos de rotación, la denominada ley de Euler. [...] El planteamiento del 
equilibrio de fuerzas en un robot real de 5 o 6 grados de libertad, [...] se puede usar la formulación Lagrangiana,
basada en consideraciones energéticas. Este planteamiento es más sistemático que el anterior, y por tanto,
facilita enormemente la formulación de un modelo tan complejo como el de un robot.}}`` \\

\newpage
\item \textbf{¿Cuál es el par articular $\tau$ necesario para mantener en equilibrio un robot monoarticular de un solo eslabón de longitud $L$,
con una masa puntual $m$ en su extremo, situado a un ángulo $\theta$ respecto a la vertical?} (p. 216)

La única respuesta correcta es la a). 

Si definimos el ángulo del eslabón $\alpha$ respecto a la horizontal, entonces:
\[
\tau = L \, m g \, \sin\alpha
\]

En cambio, si definimos el ángulo $\theta$ respecto a la vertical, se cumple que 
$\alpha = 90^\circ - \theta$. Sustituyendo:

\[
\tau = L \, m g \, \sin(90^\circ - \theta) = L \, m g \, \cos\theta
\] \\

\item \textbf{¿Qué ecuación define la Lagrangiana?} (ec. 5.3, p. 217)

La única respuesta correcta es la b), que corresponde con la ecuación 5.3 de la página 217 del libro. \\

\item \textbf{Obtener el modelo dinámico del segundo eslabón del robot de la siguiente imagen mediante la Lagrangiana.} (p. 220)

a) Obtener el modelo dinámico del segundo eslabón del robot de la siguiente imagen mediante la Lagrangiana.\\

Respuesta:
\[\tau_1 = d_{11}(q)\; \ddot{q}_1 + \phi_1(q) = (m_1 \cdot L_{c1}^2 + I_1 + m_2 \cdot L_1^2 + I_2) \ddot{q_1}
- S_1\cdot g \cdot (m_1 \cdot L_{c1} - m_2\cdot L_1) \]
\[\tau_2 = d_{22}(q)\; \ddot{q}_2 + \phi_2(q) = \frac{m_2}{4} \ddot{q_2} \]

Por cuestiones de tiempo, se adjuntan las siguientes imágenes con la resolución.

\begin{center}
  \includegraphics[scale=0.21, angle=0]{Images/tema5/lagrangiana/1_b.jpg}
  \includegraphics[scale=0.2, angle=0]{Images/tema5/lagrangiana/2_b.jpg}
  \includegraphics[scale=0.19, angle=0]{Images/tema5/lagrangiana/3.jpg}
  \includegraphics[scale=0.22, angle=0]{Images/tema5/lagrangiana/4.jpg}
\end{center}

Con \textit{Matlab} para este caso concreto:
\lstinputlisting[]{Matlab/tema5/dinamicaL.txt}
\lstinputlisting[]{Matlab/tema5/dinamicaL_salida.txt}

Nota: en el segundo eslabón aparece el término gravedad que en el cálculo manual no.
Esto se debe a que los cálculos en \textit{Matlab} no asumen que el COM2 (centro de masas del segundo eslabón)
no varía en altura. En otras palabras, la articulación $q_2$ es perpendicular a la gravedad y por tanto,
en la ecuación del segundo eslabón no aparece la gravedad, puesto que respecto a su marco la altura es $0$.
La componente de gravedad que aparece en el resultado tiene que anularse. \\

b) Obtener el modelo dinámico del segundo eslabón del robot de la siguiente imagen mediante Matlab (la \textit{toolbox}
  implementa el método recursivo de Newton-Euler).
\lstinputlisting[]{Matlab/tema5/ejerNE.txt}
\lstinputlisting[]{Matlab/tema5/ejerNE_salida.txt}

\newpage
c) Considerando solo los dos primeros eslabones, calcular el par ($N \cdot m$) necesario en la primera articulación para producir una aceleración de $\ddot{q}_1 = 1 (m/s)²$ desde reposo.\\

Al igual que en el apartado anterior, al evaluar simbólicamente con la \textit{toolbox} aparecen fracciones
racionales muy grandes. Para evitarlo, habría que calcular numéricamente calculando los momentos de inercia.

\lstinputlisting[]{Matlab/tema5/dinamicaInversa.txt}
\lstinputlisting[]{Matlab/tema5/dinamicaInversa_salida.txt}

\end{enumerate}