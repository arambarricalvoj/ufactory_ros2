\begin{table}[htb]
\centering
\begin{tabular}{|c|c|c|c|c|c|c|c|}
\hline
\rowcolor[HTML]{000000} 
{\color[HTML]{FFFFFF} 1} & {\color[HTML]{FFFFFF} 2} & {\color[HTML]{FFFFFF} 3} & {\color[HTML]{FFFFFF} 4} & {\color[HTML]{FFFFFF} 5} & {\color[HTML]{FFFFFF} 6} & {\color[HTML]{FFFFFF} 7} & {\color[HTML]{FFFFFF} 8} \\
\hline
\rowcolor[HTML]{FFFFFF} 
d & a & b & c & a & b & a & d \\
\hline
\end{tabular}
\caption{Tabla de respuestas de las cuestiones del cuarto tema}
\label{table:tema1}
\end{table}


\begin{enumerate}
\item \textbf{¿Qué estudia la cinemática del robot?} (p. 119) 

La respuesta correcta es la d), que corresponde con la página 119 del libro: \\
``\textcolor{gray}{\textit{La cinemática del robot estudia el movimiento del mismo
con respecto a un sistema de referencia sin considerar las fuerzas que intervienen.
Así, la cinemática se interesa por la descripción analítica del movimiento espacial del robot
como una función del tiempo, y en particular por las relaciones entre la posición y la orientación
del extremo final del robot con los valores que toman sus coordenadas articulares.}}`` 

Las respuestas a) y c) están relacionadas con la dinámica, y la respuesta b) con la robótica
mecánica o la ingeniería de materiales del robot. \\

\item \textbf{¿Qué es el problema cinemático directo?} (pp. 119 y 120)

La respuesta correcta es la a), que corresponde la página 119 del libro: \\
``\textcolor{gray}{\textit{el problema cinemático directo [...] consiste en determinar cuál
es la posición y orientación del extremo final del robot, con respecto a un sistema de
coordenadas que se toma como referencia, conocidos los valores de las articulaciones y los
parámetros geométricos de los elementos del robot}}`` 

La respuesta b) corresponde al problema cinemático inverso.\\
La respuesta c) corresponde a la dinámica.\\
La respuesta d) corresponde al diseño mecánico. \\

\item \textbf{¿Qué es el problema cinemático indirecto?} (pp. 119 y 134)

La respuesta correcta es la b), que corresponde con la página 119 del libro: \\
``\textcolor{gray}{\textit{[...] problema cinemático inverso, resuelve la configuración que debe
adoptar el robot para una posición y orientación del extremo conocidas.}}`` 

La respuesta a) correponde al problema cinemático directo. \\
La respuesta c) corresponde a la dinámica. \\
La respuesta d) corresponde con el diseño mecánico. \\

\newpage
\item \textbf{¿Cuál es la cinemática del extermo del robot de la imagen según el método geométrico?}

Siguiendo los pasos del método descrito en la página 120 del libro, se obtiene que la respuesta
correcta es la c) $(l_1\cdot \cos(\alpha) + l_2\cdot \cos(\alpha + \beta),\; l_1\cdot \sin(\alpha) + l_2\cdot \sin(\alpha + \beta))$.
\begin{center}
  \includegraphics[scale=0.8]{Images/tema4/4_sol.png}
\end{center}


\item \textbf{¿Cuál es la cinemática del extermo del robot de la imagen según el método geométrico?}

Siguiendo los pasos del método descrito en la página 120 del libro, se obtiene que la respuesta
correcta es la a) $(l_1\cdot \cos(\alpha),\; l_1\cdot \sin(\alpha),\; l_2)$.
\vspace{-1.5cm}
\begin{center}
  \includegraphics[scale=1]{Images/tema4/5_sol.png}
\end{center}


\item \textbf{¿Cuál es la cinemática del extermo del robot de la imagen según el método geométrico?}

Siguiendo los pasos del método descrito en la página 120 del libro, se obtiene que la respuesta
correcta es la b) $(d_2,\; l_1\cdot \cos(\alpha) + l_3\cdot \cos(\beta),\; l_1\cdot \sin(\alpha) + l_3\cdot \sin(\beta))$.
\vspace{-0.2cm}
\begin{center}
  \includegraphics[scale=1]{Images/tema4/6_sol.png}
\end{center}

\newpage
\item \textbf{¿Cuáles son los pasos del algoritmo de Denavit-Hartenberg?} (p. 125)

La respuesta correcta es la a), tal y como se muestra en la página 125 del libro:\\
``\textcolor{gray}{\textit{1. Rotación alrededor del eje $z_{i-1}$ un ángulo $\theta_i$. \\
2. Traslación a lo largo de $z_{i-1}$ una distancia $d_i$; vector $\mathbf{d}_i = (0,\,0,\,d_i)$. \\
3. Traslación a lo largo de $x_i$ una distancia $a_i$; vector $\mathbf{a}_i = (a_i,\,0,\,0)$. \\
4. Rotación alrededor del eje $x_i$ un ángulo $\alpha_i$.}}`` \\


\item \textbf{¿Qué es el modelo diferencial del robot?} (p. 147)

La respuesta correcta es la d), que corresponde con el siguiente párrafo de la página 147:
``\textcolor{gray}{\textit{la relación entre las velocidades de las coordenadas articulares y
las de la posición y orientación del extremo, o lo que es equivalente, el efecto que un movimiento
diferencial de las variables articulares tiene sobre las variables en el espacio de la tarea.
Esta relación queda definida por el modelo diferencial. Mediante él, el sistema de control del
robot puede establecer qué velocidades debe imprimir a cada articulación (a través de sus
respectivos actuadores) para conseguir que el extremo desarrolle una trayectoria temporal concreta,
por ejemplo, una línea recta a velocidad constante.}}`` \\

\item \textbf{Dado el robot de la imagen, con articulaciones rotacionales $q_1$ y $q_3$ y articulaciones
prismáticas $q_2$ y $q_4$, responder a la siguientes preguntas:}
\begin{center}
  \includegraphics[scale=0.35, angle=90]{Images/tema4/dh/9_sol.jpg}
\end{center}

\newpage
a)        Obtener la tabla de parámetros de Denavit-Hartenberg y resolver la cinemática del extremo del robot.

\begin{table}[htb]
\centering
\renewcommand{\arraystretch}{1.2}
\begin{tabular}{|c|c|c|c|c|}
\hline
\rowcolor[HTML]{000000}
{\color[HTML]{FFFFFF}\textbf{Eslabón $i$}} & 
{\color[HTML]{FFFFFF}\textbf{Rotación $Z_{i-1}$}} & 
{\color[HTML]{FFFFFF}\textbf{Traslación $Z_{i-1}$}} & 
{\color[HTML]{FFFFFF}\textbf{Traslación $X_i$}} & 
{\color[HTML]{FFFFFF}\textbf{Rotación $X_i$}} \\
\hline
\rowcolor[HTML]{000000}
{\color[HTML]{FFFFFF}} & 
{\color[HTML]{FFFFFF}$\theta_i$} & 
{\color[HTML]{FFFFFF}$d_i$} & 
{\color[HTML]{FFFFFF}$a_i$} & 
{\color[HTML]{FFFFFF}$\alpha_i$} \\
\hline
1 & $q_1$ & $0$ & $L_1$ & $0$ \\
\hline
2 & $0$ & $q_2$ & $0$ & $-90$ \\
\hline
3 & $q_3$ & $0$ & $0$ & $90$ \\
\hline
4 & $0$ & $L_3+q_4$ & $0$ & $0$ \\
\hline
\end{tabular}
\caption{Parámetros Denavit–Hartenberg del robot del ejercicio 9.}
\label{tab:dh}
\end{table}


\[A_i = R_z(\theta_i)\, T_z(d_i)\, T_x(a_i)\, R_x(\alpha_i)\]
\[A_i =
\begin{bmatrix}
\cos\theta_i & -\sin\theta_i \cos\alpha_i & \sin\theta_i \sin\alpha_i & a_i \cos\theta_i \\
\sin\theta_i & \cos\theta_i \cos\alpha_i & -\cos\theta_i \sin\alpha_i & a_i \sin\theta_i \\
0 & \sin\alpha_i & \cos\alpha_i & d_i \\
0 & 0 & 0 & 1
\end{bmatrix}
\]
\[T_0^{\,n} \;=\; A_1 \, A_2 \, \cdots \, A_n \;=\; \prod_{i=1}^{n} A_i\] \\

\[
A_1 =
\begin{bmatrix}
\cos q_1 & -\sin q_1 & 0 & L_1 \cos q_1 \\
\sin q_1 & \ \cos q_1 & 0 & L_1 \sin q_1 \\
0 & 0 & 1 & 0 \\
0 & 0 & 0 & 1
\end{bmatrix}
\]

\[
A_2 =
\underbrace{\begin{bmatrix}
1 & 0 & 0 & 0 \\
0 & 1 & 0 & 0 \\
0 & 0 & 1 & q_2 \\
0 & 0 & 0 & 1
\end{bmatrix}}_{T_z(q_2)}\;
\underbrace{\begin{bmatrix}
1 & 0 & 0 & 0 \\
0 & 0 & 1 & 0 \\
0 & -1 & 0 & 0 \\
0 & 0 & 0 & 1
\end{bmatrix}}_{R_x(-90^\circ)}
=
\begin{bmatrix}
1 & 0 & 0 & 0 \\
0 & 0 & 1 & 0 \\
0 & -1 & 0 & q_2 \\
0 & 0 & 0 & 1
\end{bmatrix}
\]

\[
A_3 =
\underbrace{\begin{bmatrix}
\cos q_3 & -\sin q_3 & 0 & 0 \\
\sin q_3 & \ \cos q_3 & 0 & 0 \\
0 & 0 & 1 & 0 \\
0 & 0 & 0 & 1
\end{bmatrix}}_{R_z(q_3)}\;
\underbrace{\begin{bmatrix}
1 & 0 & 0 & 0 \\
0 & 0 & -1 & 0 \\
0 & 1 & 0 & 0 \\
0 & 0 & 0 & 1
\end{bmatrix}}_{R_x(90^\circ)}
=
\begin{bmatrix}
\cos q_3 & 0 & \sin q_3 & 0 \\
\sin q_3 & 0 & -\cos q_3 & 0 \\
0 & 1 & 0 & 0 \\
0 & 0 & 0 & 1
\end{bmatrix}
\]

\[
A_4 =
\underbrace{\begin{bmatrix}
1 & 0 & 0 & 0 \\
0 & 1 & 0 & 0 \\
0 & 0 & 1 & L_3 + q_4 \\
0 & 0 & 0 & 1
\end{bmatrix}}_{T_z(L_3+q_4)}
\] \\

% Producto total T = A_1 A_2 A_3 A_4
\[
T_{4}^{0} = A_1\,A_2\,A_3\,A_4 =
\begin{bmatrix}
\cos q_1 \cos q_3 & -\sin q_1 & \cos q_1 \sin q_3 & \; L_1 \cos q_1 + (L_3+q_4)\,\cos q_1 \sin q_3 \\
\sin q_1 \cos q_3 & \ \cos q_1 & \sin q_1 \sin q_3 & \; L_1 \sin q_1 + (L_3+q_4)\,\sin q_1 \sin q_3 \\
-\sin q_3 & 0 & \cos q_3 & \; q_2 + (L_3+q_4)\,\cos q_3 \\
0 & 0 & 0 & 1
\end{bmatrix}
\]

La posición del extremo del robot respecto del sistema origen ${S_0}$ vendrá dado por las siguientes ecuaciones:
\[x =  L_1 \cos q_1 + (L_3+q_4)\,\cos q_1 \sin q_3\]
\[y = L_1 \sin q_1 + (L_3+q_4)\,\sin q_1 \sin q_3\]
\[z = q_2 + (L_3+q_4)\,\cos q_3\]

Este resultado se puede comprobar gráficamente dando valores a las articulaciones, por ejemplo, $q_i = 0$ y
$q_i = 90$. \\

b)    Obtener la velocidad lineal mediante la matriz jacobiana analítica.

Calculamos la matriz jacobiana analítica lineal, que viene definida por las derivadas parciales
de las coordenadas del extremo del robot respecto del origen (vector $\vec{p}$):
\[
J_v(q) \;=\; 
\frac{\partial}{\partial q}
\begin{bmatrix}
x(q) \\

y(q) \\

z(q)
\end{bmatrix}
\;=\;
\begin{bmatrix}
\dfrac{\partial x}{\partial q_1} & \dfrac{\partial x}{\partial q_2} & \cdots & \dfrac{\partial x}{\partial q_n} \\

\dfrac{\partial y}{\partial q_1} & \dfrac{\partial y}{\partial q_2} & \cdots & \dfrac{\partial y}{\partial q_n} \\

\dfrac{\partial z}{\partial q_1} & \dfrac{\partial z}{\partial q_2} & \cdots & \dfrac{\partial z}{\partial q_n}
\end{bmatrix}
\]



\[
J_v(q) =
\begin{bmatrix}
- L_1 \sin q_1 - (L_3+q_4)\sin q_1 \sin q_3 & 0 & (L_3+q_4)\cos q_1 \cos q_3 & \cos q_1 \sin q_3 \\

\;\;L_1 \cos q_1 + (L_3+q_4)\cos q_1 \sin q_3 & 0 & (L_3+q_4)\sin q_1 \cos q_3 & \sin q_1 \sin q_3 \\

0 & 1 & -(L_3+q_4)\sin q_3 & \cos q_3
\end{bmatrix}
\]

\[
\mathbf{v}(q,\dot q) \;=\; J_v(q)\,
\begin{bmatrix}
\dot q_1 \\
\dot q_2 \\
\dot q_3 \\
\dot q_4
\end{bmatrix}
\] 

%%%%
c)        Obtener la cinemática del robot hasta el segundo eslabón mediante cuaterniones (pp. 95 y 132).
Los parámetros DH relevantes son:
\[
\begin{aligned}
&\theta_1 = q_1, \quad d_1 = 0, \quad a_1 = L_1, \quad \alpha_1 = 0, \\
&\theta_2 = 0, \quad d_2 = q_2, \quad a_2 = 0, \quad \alpha_2 = -90^\circ
\end{aligned}
\]

Los vectores de traslación son:
\[
\mathbf{p}_1 = \begin{bmatrix} L_1 \\ 0 \\ 0 \end{bmatrix}, 
\qquad
\mathbf{p}_2 = \begin{bmatrix} 0 \\ 0 \\ q_2 \end{bmatrix}
\]

Cuaterniones de rotación:
\[
Q_1 = \left[\cos\frac{q_1}{2},\; 0,\; 0,\; \sin\frac{q_1}{2}\right]
\]
\[
Q_2 = Q_z(0) \circ Q_x(-90^\circ) = Q_x(-90^\circ) =
\left[\cos\frac{-90^\circ}{2},\; \sin\frac{-90^\circ}{2},\; 0,\; 0\right]
= \left[\tfrac{\sqrt{2}}{2},\; -\tfrac{\sqrt{2}}{2},\; 0,\; 0\right]
\] \\

La propagación de posición se expresa como:
El primer eslabón primero rota y después se traslada (ecuación 3.67 de la página 95):
\[
(0,\mathbf{a}_0) = Q_1 \circ (0,\mathbf{a}_1 + \mathbf{p}_1) \circ Q_1^*
\]

El segundo eslabón primero se desplaza y después rota (ecuación 3.66 de la página 95):
\[
(0,\mathbf{a}_1) = Q_2 \circ (0,\mathbf{a}_2) \circ Q_2^* + \mathbf{p}_2
\]

donde:
\[
\mathbf{p}_1 = \begin{bmatrix} L_1 \\ 0 \\ 0 \end{bmatrix}, 
\qquad
\mathbf{p}_2 = \begin{bmatrix} 0 \\ 0 \\ q_2 \end{bmatrix},
\qquad
\mathbf{a}_2 = \text{coordenadas del extremo en $\{S_2\}$.}
\] \\

La propagación de orientación se expresa como:
\[
R_0 = Q_1 \circ Q_2 \circ R_2
\]

donde $R_i$ representa el cuaternión de orientación del extremo del eslabón $i$ respecto a su propio sistema de coordenadas.  
Por tanto, $R_2$ es el cuaternión que describe la orientación del extremo del eslabón 2 en su marco local.  
Si el extremo coincide con el origen del marco 2 y no existe rotación adicional o posterior propia, entonces $R_2$
es la identidad:
\[
R_2 = [1,\,0,\,0,\,0]
\]

Por tanto, en este caso: 
\[
R_0 = Q_1 \circ Q_2
\] \\

Si expresamos la propagación de la posición de forma vectorial, obtenemos:
\[
\mathbf{a}_0 = \mathrm{Rot}(Q_1 \circ Q_2)\,\mathbf{a}_2 + \mathrm{Rot}(Q_1)\,\mathbf{p}_2 + \mathbf{p}_1  
\]

Si el extremo del eslabón 2 coincide con el origen de su marco, es decir, no hay más rotaciones posteriores:
\[
\mathbf{a}_2 = \mathbf{0} \quad \Rightarrow \quad
\mathbf{a}_0 = 
\begin{bmatrix} L_1 \\ 0 \\ 0 \end{bmatrix}
+ \mathrm{Rot}(Q_1)\begin{bmatrix} 0 \\ 0 \\ q_2 \end{bmatrix}
\]

donde $\mathrm{Rot}(Q_1)$ es una rotación alrededor de $Z$ por $q_1$:
\[
\mathrm{Rot}(Q_1) =
\begin{bmatrix}
\cos q_1 & -\sin q_1 & 0 \\
\sin q_1 & \cos q_1 & 0 \\
0 & 0 & 1
\end{bmatrix}
\]

Aplicando esta rotación al vector $\begin{bmatrix}0\\0\\q_2\end{bmatrix}$ se obtiene:
\[
\mathrm{Rot}(Q_1)\begin{bmatrix}0\\0\\q_2\end{bmatrix}
=
\begin{bmatrix}0\\0\\q_2\end{bmatrix}
\]

Por tanto, la posición final del extremo respecto a la base es la siguiente, que coincide con el vector 
$p$ de la matriz $T^2_0$ del apartado a) de este ejercicio: 
\[
\mathbf{a}_0 = \mathbf{p}_{T^2_0} =
\begin{bmatrix}
L_1 \\ 0 \\ q_2
\end{bmatrix}
\] \\

Una vez demostrada la explicación, se procede a resolver los cálculos manualmente:
\[
(0,\mathbf{a}_1) = Q_2 \circ (0,\mathbf{a}_2) \circ Q_2^* + \mathbf{p}_2
\]

Como se ha mencionado anteriormente, en este caso $\mathbf{a}_2 = 0$. Por tanto:
\[
Q_2 \circ (0,\mathbf{a}_2) \circ Q_2^* = (0,\; 0,\; 0,\; 0)
\]

De modo que:
\[
(0,\mathbf{a}_1) = Q_2 \circ (0,\mathbf{a}_2) \circ Q_2^* + (0,\mathbf{p}_2) = (0,\; 0,\; 0,\; q_2)
\]

Volvemos al primer eslabón, sustituimos y calculamos:
\[
(0,\mathbf{a}_0) = Q_1 \circ (0,\mathbf{a}_1 + \mathbf{p}_1) \circ Q_1^* = Q_1 \circ (0,\; L_1,\; 0,\; q2) \circ Q_1^*
\]
\[
Q_1 \circ (0,\; L_1,\; 0,\; q_2) \circ Q_1^* = (0,\; L_1\cos q_1,\; L_1\sin q_1,\; q_2)
\]

Resultado final:
\[
(0,\mathbf{a}_0)=(0,\,L_1\cos q_1,\,L_1\sin q_1,\,q_2), \qquad \mathbf{a}_0=p_{T^0_2} = \begin{bmatrix}L_1\cos q_1\\L_1\sin q_1\\q_2\end{bmatrix}
\]

Para la propagación de rotaciones:
\[
R_0 = Q_1 \circ Q_2 \circ R_2
\]
\[
R_0 = (\cos\tfrac{q_1}{2},\; 0,\; 0,\; \sin\tfrac{q_1}{2}) \circ (\tfrac{\sqrt{2}}{2},\; -\tfrac{\sqrt{2}}{2},\; 0,\; 0) \circ [1,\,0,\,0,\,0]
\]
\[
R_0 \;=\; \tfrac{\sqrt{2}}{2}\,\left(
\cos\tfrac{q_1}{2},\;
-\cos\tfrac{q_1}{2},\;
-\sin\tfrac{q_1}{2},\;
\sin\tfrac{q_1}{2}
\right)
\]

\newpage
\textbf{RECORDATORIO}:\\
Sean $Q_1 = (q_{10}, q_{11}, q_{12}, q_{13}),\; Q_2 = (q_{20}, q_{21}, q_{22}, q_{23})$ dos cuaterniones. \\

Multiplicación de cuaterniones:
\[
Q_1 \circ Q_2 =
\Big(
q_{10} q_{20} - (q_{11} q_{21} + q_{12} q_{22} + q_{13} q_{23}),\;
q_{10} q_{21} + q_{11} q_{20} + q_{12} q_{23} - q_{13} q_{22},\;
\]
\[
q_{10} q_{22} + q_{12} q_{20} + q_{13} q_{21} - q_{11} q_{23},\;
q_{10} q_{23} + q_{13} q_{20} + q_{11} q_{22} - q_{12} q_{21}
\Big).
\]

Conjugado:
\[
Q_1^* = \left(q_{10},\; - q_{11},\; - q_{12},\; - q_{13}\right)
\]

Suma:
\[Q_1 + Q_2 = \big(q_{10}+q_{20},\; q_{11}+q_{21},\; q_{12}+q_{22},\; q_{13}+q_{23}\big)\] \\

Comprobamos el resultado en \textit{Matlab}:

Se puede definir manualmente la cinemática y multiplicar las matrices:
\lstinputlisting[]{Matlab/tema4/ejercicio9_symb_b.m}
\lstinputlisting[]{Matlab/tema4/ejercicio9_b_salida.txt}

O también se puede hacer uso de las funciones de cinemática de la \textit{toolbox}:
\lstinputlisting[]{Matlab/tema4/ejercicio9_symb.m}
\lstinputlisting[]{Matlab/tema4/ejercicio9_salida.txt}

\vspace{0.3cm}
d)    Siendo $L_1 = 1$, $L_3 = 3$ y las posiciones del extremo $(x,\; y,\; z) = (0,\; 8,\; 2)$, 
  obtener los valores de las articulaciones $q_i$ mediante cinemática inversa.

Si resolvemos analíticamente o gráficamente, se observa que para que $x = 0$ el robot tiene que estar sobre el plano $OYZ$ y
por tanto, $q_1 = 90 ^\circ$ o $q_1 = -90 ^\circ$. Una solución posible es (en grados):
\[ (q_1,\; q_2,\; q_3,\; q_4) = (90,\; 2,\; 90,\; 4)\]
\begin{center}
  \includegraphics[scale=0.2, angle=0]{Images/tema4/cinInv.jpg}
\end{center}

Si resolvemos numéricamente mediante \textit{Matlab}, obtenemos la siguiente solución factible (en radianes):
\[ (q_1;\; q_2;\; q_3;\; q_4) = (1.5708;\; 0.6309;\; 1.3777;\; 4.1326)\]
\vspace{0.2cm}
\lstinputlisting[]{Matlab/tema4/cinInv.txt}
\lstinputlisting[]{Matlab/tema4/cinInv_salida.txt}

\item \textbf{Dado el robot de la imagen, con articulaciones rotacionales $q_1$, $q_2$ y $q_3$, responder
a la siguientes preguntas:}
\vspace{-0.95cm}
\begin{center}
  \includegraphics[scale=0.285, angle=0]{Images/tema4/dh/10_sol.jpg}
\end{center}

\newpage
a) Obtener la tabla de parámetros de Denavit-Hartenberg y resolver la cinemática del extremo del robot.

\begin{table}[htb]
\centering
\renewcommand{\arraystretch}{1.2}
\begin{tabular}{|c|c|c|c|c|}
\hline
\rowcolor[HTML]{000000}
{\color[HTML]{FFFFFF}\textbf{Eslabón $i$}} & 
{\color[HTML]{FFFFFF}\textbf{Rotación $Z_{i-1}$}} & 
{\color[HTML]{FFFFFF}\textbf{Traslación $Z_{i-1}$}} & 
{\color[HTML]{FFFFFF}\textbf{Traslación $X_i$}} & 
{\color[HTML]{FFFFFF}\textbf{Rotación $X_i$}} \\
\hline
\rowcolor[HTML]{000000}
{\color[HTML]{FFFFFF}} & 
{\color[HTML]{FFFFFF}$\theta_i$} & 
{\color[HTML]{FFFFFF}$d_i$} & 
{\color[HTML]{FFFFFF}$a_i$} & 
{\color[HTML]{FFFFFF}$\alpha_i$} \\
\hline
1 & $q_1$ & $0$ & $0$ & $90$ \\
\hline
2 & $q_2$ & $L_1$ & $0$ & $-90$ \\
\hline
3 & $q_3$ & $L_2$ & $L_3$ & $0$ \\
\hline
\end{tabular}
\caption{Parámetros Denavit–Hartenberg del robot del ejercicio 10.}
\label{tab:dh}
\end{table}

% Convención DH estándar: A_i = R_z(\theta_i)\, T_z(d_i)\, T_x(a_i)\, R_x(\alpha_i)
% Notación: c_k=\cos q_k,\; s_k=\sin q_k
% Notación simplificada: Ck = \cos(q_k), Sk = \sin(q_k)
\[
A_1 =
\begin{bmatrix}
C1 & 0  & S1 & 0 \\
S1 & 0  & -C1 & 0 \\
0  & 1  & 0  & 0 \\
0  & 0  & 0  & 1
\end{bmatrix}
\qquad
(\theta_1=q_1,\ d_1=0,\ a_1=0,\ \alpha_1=90^\circ)
\]

\[
A_2 =
\begin{bmatrix}
C2 & 0  & -S2 & 0 \\
S2 & 0  & C2  & 0 \\
0  & -1 & 0   & L_1 \\
0  & 0  & 0   & 1
\end{bmatrix}
\qquad
(\theta_2=q_2,\ d_2=L_1,\ a_2=0,\ \alpha_2=-90^\circ)
\]

\[
A_3 =
\begin{bmatrix}
C3 & -S3 & 0 & L_3 C3 \\
S3 &  C3 & 0 & L_3 S3 \\
0  &  0  & 1 & L_2     \\
0  &  0  & 0 & 1
\end{bmatrix}
\qquad
(\theta_3=q_3,\ d_3=L_2,\ a_3=L_3,\ \alpha_3=0^\circ)
\]

% Producto total T = A_1 A_2 A_3

\[
T_{\,3}^{0} =\]
\[
\begin{bmatrix}
C1 C2 C3 - S1 S3 & -C1 C2 S3 - S1 C3 & -C1 S2 & L_3(C1 C2 C3 - S1 S3) - C1 S2\, L_2 + L_1 S1 \\
S1 C2 C3 + C1 S3 & -S1 C2 S3 + C1 C3 & -S1 S2 & L_3(S1 C2 C3 + C1 S3) - S1 S2\, L_2 - L_1 C1 \\
S2 C3            & -S2 S3            & C2      & S2 L_3 C3 + C2 L_2 \\
0 & 0 & 0 & 1
\end{bmatrix}
\] \\

% Posición del efector final (última columna de T_{0}^{\,3})

\[
\begin{aligned}
x &= L_3(C1 C2 C3 - S1 S3)\; -\; C1 S2\, L_2 \;+\; L_1 S1 \\
y &= L_3(S1 C2 C3 + C1 S3)\; -\; S1 S2\, L_2 \;-\; L_1 C1 \\
z &= S2\, L_3 C3 \;+\; C2\, L_2
\end{aligned}
\]

Este resultado se puede comprobar gráficamente dando valores a las articulaciones, por ejemplo, $q_i = 0$ y
$q_i = 90$. \\

\newpage
b) Obtener la matriz jacobiana geométrica (apartado 4.3.3 en la página 157 del libro). \\

Nos interesan las columnas o vectores $z_i$ y $p_i$ de las matrices de transformación:
\[
\quad
\mathbf{p}_0 =
\begin{bmatrix}
0\\0\\0
\end{bmatrix},
\quad
\mathbf{z}_0 =
\begin{bmatrix}
0\\0\\1
\end{bmatrix}
\]

\[
A_{1}^{0} =
\begin{bmatrix}
C1 & 0  & S1 & 0 \\
S1 & 0  & -C1 & 0 \\
0  & 1  & 0  & 0 \\
0  & 0  & 0  & 1
\end{bmatrix},
\quad
\mathbf{p}_1 =
\begin{bmatrix}
0\\0\\0
\end{bmatrix},
\quad
\mathbf{z}_1 =
\begin{bmatrix}
S1\\-C1\\0
\end{bmatrix}
\]

\[
A_{2}^{0} =
\begin{bmatrix}
C1 C2 & -S1 & -C1 S2 & L_1 S1 \\
S1 C2 & \phantom{-}C1 & -S1 S2 & -L_1 C1 \\
S2    & 0    & \phantom{-}C2 & 0 \\
0     & 0    & 0     & 1
\end{bmatrix},
\quad
\mathbf{p}_2 =
\begin{bmatrix}
L_1 S1\\-L_1 C1\\0
\end{bmatrix},
\quad
\mathbf{z}_2 =
\begin{bmatrix}
- C1 S2\\- S1 S2\\C2
\end{bmatrix}
\]

\[
A_{3}^{0} =\]
\[
\begin{bmatrix}
C1 C2 C3 - S1 S3 & -C1 C2 S3 - S1 C3 & -C1 S2 & L_3(C1 C2 C3 - S1 S3) - C1 S2\, L_2 + L_1 S1 \\
S1 C2 C3 + C1 S3 & -S1 C2 S3 + C1 C3 & -S1 S2 & L_3(S1 C2 C3 + C1 S3) - S1 S2\, L_2 - L_1 C1 \\
S2 C3            & -S2 S3            & C2      & S2 L_3 C3 + C2 L_2 \\
0 & 0 & 0 & 1
\end{bmatrix}
\]

\[
\mathbf{p}_3 =
\begin{bmatrix}
 L_3(C1 C2 C3 - S1 S3) - C1 S2\, L_2 + L_1 S1\\L_3(S1 C2 C3 + C1 S3) - S1 S2\, L_2 - L_1 C1\\S2 L_3 C3 + C2 L_2
\end{bmatrix},
\quad
\mathbf{z}_3 =
\begin{bmatrix}
- C1 S2\\- S1 S2\\C2
\end{bmatrix}
\] \\

Cada columna de la jacobiana geométrica correspondiente a 
articulaciones rotacionales se construye de la siguiente manera:
\[J_i =
\begin{bmatrix}
\mathbf{z}_{i-1} \times \big(\mathbf{p}_n - \mathbf{p}_{i-1}\big) \\
\mathbf{z}_{i-1}
\end{bmatrix} \]

Cada columna de la jacobiana geométrica correspondiente a 
articulaciones prismáticas se construye de la siguiente manera:

\[ J_i =
\begin{bmatrix}
\mathbf{z}_{i-1} \\
\mathbf{0}
\end{bmatrix}\]

Calculamos $\big(\mathbf{p}_n - \mathbf{p}_{i-1}\big)$ (en este ejercicio todas las articulaciones
son rotacionales):
\[ \mathbf{p}_3^{0} = \big(\mathbf{p}_3 - \mathbf{p}_{0}\big) = \mathbf{p}_{3}\]
\[ \mathbf{p}_3^{1} = \big(\mathbf{p}_3 - \mathbf{p}_{1}\big) = \mathbf{p}_{3}\]
\[ \mathbf{p}_3^{2} = \big(\mathbf{p}_3 - \mathbf{p}_{2}\big) = \begin{bmatrix}
L_3\,(C1\,C2\,C3 - S1\,S3) - C1\,S2\,L_2 \\
L_3\,(S1\,C2\,C3 + C1\,S3) - S1\,S2\,L_2 \\
S2\,L_3\,C3 + C2\,L_2
\end{bmatrix}
\]


\newpage
Calculamos $\mathbf{z}_{i-1} \times \big(\mathbf{p}_n - \mathbf{p}_{i-1}\big)$.

Recordar que el producto vectorial se define de la siguiente manera:
\[
\mathbf{a} \times \mathbf{b} =
\begin{bmatrix}
a_1 \\ a_2 \\ a_3
\end{bmatrix}
\times
\begin{bmatrix}
b_1 \\ b_2 \\ b_3
\end{bmatrix}
=
\begin{bmatrix}
a_2 b_3 - a_3 b_2 \\
a_3 b_1 - a_1 b_3 \\
a_1 b_2 - a_2 b_1
\end{bmatrix}
\] \\

\[\mathbf{z}_{0} \times \big(\mathbf{p}_3 - \mathbf{p}_{0}\big) = 
\begin{bmatrix}
- (L_3(S1 C2 C3 + C1 S3) - S1 S2\, L_2 - L_1 C1) \\
L_3(C1 C2 C3 - S1 S3) - C1 S2\, L_2 + L_1 S1 \\
0
\end{bmatrix}\]

\[\mathbf{z}_{1} \times \big(\mathbf{p}_3 - \mathbf{p}_{1}\big) = 
\begin{bmatrix}
- C1 \,(S2 L_3 C3 + C2 L_2) \\
- S1 \,(S2 L_3 C3 + C2 L_2) \\
L_3 C2 C3 - S2 L_2
\end{bmatrix}\]

\[\mathbf{z}_{2} \times \big(\mathbf{p}_3 - \mathbf{p}_{2}\big) = 
\begin{bmatrix}
- L_3 \,(S1 C3 + C1 C2 S3) \\
L_3 \,(C1 C3 - S1 C2 S3) \\
- S2 L_3 S3
\end{bmatrix}\] \\

Finalmente, la matriz jacobiana geométrica es:
\[
J(q) =
\begin{bmatrix}
-y & -C1\,z & -L_3\,(S1\,C3 + C1\,C2\,S3) \\
\phantom{-}x & -S1\,z & \phantom{-}L_3\,(C1\,C3 - S1\,C2\,S3) \\
0 & \phantom{-}L_3\,C2\,C3 - S2\,L_2 & -S2\,L_3\,S3 \\
0 & \phantom{-}S1 & -C1\,S2 \\
0 & -C1 & -S1\,S2 \\
1 & \phantom{-}0 & \phantom{-}C2
\end{bmatrix}
\] \\

\[
\begin{aligned}
x &= L_3\,(C1\,C2\,C3 - S1\,S3) - C1\,S2\,L_2 + L_1\,S1 \\
y &= L_3\,(S1\,C2\,C3 + C1\,S3) - S1\,S2\,L_2 - L_1\,C1 \\
z &= S2\,L_3\,C3 + C2\,L_2
\end{aligned}
\] \\

c) La velocidad lineal del extremo será la submatriz $J(q)(1:3, 1:3)$ y la velocidad angular
la submatriz $J(q)(4:6, 1:3)$:
\[
J_v(q) \;=\; J(q)(1\!:\!3,\,1\!:\!3), 
\qquad
J_\omega(q) \;=\; J(q)(4\!:\!6,\,1\!:\!3)
\]

\[
\mathbf{v}(q,\dot q) \;=\; J_v(q)\,
\begin{bmatrix}
\dot q_1 \\
\dot q_2 \\
\dot q_3
\end{bmatrix},
\qquad
\boldsymbol{\omega}(q,\dot q) \;=\; J_\omega(q)\,
\begin{bmatrix}
\dot q_1 \\
\dot q_2 \\
\dot q_3
\end{bmatrix}
\]

\newpage
Nota: se observa que las columnas de la jacobiana geométrica angular coinciden con las columnas 
$z_{i-1}$ de las matrices de transformación $A_{0}^{0}$, $A_{1}^{0}$ y $A_{2}^{0}$. En caso de aparecer
articulaciones prismáticas, que por naturaleza no aportan velocidad angular, su vector $z_i$ será 
reemplazado por el vector $(0, 0, 0)$. Además, la jacobiana geométrica lineal coincide con la
jacobiana analítica lineal. \\

Por tanto, pueden obtenerse fácilmente las velocidades lineales y angulares derivando el vector
$\vec{p}$ y obteniendo las columnas $z_{i-1}$ de las matrices de transformación $A_{i-1}^{0}$, respectivamente.
No obstante, la geométrica lineal es más robusta y práctica de calcular y tiene una interpretación
física directa de la contribución de cada articulación a la velocidad lineal del extremo. \\

Comprobamos el resultado obtenido aquí por el obtenido en \textit{Matlab} mediante la diferencia
entre ambos. 
\lstinputlisting[]{Matlab/tema4/ejercicio10_symb.m}
\lstinputlisting[]{Matlab/tema4/ejercicio10_salida.txt}

\vspace{0.5cm}

\item \textbf{Dado el robot de la imagen, con articulaciones rotacionales $q_2$y $q_3$ y prismáticas
$q_1$ y $q_4$, responder a la siguientes preguntas:}
\begin{center}
  \includegraphics[scale=0.35, angle=90]{Images/tema4/dh/11_sol.jpg}
\end{center}

\newpage
a) Obtener la tabla de parámetros de Denavit-Hartenberg y resolver la cinemática del extremo del robot.

\begin{table}[htb]
\centering
\renewcommand{\arraystretch}{1.2}
\begin{tabular}{|c|c|c|c|c|}
\hline
\rowcolor[HTML]{000000}
{\color[HTML]{FFFFFF}\textbf{Eslabón $i$}} & 
{\color[HTML]{FFFFFF}\textbf{Rotación $Z_{i-1}$}} & 
{\color[HTML]{FFFFFF}\textbf{Traslación $Z_{i-1}$}} & 
{\color[HTML]{FFFFFF}\textbf{Traslación $X_i$}} & 
{\color[HTML]{FFFFFF}\textbf{Rotación $X_i$}} \\
\hline
\rowcolor[HTML]{000000}
{\color[HTML]{FFFFFF}} & 
{\color[HTML]{FFFFFF}$\theta_i$} & 
{\color[HTML]{FFFFFF}$d_i$} & 
{\color[HTML]{FFFFFF}$a_i$} & 
{\color[HTML]{FFFFFF}$\alpha_i$} \\
\hline
1 & $0$ & $q_1$ & $0$ & $0$ \\
\hline
2 & $q_2+90$ & $0$ & $L_2$ & $0$ \\
\hline
3 & $q_3-90$ & $0$ & $L_3$ & $-90$ \\
\hline
4 & $0$ & $q_4$ & $0$ & $0$ \\ 
\hline
\end{tabular}
\caption{Parámetros Denavit–Hartenberg del robot del ejercicio 11.}
\label{tab:dh}
\end{table}

\[
T^0_4 =
\begin{bmatrix}
\cos(q_2+q_3) & 0 & -\sin(q_2+q_3) & \;\; -L_2 \sin q_2 + L_3 \cos(q_2+q_3) - q_4 \sin(q_2+q_3) \\
\sin(q_2+q_3) & 0 & \;\;\cos(q_2+q_3) & \;\;\;\; L_2 \cos q_2 + L_3 \sin(q_2+q_3) + q_4 \cos(q_2+q_3) \\
0             & -1 & 0               & \;\;\;\; q_1 \\
0             & 0 & 0                & \;\;\;\; 1
\end{bmatrix}
\]

\[x =  -L_2 \sin q_2 + L_3 \cos(q_2+q_3) - q_4 \sin(q_2+q_3)\]
\[y = L_2 \cos q_2 + L_3 \sin(q_2+q_3) + q_4 \cos(q_2+q_3) \]
\[z =  q_1\] \\


b)  Obtener las velocidades lineales y angulares del extremo del robot.

Calculamos la matriz jacobiana analítica lineal, que viene definida por las derivadas parciales
de las coordenadas del extremo del robot respecto del origen (vector $\vec{p}$):
\[
J_v(q) \;=\; 
\frac{\partial}{\partial q}
\begin{bmatrix}
x(q) \\

y(q) \\

z(q)
\end{bmatrix}
\;=\;
\begin{bmatrix}
\dfrac{\partial x}{\partial q_1} & \dfrac{\partial x}{\partial q_2} & \cdots & \dfrac{\partial x}{\partial q_n} \\

\dfrac{\partial y}{\partial q_1} & \dfrac{\partial y}{\partial q_2} & \cdots & \dfrac{\partial y}{\partial q_n} \\

\dfrac{\partial z}{\partial q_1} & \dfrac{\partial z}{\partial q_2} & \cdots & \dfrac{\partial z}{\partial q_n}
\end{bmatrix}
\]

\[
J_v(q) \;=\;
\begin{bmatrix}
0 & -L_2 C2 \;-\; L_3 S23 \;-\; q_4 C23 & -L_3 S23 \;-\; q_4 C23 & -S23 \\
0 & -L_2 S2 \;+\; L_3 C23 \;-\; q_4 S23 & \;\;L_3 C23 \;-\; q_4 S23 & \;\;C23 \\
1 & 0 & 0 & 0
\end{bmatrix}
\]


La velocidad lineal será: $
\mathbf{v}(q,\dot q) \;=\; J_v(q)\,
\begin{bmatrix}
\dot q_1 \\
\dot q_2 \\
\dot q_3 \\
\dot q_4
\end{bmatrix}
$

\newpage
Calculamos la matriz jacobiana geométrica angular, que viene definida por los vectores
$z_{i-1}$ de las matrices $A^0_{i-1}$ (marcados en negrita):

\[
A^0_0 =
\begin{bmatrix}
1 & 0 & \boldsymbol{0} & 0 \\
0 & 1 & \boldsymbol{0} & 0 \\
0 & 0 & \boldsymbol{1} & 0 \\
0 & 0 & \boldsymbol{0} & 1
\end{bmatrix}
\]

\[
A^0_1 =
\begin{bmatrix}
1 & 0 & \boldsymbol{0} & 0 \\
0 & 1 & \boldsymbol{0} & 0 \\
0 & 0 & \boldsymbol{1} & q_1 \\
0 & 0 & \boldsymbol{0} & 1
\end{bmatrix}
\]

\[
A^0_2 =
\begin{bmatrix}
-\sin q_2 & -\cos q_2 & \boldsymbol{0} & -L_2 \sin q_2 \\
\;\;\cos q_2 & -\sin q_2 & \boldsymbol{0} & \;\;\;L_2 \cos q_2 \\
0 & 0 & \boldsymbol{1} & q_1 \\
0 & 0 & \boldsymbol{0} & 1
\end{bmatrix}
\]

\[
A^0_3 =
\begin{bmatrix}
\cos(q_2+q_3) & 0 & \boldsymbol{-\sin(q_2+q_3)} & -L_2 \sin q_2 + L_3 \cos(q_2+q_3) \\
\sin(q_2+q_3) & 0 & \boldsymbol{\;\;\cos(q_2+q_3)} & \;\;\;L_2 \cos q_2 + L_3 \sin(q_2+q_3) \\
0 & -1 & \boldsymbol{0} & q_1 \\
0 & 0 & \boldsymbol{0} & 1
\end{bmatrix}
\]

\[
J_\omega(q) =
\begin{bmatrix}
0 & 0 & 0 & 0\\
0 & 0 & 0 & 0\\
1 & 0 & 1 & 0
\end{bmatrix}
\]

Observar que las columnas 2 y 4 correponden con articulaciones prismáticas, y por tanto, no aportan
velocidad angular y su vector es $(0, 0, 0)$. \\

La velocidad angular será: $
\mathbf{\omega}(q,\dot q) \;=\; J_\omega(q)\,
\begin{bmatrix}
\dot q_1 \\
\dot q_2 \\
\dot q_3 \\
\dot q_4
\end{bmatrix}
$

Resolvemos en \textit{Matlab} :
\lstinputlisting[]{Matlab/tema4/ejercicio11_symb_b.m}
\lstinputlisting[]{Matlab/tema4/ejercicio11_salida.txt}



\end{enumerate}