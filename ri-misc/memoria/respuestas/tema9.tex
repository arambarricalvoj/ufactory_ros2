\begin{table}[htb]
\centering
\begin{tabular}{|c|c|c|c|c|c|c|c|c|c|}
\hline
\rowcolor[HTML]{000000} 
{\color[HTML]{FFFFFF} 1} & {\color[HTML]{FFFFFF} 2} & {\color[HTML]{FFFFFF} 3} & {\color[HTML]{FFFFFF} 4} & {\color[HTML]{FFFFFF} 5} & {\color[HTML]{FFFFFF} 6} & {\color[HTML]{FFFFFF} 7} & {\color[HTML]{FFFFFF} 8} & {\color[HTML]{FFFFFF} 9} & {\color[HTML]{FFFFFF} 10} \\
\hline
\rowcolor[HTML]{FFFFFF} 
c & a, b, d & a & c & b & c, d & b & a & d & c \\
\hline
\rowcolor[HTML]{000000} 
{\color[HTML]{FFFFFF} 11} & {\color[HTML]{FFFFFF} 12} & {\color[HTML]{FFFFFF} 13} & {\color[HTML]{FFFFFF} 14} & {\color[HTML]{FFFFFF} 15} & {\color[HTML]{FFFFFF} 16} & {\color[HTML]{FFFFFF} } & {\color[HTML]{FFFFFF} } & {\color[HTML]{FFFFFF} } & {\color[HTML]{FFFFFF} } \\
\hline
\rowcolor[HTML]{FFFFFF} 
b & a & d & a, b & c & - &  &  &  &  \\
\hline
\end{tabular}
\caption{Tabla de respuestas de las cuestiones del noveno tema}
\label{table:tema9}
\end{table}


\begin{enumerate}
\item \textbf{¿En qué consiste la definición del \textit{lay-out?}} (p. 402)

La respuesta correcta es la c), que se ha extraído de la página 402 del libro: \\
``\textcolor{gray}{\textit{La definición del lay-out del sistema es un proceso iterativo del que debe resultar la
especificación del tipo y número de robots a utilizar, así como de los elementos periféricos, 
indicando la disposición relativa de los mismos. En este proceso iterativo es clave la experiencia
del equipo técnico responsable del diseño. La utilización de herramientas informáticas,
como sistemas CAD, simuladores específicos para robots y simuladores de sistemas de fabricación flexible facilitan enormemente esta tarea.}}`` \\

\item \textbf{¿Qué tipo de robots son adecuados para situarse en el centro de la célula?} (p. 403) (3 respuestas)

Las respuestas correctas son la a), b) y d) que corresponden con el siguiente texto de la página 403 del libro: \\
``\textcolor{gray}{\textit{En esta disposición el robot se sitúa de modo que quede rodeado por el resto de elementos
que intervienen en la célula. Se trata de una disposición típica para robots de estructura articular,
polar, cilíndrica o SCARA, en la que se puede aprovechar al máximo su campo de acción, que presenta una forma básica de esfera.}}`` \\

\item \textbf{¿Los robots en línea pueden trabajar con un transporte continuo de piezas?} (p. 404)

La respuesta correcta es la a), que corresponde con el siguiente texto de la página 404 del libro: \\
``\textcolor{gray}{\textit{Si el transporte es continuo, esto es, si las piezas no se detienen delante del robot, éste 
deberá trabajar sobre la pieza en movimiento, para lo que el transporte deberá limitar su velocidad
de modo que la pieza quede dentro del alcance del robot durante al menos el tiempo de ciclo.}}`` \\

\item \textbf{¿Cuándo son útiles los robots móviles sobre una plataforma?} (p. 405)

La respuesta correcta es la c), que corresponde con el siguiente texto de la página 405 del libro: \\
``\textcolor{gray}{\textit{Otra situación en la que el empleo del robot con capacidad de desplazamiento lineal es
particularmente ventajosa es cuando éste debe cubrir un elevado campo de acción. Por ejemplo, en la pintura de carrocerías de coches,
el dotar al robot de este grado de libertad adicional permite que dos robots de dimensiones medias (2 metros de radio de alcance aproximadamente)
lleguen con la orientación adecuada a todos los puntos de proyección correspondientes a un coche.}}`` \\

\item \textbf{¿Qué ventaja tienen los robots suspendidos?} (p. 406)

La respuesta correcta es la b), que corresponde con el siguiente texto de la página 406 del libro: \\
``\textcolor{gray}{\textit{Las ventajas fundamentales que se obtienen en este segundo caso son las de un mejor
aprovechamiento del área de trabajo, pues de este modo el robot puede acceder a puntos situados sobre su propio eje vertical.}}`` \\

\item \textbf{¿Qué funciones tiene que contemplar un buen sistema de control para una célula robotizada?} (p. 407) (2 respuestas)

Las respuestas correctas son la c) y d), que corresponden con los puntos expuestos en el apartado 9.1.2 de la página 407 del libro: \\

\item \textbf{¿Qué recomienda la norma EN ISO 9946?} (pp. 408 y 409)

La respuesta correcta es la b), que corresponde con el siguiente texto de las páginas 408 y 409 del libro: \\
``\textcolor{gray}{\textit{La norma EN ISO 9946: «1999 Robots Manipuladores Industriales. Presentación de las
características», [EN-99] recomienda cuáles deben se las características de los robots a presentar por sus fabricantes, aconsejando,
incluso, el modo en que éstas son recogidas en una ficha. Esta norma se complementa con la UNE EN ISO 9283 - 2003 «Criterios de análisis de
prestaciones» [UNE-03], en la que se especifica el modo en que deben ser medidas muchas
de estas características.}}`` \\

\item \textbf{¿Qué es el área de trabajo?} (p. 409)

La respuesta correcta es la a), que corresponde con el siguiente texto de la página 409 del libro: \\
``\textcolor{gray}{\textit{El área de trabajo o campo de acción es el volumen espacial al que puede llegar el extremo
del robot. Este volumen está determinado por el tamaño, forma y tipo de los eslabones que integran el robot (Figura 9.7),
así como por las limitaciones de movimiento impuestas por el sistema de control. Nunca deberá utilizarse el efector colocado
en la muñeca para la obtención del espacio de trabajo, ya que se trata de un elemento añadido al robot, y en el caso de variar
el efector el área de trabajo se tendría que calcular de nuevo.}}`` \\

\item \textbf{¿Qué es la repetibilidad y por qué es el criterio utilizado para seleccionar un robot u otro según su exactitud?} (p. 412)

La respuesta correcta es la d), que corresponde con la definición descrita en la página 412 del libro. Las respuestas b) y a) se refieren a las 
definiciones de resolución y precisión, respectivamente. \\

\newpage
\item \textbf{¿Qué indica la norma UNE EN ISO 9283:2003?} (p. 413)

La respuesta correcta es la c), que corresponde con el siguiente texto de la página 413 del libro: \\
``\textcolor{gray}{La norma UNE EN ISO 9283: 2003 «Robots Manipuladores Industriales. Criterios de
análisis de prestaciones y métodos de ensayo relacionados», [UNE-03] incide de manera especial el modo en que
debe ser evaluada la calidad del posicionamiento y movimiento de un robot.}`` \\

\item \textbf{¿Qué dato de velocidad se suele utilizar para calcular el tiempo de ciclo de un robot industrial?} (p. 414)

La respuesta correcta es la b), que corresponde con el siguiente texto de la página 414 del libro: \\
``\textcolor{gray}{En la práctica, en la mayoría de los casos los movimientos del robot son rápidos y cortos,
con lo que la velocidad nominal es alcanzada en contadas ocasiones. Por este motivo, la me-
dida del tiempo de ciclo no puede ser obtenida a partir de la velocidad, siendo ésta una va-
loración cualitativa del mismo. En vez de este dato, algunos robots indican el tiempo em-
pleado en realizar un movimiento típico (un pick \& place, por ejemplo).}`` \\

\item \textbf{En cuanto a la capacidad de carga, ¿qué habitúan a entregar los fabricantes?} (p. 414)

La respuesta correcta es la a), que corresponde con el siguiente texto de la página 414 del libro: \\
``\textcolor{gray}{... el fabricante puede proporcionar un cuadro en el que se indica la disminución de la posible carga a transportar,
sin disminuir prestaciones, a medida que el centro de gravedad de la misma se aleja del centro de la muñeca.}`` \\

\item \textbf{¿Qué proporciona la norma ISO 10218-1:2006?} (p. 418)

La respuesta correcta es la d), que corresponde con el siguiente texto de la página 418 del libro: \\
``\textcolor{gray}{En Europa, la norma ISO 10218-1:2006 «Robots for industrial environments - Safety requirements - Part 1: Robot» 
[ISO-06], proporciona una guía para garantizar la seguridad en los sistemas de fabricación automatizados que incorporan robots manipuladores}`` \\

\item \textbf{¿Qué accidentes causados por robots industriales son más comunes?} (p. 418) (2 respuestas)

Las respuestas correctas son la a) y b), que corresponden con el apartado 9.3.1 de la página 418 del libro. \\

\item \textbf{¿Qué costes y factores hay que tener en cuenta en la implantación de un robot indsutrial?} (pp. 422-424)

La respuesta correcta es la c), que corresponden con el apartado 9.4.1 de la página 422 del libro. \\

\newpage
\item \textbf{Calcular la precisión y repetibilidad de un robot dada la siguiente tabla, donde $A$ es el punto 
de partida:}

\begin{table}[htb]
\centering
\begin{tabular}{>{\bfseries}c c c c}
\toprule
Punto & $x$ (mm) & $y$ (mm) & $z$ (mm) \\
\midrule
$A$   & 0.0 & 0.0 & 0.0 \\
$B$   & 3.0 & 3.0 & 3.0 \\
$B_1$ & 2.9 & 2.8 & 3.0 \\
$B_2$ & 2.9 & 2.7 & 3.0 \\
$B_3$ & 2.7 & 2.7 & 3.0 \\
$B_4$ & 2.6 & 3.0 & 2.9 \\
\bottomrule
\end{tabular}
\caption{Coordenadas de los puntos programados y observados de la pregunta 16.}
\end{table}

\begin{enumerate}
\item \textbf{Baricentro de los puntos alcanzados (estimador insesgado del
 “punto medio” de las repeticiones y minimiza la suma de distancias
 cuadráticas):}
\[
B_m = \left(\frac{2.9+2.9+2.7+2.6}{4}, \frac{2.8+2.7+2.7+3.0}{4}, \frac{3.0+3.0+3.0+2.9}{4}\right)\]
\[= (2.775,\ 2.800,\ 2.975)\ \text{mm}\] \\

\item \textbf{Exactitud o precisión de posicioonamiento:} distancia entre el punto programado $B$ y el baricentro $B_m$:
\[
A_p = \sqrt{(3-2.775)^2 + (3-2.800)^2 + (3-2.975)^2} \approx 0.3021\ \text{mm}
\]\\

\item \textbf{Repetibilidad:}
\begin{itemize}
\item Distancias de cada punto al baricentro:
\[
L_1 = 0.1275,\quad L_2 = 0.1639,\quad L_3 = 0.1299,\quad L_4 = 0.2773\ \text{mm}
\]

\item Media:
\[
L_m = \frac{1}{n}\sum_{j=1}^{n} L_j = \frac{1}{4}\sum L_j \approx 0.1747\ \text{mm}
\]

\item Desviación estándar muestral:
\[
L_s = \sqrt{\frac{1}{n-1}\sum_{j=1}^{n} (L_j - L_m)^2} = \sqrt{\frac{1}{3}\sum (L_j - L_m)^2} \approx 0.0703\ \text{mm}
\]

\item Repetibilidad:
\[
RP = L_m + 3L_s \approx 0.3847\ \text{mm}
\]

\end{itemize}
\end{enumerate}

\newpage
Si resolvemos en \textit{Matlab}:
\lstinputlisting[]{Matlab/tema9/metrologia.m}
\lstinputlisting[]{Matlab/tema9/metrologia_salida.txt}

\begin{figure}[htbp]
    \centering
    \includegraphics[width=0.8\linewidth]{Images/tema9/metrologia.png}
    \caption{Resultado gráfico de la pregunta 16 (metrología).}
    \label{fig:metrología}
\end{figure}
\end{enumerate}

