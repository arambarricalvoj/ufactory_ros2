\begin{table}[htb]
\centering
\begin{tabular}{|c|c|c|c|c|c|c|c|c|c|}
\hline
\rowcolor[HTML]{000000} 
{\color[HTML]{FFFFFF} 1} & {\color[HTML]{FFFFFF} 2} & {\color[HTML]{FFFFFF} 3} & {\color[HTML]{FFFFFF} 4} & {\color[HTML]{FFFFFF} 5} & {\color[HTML]{FFFFFF} 6} & {\color[HTML]{FFFFFF} 7} & {\color[HTML]{FFFFFF} 8} & {\color[HTML]{FFFFFF} 9} & {\color[HTML]{FFFFFF} 10} \\
\hline
\rowcolor[HTML]{FFFFFF} 
a & d & b & c & b & c & c & d & a & b \\
\hline
\rowcolor[HTML]{000000} 
{\color[HTML]{FFFFFF} 11} & {\color[HTML]{FFFFFF} 12} & {\color[HTML]{FFFFFF} 13} & {\color[HTML]{FFFFFF} 14} & {\color[HTML]{FFFFFF} 15} & {\color[HTML]{FFFFFF} 16} & {\color[HTML]{FFFFFF} 17} & {\color[HTML]{FFFFFF} } & {\color[HTML]{FFFFFF} } & {\color[HTML]{FFFFFF} } \\
\hline
\rowcolor[HTML]{FFFFFF} 
b & c & b & a & b & - & - &  &  &  \\
\hline
\end{tabular}
\caption{Tabla de respuestas de las cuestiones del séptimo tema}
\label{table:tema7}
\end{table}


\begin{enumerate}
\item \textbf{¿Cuál es el objetivo del control dinámico?} (p. 309)

La respuesta correcta es la a), que corresponde con el siguiente texto de la página 309 del libro: \\
``\textcolor{gray}{\textit{El control dinámico tiene por misión procurar que las trayectorias realmente seguidas por
el robot q(t) sean lo más parecidas posibles a las propuestas por el control cinemático. Para
ello hace uso del conocimiento del modelo dinámico del robot (obtenido en el Capítulo 5) y de
las herramientas de análisis y diseño aportadas por la teoría del servocontrol (representación
interna, estado, estabilidad de Lyapunov, control PID, control adaptativo, etc.).}}`` \\

\item \textbf{La principal característica del control dinámico es...} (p. 309)

La respuesta correcta es la d), que corresponde con el siguiente texto de la página 309 del libro: \\
``\textcolor{gray}{\textit{Como se vio al obtener el modelo dinámico de un robot, éste es fuertemente no lineal,
multivariable, acoplado y de parámetros variantes, por lo que en, general, su control es extremadamente complejo.
En la práctica, ciertas simplificaciones, válidas para un gran número de
los robots existentes, facilitan el diseño del sistema de control, dando unos resultados razonablemente aceptables,
aunque limitando en ciertas situaciones la calidad de sus prestaciones.}}`` \\

\item \textbf{¿Cuándo se pueden considerar las articulaciones del robot desacopladas?} (p. 312)

La respuesta correcta es la b), que corresponde con el siguiente texto de la página 312 del libro: \\
``\textcolor{gray}{\textit{Obsérvese que si los factores de reducción son elevados, esto es, si las velocidades del
movimiento de los actuadores $q_a$ son notablemente mayores que las de las articulaciones $q$, se
tendrá que los elementos de la matriz diagonal $K$ serán muy superiores a la unidad, y consecuentemente
$K^{-1}$ será una matriz diagonal de elementos de reducido valor. En este caso, el término 
$\tau_p$ puede llegar a ser despreciable frente al par $\tau_a$ y el par de rozamiento, permitiendo
considerar que las articulaciones del robot están desacopladas.}}`` \\

\item \textbf{¿Es viable el control en cadena abierta con el uso exclusivo de la pre-alimentación?} (p. 316)

La respuesta correcta es la c), que corresponde con el siguiente texto de la página 316 del libro:  \\
``\textcolor{gray}{\textit{El conocimiento exacto de los valores de K, J y B no es por lo general posible, siendo preciso conformarse,
a lo sumo, con disponer de valores aproximados, lo que impedirá la cancelación exacta de la dinámica del sistema y,
por tanto, q y qd no serán exactamente iguales.
Por su parte el término τp engloba tanto las perturbaciones externas del robot como los pares procedentes del acoplamiento del
movimiento de los demás grados de libertad, cuyo valor no será nulo. Las consecuencias de este valor no nulo del par perturbador pueden ser
inadmisibles, haciendo inviable el control en cadena abierta planteado con el uso exclusivo de la pre-alimentación.}}`` \\

\item \textbf{¿Qué soluciona el control realimentado?} (p. 317)

La respuesta correcta es la b), que corresponde con el siguiente texto de la página 317 del libro: \\
``\textcolor{gray}{\textit{Los problemas derivados del desconocimiento preciso del modelo y de la aparición de
perturbaciones pueden ser resueltos adecuadamente por el control por realimentación que será
la próxima alternativa a considerar.}}`` \\

\item \textbf{¿Qué controlador mínimo es necesario para el control de la articulación?} (p. 320)

La respuesta correcta es la c), que corresponde con el siguiente texto de la página 320 del libro: \\
``\textcolor{gray}{\textit{En consecuencia, el uso de la realimentación, si bien no consigue igualar la salida q(t) con
la referencia qd (t) en todo instante de tiempo, como en condiciones ideales conseguía la prealimentación, si que logra que pasado un cierto tiempo
(régimen permanente) ambas coincidan, bastando para ello, en caso de qd escalón, con el uso de un regulador proporcional.
Más aún, mientras el control por pre-alimentación era incapaz de actuar ante una perturbación,
resultando que la articulación se movería sin control, el control realimentado ante una perturbación en escalón, es capaz de parar el movimiento con el uso
un regulador proporcional e incluso
de acabar compensando el efecto de aquélla, si se utiliza regulador proporcional integral.
Dado que las perturbaciones siempre estarán presentes, al menos como efecto del resto de
las articulaciones y puesto que no es admisible un error en el posicionamiento del robot, cabe
concluir la necesidad de usar un regulador de tipo proporcional integral para el control de la
articulación.}}`` \\

\item \textbf{¿Cuál es la ventaja principal de combinar el control realimentado con el control pre-alimentado (PID + FF)?} (p. 320)

La respuesta correcta es la c), que corresponde con el siguiente texto de la página 320 del libro: \\
``\textcolor{gray}{\textit{el control pre-alimentado presenta características
inmejorables en condiciones ideales (perfecto conocimiento en todo instante del modelo y ausencia de perturbaciones)
pero es incapaz de trabajar correctamente cuando éstas no se dan.
Por su parte, el control realimentado no llega al comportamiento ideal del pre-alimentado,
consiguiendo que la salida q iguale a la referencia qd sólo pasado un cierto tiempo (régimen
permanente) y no desde el primer momento como el pre-alimentado. Sin embargo al contrario que aquél,
es robusto, consiguiendo a la salida al valor deseado, incluso cuando no se conoce perfectamente la
dinámica del sistema o cuando se presentan ciertas perturbaciones.
Cabe pues considerar la posibilidad de combinar el efecto de ambos modos de control, de
modo que ante las condiciones ideales de conocimiento perfecto de la dinámica y ausencia de
perturbaciones q y qd sean iguales en todo momento, mientras que en caso de no darse estas
circunstancias, el sistema siga controlado, llegando a igualar el valor real de la referencia pasado cierto tiempo.
La combinación de ambos modos de control (Figura 7.11) consigue este efecto.
Al usar este control (que se denominará PID + FF) se tiene que: ...}}`` \\

\item \textbf{¿Qué estrategia permite eliminar el error en régimen permanente frente a las perturbaciones gravitatorias sin emplear acción integral en el regulador?} (p. 322)

La respuesta correcta es la d), que corresponde con el siguiente texto de la página 322 del libro: \\
``\textcolor{gray}{\textit{el efecto en régimen permanente de la acción integral sobre el error ante perturbación,
puede ser sustituido por la prealimentación del término $\frac{C(q)}{K}$ obtenido a partir del
par de gravedad originado sobre una articulación por el resto.}}`` \\

\item \textbf{¿Qué establece la ley de control del control multiarticular PID con pre-alimentación?} (p. 325)
 
La respuesta correcta es la a), que corresponde con el siguiente texto de la página 325 del libro: \\
``\textcolor{gray}{\textit{Siendo ésta la denominada ley de control que establece el cálculo del par que debe realizar
la unidad de control para fijar las consignas de par de los actuadores. La evaluación en tiempo real de la Expresión [7.37] 
puede ser excesivamente costosa. En ocasiones, y considerando que las velocidades q de las articulaciones no son muy elevadas
(del orden de una vuelta por segundo) puede simplificarse su cálculo, despreciando el efecto de las matrices H(q, q)
y F v siendo sólo preciso evaluar D(q) y C(q). La Expresión [7.37] es el caso general de control por par calculado y según se
consideren o no ciertos términos, se tendrán diferentes casos particulares.}}`` \\

\item \textbf{¿Cuál es el propósitvo del control adaptativo y qué tipos principales se diferencian?} (p. 326)

La respuesta correcta es la b), que corresponde con el siguiente texto de la página 326 del libro: \\
``\textcolor{gray}{\textit{La técnica del control adaptativo se puede aplicar con buenos resultados en aquellas ocasiones en las que el modelo,
aun siendo conocido, cambia continuamente por variar las condiciones de funcionamiento. La idea básica del control
adaptativo es modificar en tiempo real los parámetros que definen el regulador (PID por ejemplo) de acuerdo al comportamiento
instántaneo del sistema [RODRIGUEZ-96]. En el caso de un robot, es evidente que el comportamiento del sistema y, por tanto, el modelo
del mismo, cambia con los valores de sus variables articulares y con la carga que transporta. De este modo,
supuesto que se hubiese ajustado adecuadamente un controlador para unas condiciones determinadas de localización del robot
y carga manipulada, este ajuste no sería válido cuando alguna de estas condiciones cambiase. [...]
Existen diferentes esquemas de control que entran dentro del concepto de control adaptativo. [...]
En todos ellos se utilizan procedimientos de identificación de los parámetros del modelo del sistema y
El control por planificación de ganancias puede ser considerado como un caso simple de control adaptativo, en el que un número determinado de reguladores están precalculados
para diferentes condiciones de funcionamiento. Según se detecta que éstas se modifican, se conmuta de un regulador a otro.
El control adaptativo con modelo de referencia realiza una comparación en línea entre el comportamiento del sistema real y el deseado. El error entre ambos comportamientos se
utiliza para modificar los parámetros del regulador.
El control adaptativo autoajustable identifica continuamente el modelo del sistema y
utiliza algún algoritmo de diseño para ajustar los parámetros del regulador. Como caso particular de éste se comentará
el control de par calculado adaptativo.}}`` \\

\item \textbf{¿Los parámetros de control Kp, Ki, Kd son únicos para todo el robot?} (p. 328)

La respuesta correcta es la b), que corresponde con el siguiente texto de la página 328 del libro: \\
``\textcolor{gray}{\textit{Téngase en cuenta que la determinación de los parámetros ka, kp, ki, kd debe realizarse para
cada uno de los grados de libertad sobre los que se desee reajustar el regulador.}}`` \\

\item \textbf{¿Cuál de las siguientes estrategias puede emplearse para hacer frente a las oscilaciones no deseables generadas por la flexibilidad estructural de los eslabones?} (p. 337)

La respuesta correcta es la c), que se expone en la página 337 del libro. \\


\item \textbf{Un robot de una articulación está sometido a la dinámica:}
\[
J \ddot{q}(t) + B \dot{q}(t) + C(q) = \tau(t)
\]

\textbf{donde $J=2 \,\text{kg·m}^2$, $B=1 \,\text{N·m·s/rad}$ y $C(q)=5 \sin(q)$ representa el par de gravedad.  
Se aplica un controlador PD con compensación de gravedad:}

\[
\tau(t) = K_p (q_d - q) + K_d (\dot{q}_d - \dot{q}) + C(q)
\]

\textbf{Si la referencia es $q_d = \pi/4$ rad y $\dot{q}_d = 0$, estando el robot inicialmente en $q=0$ rad y $\dot{q}=0$, ¿cuál es el par de control inicial $\tau(0)$?}

La respuesta correcta es la b), porque al evaluar la ley de control en $t=0$ se tiene:  

\[
\tau(0) = K_p \left(\tfrac{\pi}{4} - 0\right) + K_d (0 - 0) + 5 \sin(0)
\]

El término derivativo se anula ya que las velocidades iniciales son cero, y el
par de gravedad también se anula porque $\sin(0)=0$. Por tanto, el par inicial
depende únicamente del error de posición multiplicado por la ganancia
proporcional $K_p$. \\

\item \textbf{Un robot de una articulación tiene la dinámica:}
\[
J \ddot{q}(t) + B \dot{q}(t) + C(q) = \tau(t)
\]

\textbf{con $J=2 \,\text{kg·m}^2$, $B=1 \,\text{N·m·s/rad}$ y $C(q)=5 \sin(q)$.  
El controlador de par calculado adaptativo utiliza parámetros estimados $\hat{J}=2$, $\hat{B}=1$ y compensa la gravedad con $\hat{C}(q)=5 \sin(q)$.  
La ley de control es:}

\[
\tau(t) = \hat{J}\ddot{q}_d + \hat{B}\dot{q}_d + \hat{C}(q) + K_p (q_d - q) + K_d (\dot{q}_d - \dot{q})
\]

\textbf{Si la referencia es $q_d = \pi/3$ rad, $\dot{q}_d = 1$ rad/s, $\ddot{q}_d = 0$, y el robot está inicialmente en $q=0$ rad, $\dot{q}=0.5$ rad/s, con ganancias $K_p=8$, $K_d=4$, ¿cuál es el par de control inicial $\tau(0)$?}
\newpage
La respuesta correcta es la a), porque al evaluar la ley de control en
$t=0$ se obtiene:

\[
\tau(0) = 0 + 1 + 0 + 8 \cdot \tfrac{\pi}{3} + 4 \cdot 0.5
= 1 + \tfrac{8\pi}{3} + 2 \approx 11.38 \,\text{N·m}
\]

De este modo, el par inicial incluye tanto la acción proporcional por el error
de posición como la acción derivativa por la diferencia de velocidades, además
de la compensación de la referencia de velocidad. \\


\item \textbf{Un robot de una articulación tiene la dinámica:}
\[
J \ddot{q}(t) + B \dot{q}(t) + C(q) = \tau(t)
\]

\textbf{con $J=3 \,\text{kg·m}^2$, $B=2 \,\text{N·m·s/rad}$ y $C(q)=4 \sin(q)$.  
Se aplica un controlador PID con prealimentación:}

\[
\tau(t) = J \ddot{q}_d + B \dot{q}_d + C(q_d) + K_p (q_d - q) + K_i \int (q_d - q)\,dt + K_d (\dot{q}_d - \dot{q})
\]

\textbf{Si la referencia es $q_d = \pi/4$ rad, $\dot{q}_d = 1$ rad/s, $\ddot{q}_d = 0.5$ rad/s$^2$, y el robot está inicialmente en $q=0$ rad, $\dot{q}=0.5$ rad/s, con ganancias $K_p=6$, $K_i=2$, $K_d=3$, y suponiendo que la integral del error acumulado hasta $t=0$ es $I_e=0.2$, ¿cuál es el par de control inicial $\tau(0)$?}

La respuesta correcta es la b), porque al evaluar la ley de control en $t=0$ se obtiene:
\[
\tau(0) = 3 \cdot 0.5 + 2 \cdot 1 + 4 \sin\!\left(\tfrac{\pi}{4}\right) + 6 \cdot \tfrac{\pi}{4} + 2 \cdot 0.2 + 3 \cdot (1 - 0.5)
\]

\[
= 1.5 + 2 + 2.83 + 4.71 + 0.4 + 1.5 \approx 14.36 \,\text{N·m}
\]

De este modo, el controlador PID con prealimentación combina la compensación
dinámica (inercia, rozamiento y gravedad) con las acciones proporcional,
integral y derivativa, logrando un par inicial que reduce el error y mejora
el seguimiento. \\



\item \textbf{Considerar un robot de dos articulaciones con dinámica simplificada:}

\[
M(q)\ddot{q} + C(q,\dot{q})\dot{q} + G(q) = \tau
\]

\textbf{donde $M(q)$ es la matriz de inercia, $C(q,\dot{q})$ los términos de Coriolis y $G(q)$ el vector de gravedad.}  

\textbf{Se pide diseñar un controlador realimentado (sin prealimentación de modelo), explicando los siguientes 
puntos:}
\begin{enumerate}
  \item \textbf{La ley de control que se implementaría para un regulador PD y para un regulador PID multivariable.}
  \item \textbf{Si la referencia es constante $q_d$, cuándo se garantiza que el error en régimen permanente frente a la gravedad sea nulo.}
  \item \textbf{Qué limitaciones tiene este esquema si el modelo dinámico no se compensa explícitamente.}
\end{enumerate}

\begin{enumerate}
  \item La ley de control realimentado para un regulador PD es:
  

\[
  \tau = K_p (q_d - q) + K_d (\dot{q}_d - \dot{q})
  \]


  y para un regulador PID:
  

\[
  \tau = K_p (q_d - q) + K_d (\dot{q}_d - \dot{q}) + K_i \int_0^t (q_d - q)\,dt
  \]


  donde $K_p, K_d, K_i$ son matrices de ganancias positivas (habitualmente diagonales). \\

  \item En régimen permanente, si $q_d$ es constante:
  \begin{itemize}
    \item Con PD puro, el error estacionario es:
    

\[
    e_\infty = q_d - q_\infty = K_p^{-1} G(q_\infty)
    \]


    por lo que no se anula salvo que $G(q_\infty)=0$.  \\
    \item Con PID, la acción integral acumula el par necesario para compensar $G(q)$, garantizando que:
    

\[
    e_\infty = 0
    \]


    frente a perturbaciones gravitatorias constantes.\\
  \end{itemize}

  \item Limitaciones del esquema:
  \begin{itemize}
    \item Con PD sin integral, siempre existe error estático frente a la gravedad.
    \item No se compensan explícitamente los acoplamientos entre articulaciones ni las no linealidades de $M(q)$ y $C(q,\dot{q})$.
    \item La precisión depende fuertemente de las ganancias elegidas; valores altos pueden provocar oscilaciones o saturación.
    \item Cambios en la carga o en la configuración del robot alteran $G(q)$ y degradan el desempeño si no se ajustan las ganancias. \\
  \end{itemize}
\end{enumerate}


\item \textbf{Se desea diseñar un regulador PID para una articulación de un robot cuya dinámica
simplificada se aproxima por un sistema de segundo orden:}
\[
G(s) = \frac{1}{J s^2 + B s}
\]

\textbf{con $J = 2 \,\text{kg·m}^2$ y $B = 4 \,\text{N·m·s/rad}$.}  

\textbf{Las especificaciones de diseño son:  
\begin{itemize}
  \item Tiempo de establecimiento $t_s < 2$ s.  
  \item Sobreimpulso máximo $M_p < 10\%$.  
  \item Error en régimen permanente nulo para una entrada escalón de posición.  
\end{itemize}}

\newpage
\textbf{Se pide:  
\begin{enumerate}
  \item Determinar qué acciones del PID son necesarias para cumplir cada especificación. 
  \item Determinar los valores de la razón de amortiguamiento $\zeta$ y la frecuencia natural $\omega_n$ que cumplen las especificaciones de $t_s$ y $M_p$.   
  \item A partir de estos valores, calcular los parámetros aproximados $K_p$, $K_i$, $K_d$ del regulador PID.  
  \item Justificar cómo cada parámetro afecta al comportamiento dinámico del sistema.  
\end{enumerate}}
\begin{enumerate}
  \item Para cumplir las especificaciones:  
  \begin{itemize}
    \item El error en régimen permanente nulo requiere acción integral, $K_i > 0$.  
    \item El tiempo de establecimiento reducido se logra aumentando $K_p$, lo que incrementa la rapidez de respuesta.  
    \item El sobreimpulso limitado se controla con la acción derivativa $K_d$, que añade amortiguamiento. \\
  \end{itemize}

  \item Para un sistema de segundo orden (de la teoría de control):  
  
\[
  t_s \approx \frac{4}{\zeta \omega_n}, \quad M_p \approx e^{-\frac{\pi \zeta}{\sqrt{1-\zeta^2}}}
\]

Condición $t_s < 2$:  $\zeta \omega_n > 2$

Condición $M_p < 0.1$: se obtiene aproximadamente $\zeta > 0.6$.\\  

Para despejar $\zeta$ de la expresión del sobreimpulso:  
  
\[
  \ln(M_p) = -\frac{\pi \zeta}{\sqrt{1-\zeta^2}}
  \]

Definiendo $A = -\ln(M_p)$, se llega a:  
  
\[\zeta = \sqrt{\frac{A^2}{A^2 + \pi^2}}, \quad A = -\ln(M_p) \] \\

Ejemplo: para $M_p = 0.1$, $A \approx 2.302$ y $\zeta \approx 0.59$.  

Tomando $\zeta = 0.7$, se requiere $\omega_n > \frac{2}{0.7} \approx 2.86$ rad/s. \\

  \item El regulador PID debe situar los polos dominantes en $s^2 + 2\zeta \omega_n s + \omega_n^2 = 0$.  
  
Para $\zeta=0.7$, $\omega_n=3$ rad/s:  
\[
  s^2 + 4.2s + 9 = 0
\]

Comparando con la dinámica del sistema y aplicando el método de asignación de polos se obtienen las ganancias aproximada.
Este método consiste en comparar la ecuación característica deseada con la ecuación característica real del sistema cerrado bajo control PID.  
En general, el denominador del sistema controlado con PID toma la forma:
\[
s^2 + K_d s + K_p + K_i/s
\]
o, si se considera la parte proporcional y derivativa en la ecuación característica dominante (ignorando el término integral para el cálculo inicial de polos):
\[
s^2 + K_d s + K_p = 0
\]
De esta manera, se igualan los coeficientes de la ecuación deseada con los del sistema controlado:
\[
K_d \approx 4.2, \quad K_p \approx 9
\]
El término integral $K_i$ no afecta directamente a la ubicación de los polos
dominantes, pero se introduce para garantizar error nulo en régimen permanente
frente a entradas escalón. Por ello se toma un valor positivo moderado:
\[
  K_p \approx 9, \quad K_d \approx 4.2, \quad K_i > 0 \text{ (para eliminar error estático)}
\]

Un valor razonable es $K_i = 2$, ya que:  
  \begin{itemize}
    \item Si $K_i$ es demasiado pequeño, el error se corrige muy lentamente.  
    \item Si $K_i$ es demasiado grande, el sistema puede oscilar o volverse inestable.  
    \item El valor $K_i = 2$ mantiene un equilibrio con $K_p$ y $K_d$, asegurando corrección del error estático en un tiempo razonable sin introducir oscilaciones excesivas. \\  
  \end{itemize} 

  \item Justificación:  
  \begin{itemize}
    \item Acción proporcional ($K_p$): aumenta la rapidez de respuesta y fija la posición de los polos, aunque si es excesiva puede generar sobreimpulso.  
    \item Acción integral ($K_i$): elimina el error estático acumulando la diferencia entre referencia y salida, asegurando error nulo en régimen permanente frente a entradas escalón.  
    \item Acción derivativa ($K_d$): introduce amortiguamiento anticipando la tendencia del error, reduciendo oscilaciones y limitando el sobreimpulso.  
  \end{itemize}
\end{enumerate}

\end{enumerate}