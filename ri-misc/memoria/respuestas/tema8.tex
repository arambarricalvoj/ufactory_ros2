\begin{table}[htb]
\centering
\begin{tabular}{|c|c|c|c|c|c|c|c|c|c|}
\hline
\rowcolor[HTML]{000000} 
{\color[HTML]{FFFFFF} 1} & {\color[HTML]{FFFFFF} 2} & {\color[HTML]{FFFFFF} 3} & {\color[HTML]{FFFFFF} 4} & {\color[HTML]{FFFFFF} 5} & {\color[HTML]{FFFFFF} 6} & {\color[HTML]{FFFFFF} 7} & {\color[HTML]{FFFFFF} 8} & {\color[HTML]{FFFFFF} 9} & {\color[HTML]{FFFFFF} 10} \\
\hline
\rowcolor[HTML]{FFFFFF} 
d & b & d & d & c & a & c & b & a & d \\
\hline
\rowcolor[HTML]{000000} 
{\color[HTML]{FFFFFF} 11} & {\color[HTML]{FFFFFF} 12} & {\color[HTML]{FFFFFF} 13} & {\color[HTML]{FFFFFF} 14} & {\color[HTML]{FFFFFF} 15} & {\color[HTML]{FFFFFF} } & {\color[HTML]{FFFFFF} } & {\color[HTML]{FFFFFF} } & {\color[HTML]{FFFFFF} } & {\color[HTML]{FFFFFF} } \\
\hline
\rowcolor[HTML]{FFFFFF} 
b & a & b & d & a &  &  &  &  &  \\
\hline
\end{tabular}
\caption{Tabla de respuestas de las cuestiones del octavo tema}
\label{table:tema8}
\end{table}


\begin{enumerate}
\item \textbf{¿Qué es la programación por guiado?} (p. 356)

La respuesta correcta es la d), que se ha extraído de la página 356 del libro. \\

\item \textbf{¿A qué hace referencia el guiado pasivo?} (p. 356)

La respuesta correcta es la b), que se ha extraído de la página 356 del libro. \\

\item \textbf{¿En el guiado activo las posiciones o configuraciones del robot se graban continuamente?} (p. 357)

La respuesta correcta es la d), que corresponde con el siguiente texto de la página 357 del libro: \\
``\textcolor{gray}{\textit{otra posibilidad permite emplear el propio sistema de
accionamiento del robot, controlado desde una botonera o un bastón de mando (joystick)
situado en un panel de programación portátil (Figura 8.2), para que sean los accionamientos
los encargados de mover las articulaciones. Se dirá entonces que se trata de un guiado activo.
Ejemplos de este tipo se encuentran en los robots de ABB o en los de KUKA. En
este caso, lo habitual es que la unidad de control únicamente registre aquellas configuraciones del robot que el programador indique expresamente.
De este modo, el proceso de programación consiste en mover mediante el panel de programación al robot hasta
una configuración determinada y grabar esta configuración. En este caso, el movimiento que ha llevado el robot hasta alcanzar la configuración
final es irrelevante, siendo necesario incorporar los datos que definen la trayectoria del robot desde la configuración
anterior hasta la nueva (habitualmente tipo de trayectoria, velocidad, precisión).}}`` \\

\item \textbf{¿Cuál es la principal consecuencia de la programación por guiado?} (pp. 357 y 358)

La respuesta correcta es la d), que corresponde con el siguiente texto entre las páginas 357 y 358 del libro: \\
``\textcolor{gray}{\textit{dado que el programador no hace uso directo de las coordenadas de los objetos del entorno,
la unidad de control no precisa de las funciones del modelado y control cinemático, desapareciendo, consecuentemente,
los posibles problemas asociados a éste como la existencia de soluciones múltiples o los puntos singulares.}}`` \\

\item \textbf{¿Qué incovenientes presenta la programación por guiado?} (p. 358)

La respuesta correcta es la c), que corresponde con el siguiente texto de la página 358 del libro: \\
``\textcolor{gray}{\textit{presentan una serie de inconvenientes, de los que el más destacable es la necesidad de utilizar
al propio robot y su entorno para realizar la programación, obligando a sacar al robot
de la línea de producción e interrumpiendo ésta. Otros inconvenientes frecuentes son la
inexistencia de una documentación del programa y la dificultad de realizar modificaciones en el mismo,
inconvenientes ambos que conducen a una difícil depuración y puesta a punto de las aplicaciones.
8.1.2. Programación}}`` \\

\item \textbf{¿Qué es la programación textual y cuál es su principal ventaja?} (p. 358)

La respuesta correcta es la a), que corresponde con el siguiente texto de la página 358 del libro: \\
``\textcolor{gray}{\textit{el método de programación textual permite
indicar la tarea al robot mediante el uso de un lenguaje de programación específico. Un programa se corresponde ahora,
como en el caso de un programa general, con una serie de órdenes que son editadas y posteriormente ejecutadas.
Existe, por tanto, un texto para el programa.
El texto del programa es editado en un sistema informático que puede ser independiente
del robot, no precisando por ello, a diferencia de la programación por guiado, la presencia de
éste durante la fase de desarrollo del programa. Por este motivo también se conoce a este método de programación como fuera de línea.}}`` \\

\item \textbf{¿Qué niveles de programación textual se clasifican?} (p. 358)

La respuesta correcta es la c), que corresponde con el siguiente texto de la página 358 del libro: \\
``\textcolor{gray}{\textit{La programación textual puede ser clasificada en tres niveles: robot, objeto y tarea, dependiendo
de que las órdenes se refieran a los movimientos a realizar por el robot, al estado en que deben ir quedando los objetos manipulados o al objetivo
(o subobjetivo parcial) a conseguir.}}`` \\

\item \textbf{¿Qué característica tiene la programación a nivel de tarea?} (p. 359)

La respuesta correcta es la b), que corresponde con el siguiente texto de la página 359 del libro: \\
``\textcolor{gray}{\textit{La programación a nivel tarea precisa de la resolución de complejos retos propios de la
Inteligencia Artificial, de entre los que puede citarse la planificación automática de modos
de agarre y de trayectorias del robot, mediante la cual el sistema de control debe decidir de
qué modo debe agarrar los objetos para garantizar su estabilidad durante el transporte a la
vez que se dejan libres las superficies que deberán entrar en contacto con el resto del entorno. 
Además el robot debe ser capaz de conocer el estado de su entorno y corregir continuamente las discrepancias entre el
modelo previsto y el estado real, haciendo uso para ello de un sistema sensorial y de un adecuado tratamiento de la información registrada por él.}}`` \\

\newpage
\item \textbf{¿Cuáles son los requerimientos generales para un sistema de programación de robots?} (p. 361)

La respuesta correcta es la a), que corresponde con los puntos expuestos en la página 361 del libro. \\

\item \textbf{¿Qué establece la norma UNE EN ISO 15187-2003?} (p. 362)

La respuesta correcta es la d), que corresponde con el siguiente texto de la página 362 del libro: \\
``\textcolor{gray}{\textit{... la tendencia al uso de interfases gráficas ha llevado al establecimiento
de la norma UNE EN ISO 15187-2003 [UNE-03], en la que se normalizan las interfases gráficas que el
usuario puede utilizar para la programación de robots (GUI-R). La norma no es en ningún caso la especificación
de un lenguaje gráfico de programación de robots, sino la especificación de cómo las funciones más importantes
de un lenguaje concreto deben ser representadas de manera gráfica.}}`` \\

\item \textbf{¿De qué se encarga el modelado del entorno?} (pp. 362 y 363)

La respuesta correcta es la b), que corresponde con el siguiente texto de las páginas 362 y 363 del libro: \\
``\textcolor{gray}{\textit{Normalmente, este modelo se limita a características geométricas: posición y
orientación de los objetos, y en ocasiones a su forma, dimensiones, peso, etc.
Para definir la posición y orientación de los objetos del modelo, lo más frecuente es asignar a cada objeto
de manera solidaria un sistema de referencia, de manera que la posición y orientación de este sistema referidos
a un sistema base, normalmente denominado sistema del mundo, definen de manera única las del objeto.
Algunos modelos del entorno permiten establecer relaciones entre objetos [FINKEL-74]
[RODRIGUEZ-94]. Éstas establecen la posible unión física entre los objetos. Dos objetos pueden ser independientes
(el movimiento de uno no afecta al otro), tener dependencia de unión
rígida (el movimiento de uno implica el del otro y viceversa) o tener una dependencia de
unión no rígida (el movimiento de uno implica el del otro, pero no al revés). Este modelo relacional,
que puede ser representado mediante una estructura arborescente, puede ser actualizado de manera automática durante la ejecución del programa 
(Figura 8.5), simplificando notablemente la tarea del programador. Alternativamente es el propio programa el que debe
mantener actualizadas las relaciones entre los objetos, haciendo uso de la reasignación de coordenadas a los objetos
y la expresión de éstas mediante la composición de transformaciones.}}`` \\

\item \textbf{Además de los tipos de datos convencionales, ¿cuáles son necesarios?} (p. 363)

La respuesta correcta es la a), que corresponde con el siguiente texto de la página 363 del libro: \\
``\textcolor{gray}{\textit{...además, de con los tipos de datos convencio-
nales (enteros, reales, booleanos, etc.) con otros específicamente destinados a definir las ope-
raciones de interacción con el entorno, como son, por ejemplo, los que especifican la posición
y orientación de los puntos y objetos a los que debe acceder el robot. Como ya se vio en el
Capítulo 3, correspondiente a herramientas matemáticas, la posición y orientación espacial de
un objeto puede ser especificada de diversas formas}}`` \\

\newpage
\item \textbf{¿Qué aplicaciones o usos importantes derivan del manejo de entradas y salidas?} (p. 364)

La respuesta correcta es la b), que corresponde con el siguiente texto de la página 364 del libro: \\
``\textcolor{gray}{\textit{Una utilización especial de las entradas binarias es la generación de interrupciones. 
En estos casos una determinada señal o una combinación lógica de las mismas, se monitoriza de
manera automática. Si la citada condición se verifica, la unidad de control del robot ejecuta
una rutina especial, interrumpiéndose el flujo normal del programa. Esta interrupción puede
realizarse inmediatamente (incluso en mitad del movimiento del robot), cuando finaliza la ejecución de la instrucción en curso
(al finalizar el movimiento), o cuando finaliza el ciclo de trabajo fijado por el programa (al finalizar la secuencia).
Otra aplicación importante de las entradas-salidas del robot, ya sean digitales o analógicas, es la integración de sensores,
incorporando la información de éstos al desarrollo de la tarea. Los sensores permiten ante todo realizar determinadas aplicaciones en un entorno 
parcialmente desconocido o con algun grado de incertidumbre como, por ejemplo, la localización
precisa de una pieza.}}`` \\

\item \textbf{¿Qué permite la comunicación entre robots o máquinas equivalentes que participan en la producción?} (p. 365)

La respuesta correcta es la d), que corresponde con los puntos expuestos en la página 365 del libro. \\

\item \textbf{En cuanto al control de movimientos, ¿qué elementos se deben incluir?} (p. 365)

La respuesta correcta es la a), que corresponde con el siguiente párrafo de la página 365 del libro: \\
``\textcolor{gray}{\textit{Indudablemente, un método de programación de robots debe incluir la posibilidad de espe-
cificar el movimiento del robot. Además del punto de destino, es necesario especificar el tipo
de trayectoria espacial que debe ser realizada, la velocidad media del recorrido y la precisión
con que se debe alcanzar el punto destino. Incluso en ocasiones puede ser necesario indicar si
el movimiento debe realizarse en cualquier caso o debe estar condicionado por algún tipo de
circunstancia como, por ejemplo, la medida proporcionada por un sensor.}}`` \\

\end{enumerate}

