\begin{table}[htb]
\centering
\begin{tabular}{|c|c|c|c|c|c|c|c|c|c|}
\hline
\rowcolor[HTML]{000000} 
{\color[HTML]{FFFFFF} 1} & {\color[HTML]{FFFFFF} 2} & {\color[HTML]{FFFFFF} 3} & {\color[HTML]{FFFFFF} 4} & {\color[HTML]{FFFFFF} 5} & {\color[HTML]{FFFFFF} 6} & {\color[HTML]{FFFFFF} 7} & {\color[HTML]{FFFFFF} 8} & {\color[HTML]{FFFFFF} 9} & {\color[HTML]{FFFFFF} 10} \\
\hline
\rowcolor[HTML]{FFFFFF} 
b & a & a & b & c & a & c & c & d & a \\
\hline
\rowcolor[HTML]{000000} 
{\color[HTML]{FFFFFF} 11} & {\color[HTML]{FFFFFF} 12} & {\color[HTML]{FFFFFF} 13} & {\color[HTML]{FFFFFF} 14} & {\color[HTML]{FFFFFF} 15} & {\color[HTML]{FFFFFF} 16} & {\color[HTML]{FFFFFF} 17} & {\color[HTML]{FFFFFF} } & {\color[HTML]{FFFFFF} } & {\color[HTML]{FFFFFF} } \\
\hline
\rowcolor[HTML]{FFFFFF} 
a & - & b & - & - & - & - &  &  &  \\
\hline
\end{tabular}
\caption{Tabla de respuestas de las cuestiones del tercer tema}
\label{table:tema1}
\end{table}


\begin{enumerate}
\item \textbf{¿Qué caracteriza la posición del extremo de un vector sobre un plano bidimensional, por ejemplo, $OX$ y $OY$?} (p. 66)

Corresponde con el siguiente párrafo de la página 66 del libro: \\
``\textcolor{gray}{\textit{Si se trabaja en un plano, con su sistema coordenado OXY de referencia asociado, un punto a vendrá expresado por las componentes (x, y) correspondientes a los ejes coordenados del sistema OXY. Este punto tiene asociado un vector p(x, y), que va desde el origen O del sistema OXY hasta el punto a (véase Figura 3.1a). Por tanto, \textbf{la posición del extremo del vector p está caracterizada por las dos componentes (x, y), denominadas coordenadas cartesianas del vector y que son las proyecciones del vector p sobre los ejes OX y OY.}}}`` \\

\item \textbf{¿Qué componentes tienen las coordenadas polares?} (p. 67)
\item \textbf{¿Qué componentes tienen las coordenadas cilíndricas?} (p. 67)

Corresponden con el siguiente párrafo de la página 67 del libro: \\
``\textcolor{gray}{\textit{\textbf{r representa la distancia desde el
origen O del sistema hasta el extremo del vector p, mientras que θ es el ángulo que forma el
vector p con el eje OX}. [...] En el caso de trabajar \textbf{en tres dimensiones}, un vector p podrá expresarse con respecto a un sistema de referencia OXYZ, mediante las coordenadas cilíndricas p(r, θ, z) (Figura 3.2b). Las componentes \textbf{r y θ tienen el mismo significado que en el caso de coordenadas polares}, aplicado el razonamiento sobre el plano OXY, \textbf{mientras que la componente z expresa la proyección sobre el eje OZ del vector p.}}}`` \\

\item \textbf{¿Qué componentes tienen las coordenadas esféricas?} (p. 67)

Corresponde con el siguiente párrafo de la página 67 del libro: \\
``\textcolor{gray}{\textit{coordenadas esféricas (r, θ, φ), donde la componente \textbf{r es la distancia desde el origen O
hasta el extremo del vector p; la componente θ es el ángulo formado por la proyección del
vector p sobre el plano OXY con el eje OX; y la componente φ es el ángulo formado por el
vector p con el eje OZ} (Figura 3.3).}}`` \\

\newpage
\item \textbf{En una matriz de transformación homogénea, ¿qué representa la última columna?} (pp. 78 y 86)

Corresponde con la matriz 3.22 de la página 78 y con la siguiente explicación de la página 86 del libro: \\
``\textcolor{gray}{\textit{n, o, a es una terna ortonormal que representa la orientación y p es un vector que representa la posición. [...] ...\textbf{este vector columna representa la posición del origen de O′UVW con respecto del sistema OXYZ}.}}`` \\

\item \textbf{Si aplicamos la matriz de transformación homogénea a un robot, ¿qué conseguimos?} (p. 87)

Corresponde el siguiente párrafo de la página 87 del libro: \\
``\textcolor{gray}{\textit{Si se aplica lo anteriormente indicado a un robot, \textbf{la matriz de transformación homogénea permite describir la localización (posición y orientación) de su extremo con respecto a su base}. Así, asociando a la base del robot un sistema de referencia fijo (OXYZ) y al extremo un sistema de referencia que se mueva con él, cuyo origen se encuentre en el punto p y los vectores directores son n o a escogidos de modos que (Figura 3.19)...}}`` \\

\item \textbf{¿Cuál es la principal ventaja de los cuaterniones frente a las matrices de transformación homogénea?} (p. 95)

Corresponde el siguiente párrafo de la página 95 del libro: \\
``\textcolor{gray}{\textit{Se observa que el empleo de cuaternios para la composición de rotaciones \textbf{es un método computacionalmente muy práctico, pues basta multiplicar cuaternios entre sí, lo que corresponde a una expresión de productos y sumas muy simple. Además permiten representar las rotaciones mediante solo 4 elementos, frente a los 9 utilizados en las matrices de rotación}. Ésta es su principal ventaja, tal y como se verá en la comparación de los distintos
métodos.}}`` \\


\item \textbf{¿Qué composición de matrices de transformación homogénea corresponde con los ángulos de euler XYZ-RPY?} (p. 98) 

En los ángulos de Euler XYZ los giros se realizan siempre sobre el sistema fijo,
primero sobre el eje $X$, luego sobre el $Y$ y después sobre el $Z$. Por tanto,
aplicando la composición de matrices de transformación homogénea obtenemos
que los giros en $Y$ y en $Z$ tienen que ir premultiplicados. La opción correcta
es la c), ecuación 3.74 de la página 98 del libro:
\[T_{XYZ} = RotZ(\phi) RotY(\theta) RotX(\psi)\]

La opción a) $T_{XYZ} = RotY(\theta) RotZ(\phi) RotX(\psi)$ representa:
\begin{itemize}
    \item un primer giro en X sobre el sistema fijo, y después un giro
     en Z y otro en Y sobre los sistemas fijos.
    \item un primer giro en Y sobre el sistema fijo, y después un giro
     en Z y otro en X sobre los sistemas móviles.
    \item un primer giro en Z sobre el sistema fijo, y después un giro
     en X sobre el sistema móvil y otro en Y sobre el sistema fijo.
\end{itemize} 

\newpage
La opción b) $T_{XYZ} = RotX(\psi) RotY(\theta) RotZ(\phi)$ representa:
\begin{itemize}
    \item un primer giro en Z sobre el sistema fijo, y después un giro
     en Y y otro en X sobre los sistemas fijos.
    \item un primer giro en X sobre el sistema fijo, y después un giro
     en Y y otro en Z sobre los sistemas móviles.
    \item un primer giro en Y sobre el sistema fijo, y después un giro
     en Z sobre el sistema móvil y otro en X sobre el sistema fijo.
\end{itemize} 

La opción d) $T_{XYZ} = RotX(\psi) RotZ(\phi) RotY(\theta)$ representa:
\begin{itemize}
    \item un primer giro en Y sobre el sistema fijo, y después un giro
     en Z y otro en X sobre los sistemas fijos.
    \item un primer giro en X sobre el sistema fijo, y después un giro
     en Z y otro en Y sobre los sistemas móviles.
    \item un primer giro en Z sobre el sistema fijo, y después un giro
     en Y sobre el sistema móvil y otro en X sobre el sistema fijo. \\
\end{itemize} 


\item \textbf{¿Qué composición de matrices de transformación homogénea corresponde con los ángulos de euler WUW (orden de ángulos: $\phi, \theta, \psi$)?} (página 97)

En los ángulos de Euler WUW los giros se realizan siempre sobre el sistema móvil,
primero sobre el eje $Z$, luego sobre el $X'$ y después sobre el $Z''$. Por tanto,
aplicando la composición de matrices de transformación homogénea obtenemos
que los giros en $X'$ y en $Z''$ tienen que ir postmultiplicados. La opción correcta
es la d), ecuación 3.70 de la página 97 del libro:
\[T_{WUW} = RotZ(\phi) RotX(\theta) RotZ(\psi)\]

La opción a) $T_{WUW} = RotZ(\phi) RotZ(\theta) RotX(\psi)$ representa un primer giro en Z sobre el sistema fijo, y después un giro
en Z y otro en X sobre los sistemas móviles.
    
La opción b) $T_{WUW} = RotZ(\psi) RotX(\theta) RotZ(\phi)$ representa un primer giro en Z sobre el sistema fijo, y después un giro
en X y otro en Z sobre los sistemas fijos.

La opción c) $T_{WUW} = RotZ(\phi) RotY(\theta) RotZ(\psi)$ representa un primer giro en Z sobre el sistema fijo, y después un giro
en Y y otro en Z sobre los sistemas móviles. \\


\item \textbf{¿Qué composición de matrices de transformación homogénea corresponde al siguiente cuaternión?}

Del cuaternión $Q = (\frac{\sqrt{2}}{2}, 0, \frac{\sqrt{2}}{2}, 0)$ sabemos:
\begin{itemize}
    \item $q_0 = \cos(\frac{\theta}{2}) = \frac{\sqrt{2}}{2}; \; \; \; \; \; q_i = k_i · \sin(\frac{\theta}{2}) = \frac{\sqrt{2}}{2}, \; i = 1, 2, 3$ 
    \item $\theta = 2 · \arctan(\frac{\sqrt{2}}{2}) = 90^\circ = \frac{\pi}{2}\;rad$\\
    \item $q_1 = 0; \; \; \; \; \; q_2 = k_2 · \sin(\frac{\theta}{2}) = \frac{\sqrt{2}}{2}; \; \; \; \; \; q_3 = 0$
    \item $k_1 = k_3 = 0; \; \; \; \; \; k_2 = \frac{q_2}{\sin(\frac{\frac{\pi}{2}}{2})} = \frac{\frac{\sqrt{2}}{2}}{\sin(\frac{\frac{\pi}{2}}{2})} = 1$ \\
\end{itemize}
Por tanto, estamos ante un giro de $90^\circ$ o $\frac{\pi}{2}$ radianes en el eje Y, cuya matriz de rotación es (ecuación 3.10, página 71 del libro):

\[\begin{bmatrix}\cos(\theta) & 0 & \sin(\theta) \\ 0 & 1 & 0 \\ -\sin(\theta) & 0 & \cos(\theta)\end{bmatrix} = \begin{bmatrix}\cos(\frac{\pi}{2}) & 0 & \sin(\frac{\pi}{2}) \\ 0 & 1 & 0 \\ -\sin(\frac{\pi}{2}) & 0 & \cos(\frac{\pi}{2})\end{bmatrix} = \begin{bmatrix}0 & 0 & 1 \\ 0 & 1 & 0 \\ -1 & 0 & 0\end{bmatrix}\]
\\

\item \textbf{¿Qué representación gráfica correponde a esta MTH?} 

De la matriz de transformación homogénea $\boldsymbol{T = \begin{bmatrix}1 & 0 & 0 & 3 \\ 0 & -1 & 0 & -4 \\ 0 & 0 & -1 & 2 \\ 0 & 0 & 0 & 1\end{bmatrix}}$
se obtiene que:
\begin{itemize}
    \item El vector $X'$ tiene componente en $X$ ya que $\vec{n'} = \vec{n} = \begin{pmatrix}1 & 0 & 0\end{pmatrix}$.
    \item El vector $Y'$ tiene componente negativa en $Y$ ya que $\vec{o'} = \vec{o} · (-1) = \begin{pmatrix}0 & -1 & 0\end{pmatrix}$.
    \item El vector $Z'$ tiene componente negativa en $Z$ ya que $\vec{a'} = \vec{a} · (-1) = \begin{pmatrix}0 & 0 & -1\end{pmatrix}$.
    \item El origen del sistema móvil se encuentra en las coordenadas $(x, y, z) = p = (3, -4, 2)$.
\end{itemize}
Se ha producido un giro en $X$ de $180^\circ$ o $\pi$ radianes y una traslación 
del vector $\vec{p} = (3, -4, 2)$, donde la submatriz de $3x3$,
$\boldsymbol{T = \begin{bmatrix}1 & 0 & 0 &\\ 0 & -1 & 0 \\ 0 & 0 & -1\end{bmatrix}}$,
corresponde con la matriz de rotación en el eje $X$:
\[\begin{bmatrix}1 & 0 & 0 \\ 0 & \cos(\theta) & -\sin(\theta) \\ 0 & \sin(\theta) & \cos(\theta)\end{bmatrix} = \begin{bmatrix}1 & 0 & 0 \\ 0 & \cos(\pi) & -\sin(\pi) \\ 0 & \sin(\pi) & \cos(\pi)\end{bmatrix} = \begin{bmatrix}1 & 0 & 0 \\ 0 & -1 & 0 \\ 0 & 0 & -1\end{bmatrix}\]

Por tanto, la única respuesta correcta es la a), ya que la b) y la d)
realizan una traslación diferente, y la c) no forma un sistema dextrógiro.

\begin{center}
\begin{tabular}{cccc}
\textbf{\textcolor{red}{a)}} \includegraphics[scale=0.75]{Images/tema3/11_correcta.png} &
\textbf{b)} \includegraphics[scale=0.75]{Images/tema3/11_mal1.png} \\
[1em]
\textbf{c)} \includegraphics[scale=0.75]{Images/tema3/11_mal2.png} &
\textbf{d)} \includegraphics[scale=0.75]{Images/tema3/11_mal3.png}
\end{tabular}
\end{center}

\newpage
\item \textbf{Calcular mediante MTH las transformaciones descritas en la siguiente imagen y
calcular también las rotaciones mediante cuaterniones.} 
\begin{center}
  \includegraphics[scale=0.8]{Images/tema3/12.png}
\end{center}
Hay diferentes maneras de solucionar el ejercicio. Por simplicidad,
esta resolución solo contemplará una posibilidad: rotación en $X$ de 
$-90^\circ$ o $-\pi$ radianes, rotación en $Z_{aux}$ (postmultiplicar) de $-90^\circ$ 
o $-\pi$ radianes y traslación del sistema $OX'Y'Z'$ un vector $\vec{p} = (-2, 6, 5)$
respecto del sistema original o fijo (premultiplicar).

\[T = Trasl(-2, 6, 5) \; RotX(\frac{-\pi}{2}) \; RotZ(\frac{-\pi}{2}) = \]
\[\begin{bmatrix}1 & 0 & 0 & -2\\ 0 & 1 & 0 & 6 \\ 0 & 0 & 1 & 5 \\ 0&0&0&1\end{bmatrix}
\begin{bmatrix}1 & 0 & 0 & 0\\ 0 & \cos(\frac{-\pi}{2}) & -\sin(\frac{-\pi}{2}) & 0 \\ 0 & \sin(\frac{-\pi}{2}) & \cos(\frac{-\pi}{2}) & 0 \\ 0&0&0&1\end{bmatrix} · 
\begin{bmatrix}\cos(\frac{\frac{-\pi}{2}}{2}) & -\sin(\frac{-\pi}{2}) & 0 & 0\\ \sin(\frac{-\pi}{2}) & \cos(\frac{-\pi}{2}) & 0 & 0 \\ 0 & 0 & 1 & 0 \\ 0&0&0&1\end{bmatrix} = \]
\[\begin{bmatrix}0 & 1 & 0 & -2\\ 0 & 0 & 1 & 6 \\ 1 & 0 & 0 & 5 \\ 0&0&0&1\end{bmatrix}\]

Se puede comprobar el resultado gráficamente:
\begin{itemize}
    \item El vector $\vec{n'} = (0, 0, 1)$ tiene componente en el eje $Z$, por lo que el eje $X'$
    y $Z$ tienen la misma dirección.
    \item El vector $\vec{o'} = (1, 0, 0)$ tiene componente en el eje $X$, por lo que el eje $Y'$
    y $X$ tienen la misma dirección.
    \item El vector $\vec{a'} = (0, 1, 0)$ tiene componente en el eje $Y$, por lo que el eje $Z'$
    y $Y$ tienen la misma dirección.
\end{itemize}

\newpage
Para calcular las rotaciones con cuaterniones, lo primero es construir
dos cuaterniones, $Q_1$ para $RotX(\frac{-\pi}{2})$ y $Q_2$
para $RotZ(\frac{-\pi}{2})$, a partir de la definición de cuaternio:
\[ Q_i = (q_{i0}, q_{i1}, q_{i2}, q_{i3}) = (\cos(\frac{\theta_i}{2}), \; k_{i1} · \sin(\frac{\theta_i}{2}), \; k_{i2} · \sin(\frac{\theta_i}{2}), \; k_{i3} · \sin(\frac{\theta_i}{2}))\]
\[\theta_1 = \theta_2 = \frac{-\pi}{2}, \; \qquad \frac{\theta_i}{2} = \frac{-\pi}{4}\]

El primer giro se realiza sobre el eje unitario $\vec{x} = (1, 0, 0)$, por lo que 
$k_{11} = 1$, $k_{12} = 0$, $k_{13} = 0$:
\[ Q_1 = (\cos(\frac{-\pi}{4}), \; \sin(\frac{-\pi}{4}), \; 0, \; 0)\]

El segundo giro se realiza sobre el eje unitario $\vec{z} = (0, 0, 1)$, por lo que 
$k_{21} = 0$, $k_{22} = 0$, $k_{23} = 1$:
\[ Q_2 = (\cos(\frac{-\pi}{4}), \; 0, \; 0, \; \sin(\frac{-\pi}{4}))\]

El segundo giro es sobre el sistema móvil, por lo que postmultiplicamos:
\[ Q_{12} = (q_{12_0}, q_{12_1}, q_{12_2}, q_{12_3}) = Q_1 \circ Q_2 = (\frac{1}{2}, \frac{-1}{2}, \frac{1}{2}, \frac{1}{2})\]
\[ q_{12_0} = q_{10} \cdot q_{20} - (q_{11} \cdot q_{21} + q_{12} \cdot q_{22} + q_{13} \cdot q_{23}) = \frac{1}{2}\]
\[ q_{12_1} = q_{10} \cdot q_{21} + q_{11} \cdot q_{20} + q_{12} \cdot q_{23} - q_{13} \cdot q_{22} = \frac{-1}{2}\]
\[ q_{12_2} = q_{10} \cdot q_{22} + q_{12} \cdot q_{20} + q_{13} \cdot q_{21} - q_{11} \cdot q_{23} = \frac{1}{2}\]
\[ q_{12_3} = q_{10} \cdot q_{23} + q_{13} \cdot q_{20} + q_{11} \cdot q_{22} - q_{12} \cdot q_{21} = \frac{-1}{2}\] \\

Si lo resolvemos en Matlab \cite{matlabonline2025} utilizando la \textit{toolbox} de robótica de Peter Corke \cite{corke_robotics_toolbox}:
\lstinputlisting[]{Matlab/ejercicio12.m}
\lstinputlisting[]{Matlab/salida12.txt}

Nota: al utilizar la \textit{toolbox} de robótica de Peter Corke con cuaterniones
se utilizan cuaterniones unitarios, es decir, cuya norma es 1:
$||Q|| = \sqrt{q_0² + q_1² + q_2² + q_3²} = 1$.
En los cuaterniones unitarios el signo no afecta a la rotación que representa.
Por tanto, el cuaternión unitario calculado por \textit{Matlab} es 
equivalente al calculado manualmente. \\

\item \textbf{¿Qué MTH corresponde a los ángulos de Euler WVW = $\boldsymbol{(45^\circ, 90^\circ, 120^\circ)}$
.} 

En los ángulos de Euler WYW los giros se realizan siempre sobre el sistema móvil,
primero sobre el eje $Z$, luego sobre el $Y'$ y después sobre el $Z''$. Por tanto,
aplicando la composición de matrices de transformación homogénea obtenemos
que los giros en $Y'$ y en $Z''$ tienen que ir postmultiplicados. La opción correcta
es la b), ecuación 3.72 de la página 98 del libro:
\[T_{WUW} = RotZ(\phi) RotY(\theta) RotZ(\psi) = RotZ(45^\circ) RotY(90^\circ) RotZ(120^\circ) = \]
\[\begin{bmatrix}\cos(45^\circ) & -\sin(45^\circ)\\ \sin(45^\circ) & \cos(45^\circ)\\ 0 & 0\end{bmatrix} 
\begin{bmatrix}\cos(90^\circ) & 0 & \sin(90^\circ) \\ 0 & 1 & 0\\ -\sin(90^\circ) & 0 & \cos(90^\circ)\end{bmatrix} 
\begin{bmatrix}\cos(120^\circ) & -\sin(120^\circ)\\ \sin(120^\circ) & \cos(120^\circ)\\ 0 & 0\end{bmatrix} = \]
\[
\begin{bmatrix}
C_{45} C_{120} C_{90} - S_{45} S_{120} &
- C_{120} S_{45} - C_{45} C_{90} S_{120} &
C_{45} S_{90} \\
C_{45} S_{120} + C_{120} C_{90} S_{45} &
C_{45} C_{120} - C_{90} S_{45} S_{120} &
S_{45} S_{90} \\
- C_{120} S_{90} &
S_{120} S_{90} &
C_{90}
\end{bmatrix}
\] \\
Si lo resolvemos en Matlab:
\lstinputlisting[]{Matlab/ejercicio13.m}
\lstinputlisting[]{Matlab/salida13.txt}
\vspace{0.3cm}

\item \textbf{¿Qué ángulos de Euler XYZ corresponden a la siguiente matriz de rotación?} 

\[T = \begin{bmatrix}0.3536 & -0.5732 & 0.7392\\ 0.6124 & 0.7392 & 0.2803\\ -0.7071 & 0.3536 & 0.6124\end{bmatrix}\] \\
En los ángulos de Euler XYZ los giros se realizan siempre sobre el sistema fijo,
primero sobre el eje $X$, luego sobre el $Y$ y después sobre el $Z$. Por tanto,
aplicando la composición de matrices de transformación homogénea obtenemos
que los giros en $Y$ y en $Z$ tienen que ir premultiplicados (ecuación 3.74 de la
página 98 del libro):
\[T_{XYZ} = RotZ(\phi) RotY(\theta) RotX(\psi) = \begin{bmatrix}
C_{\phi} C_{\theta} & C_{\phi} S_{\psi} S_{\theta} - C_{\psi} S_{\phi} & S_{\phi} S_{\psi} + C_{\phi} C_{\psi} S_{\theta} \\
C_{\theta} S_{\phi} & C_{\phi} C_{\psi} + S_{\phi} S_{\psi} S_{\theta} & C_{\psi} S_{\phi} S_{\theta} - C_{\phi} S_{\psi} \\
- S_{\theta} & C_{\theta} S_{\psi} & C_{\psi} C_{\theta}
\end{bmatrix}\] \\
Resolvemos el sistema de ecuaciones $T = T_{XYZ}$:
\begin{itemize}
    \item $\frac{T_{32}}{T_{33}} = \frac{C_{\theta} S_{\psi}}{C_{\psi} C_{\theta}} = \frac{0.3536}{0.6124}$.
    Anulamos $C_{\theta}$ y nos queda $\frac{S_{\psi}}{C_{\psi}} = \tan(\psi) = \frac{0.3536}{0.6124}$. \\
    Despejamos $\psi = \arctan(\frac{0.3536}{0.6124}) = 30^\circ = \frac{\pi}{6}$ radianes.

    \item $\frac{T_{21}}{T_{11}} = \frac{C_{\theta} S_{\phi}}{C_{\phi} C_{\theta}} = \frac{0.6124}{0.3536}$.
    Anulamos $C_{\theta}$ y nos queda $\frac{S_{\phi}}{C_{\phi}} = \tan(\phi) = \frac{0.6124}{0.3536}$. \\
    Despejamos $\phi = \arctan(\frac{0.6124}{0.3536}) = 60^\circ = \frac{\pi}{3}$ radianes.

    \item $T_{33} = C_{\psi} C_{\theta} = C_{30^\circ} C_{\theta}$. \\
    Despejamos $\theta = \arccos(\frac{0.6124}{C_{30^\circ}}) = 45^\circ = \frac{\pi}{4}$ radianes. \\
\end{itemize}
Resultado: $(X, Y, Z) = (30^\circ, 45^\circ, 60^\circ)$. \\

\newpage
Si lo resolvemos en Matlab:
\lstinputlisting[]{Matlab/ejercicio14.m}
\lstinputlisting[]{Matlab/salida14.txt}
\vspace{0.3cm}


\item \textbf{¿Qué matriz de rotación corresponde al siguiente cuaternión?} 

\[Q = (\frac{1}{2}, \frac{1}{2}, \frac{1}{2}, \frac{1}{2})\] \\
Aplicamos la relación descrita en la ecuación 3.86 de la página 102 del libro:

\[
T = 2 \cdot \begin{bmatrix}
q_0^2 + q_1^2 - \frac{1}{2} & q_1 q_2 - q_3 q_0 & q_1 q_3 + q_2 q_0 & 0 \\
q_1 q_2 + q_3 q_0 & q_0^2 + q_2^2 - \frac{1}{2} & q_2 q_3 - q_1 q_0 & 0 \\
q_1 q_3 - q_2 q_0 & q_2 q_3 + q_1 q_0 & q_0^2 + q_3^2 - \frac{1}{2} & 0 \\
0 & 0 & 0 & \frac{1}{2}
\end{bmatrix}
\]
\[T = 2 \cdot \begin{bmatrix}
\frac{1}{4} + \frac{1}{4} - \frac{1}{2} & \frac{1}{4} - \frac{1}{4} & \frac{1}{4} + \frac{1}{4} & 0 \\
\frac{1}{4} + \frac{1}{4} & \frac{1}{4} + \frac{1}{4} - \frac{1}{2} & \frac{1}{4} - \frac{1}{4} & 0 \\
\frac{1}{4} - \frac{1}{4} & \frac{1}{4} + \frac{1}{4} & \frac{1}{4} + \frac{1}{4} - \frac{1}{2} & 0 \\
0 & 0 & 0 & \frac{1}{2}
\end{bmatrix}
 = 
\begin{bmatrix}
0 & 0 & 1 & 0 \\
1 & 0 & 0 & 0 \\
0 & 1 & 0 & 0 \\
0 & 0 & 0 & 1
\end{bmatrix}
 \]\\
Si lo resolvemos en Matlab:
\lstinputlisting[]{Matlab/ejercicio15.m}
\lstinputlisting[]{Matlab/salida15.txt}
\vspace{0.3cm}

\item \textbf{¿Qué cuaternión corresponde a la siguiente matriz de rotación?} 

\[  T = \begin{bmatrix}
1 & 0 & 0 \\
0 & \frac{1}{\sqrt{2}} & \frac{-1}{\sqrt{2}}  \\
0 & \frac{1}{\sqrt{2}}  & \frac{1}{\sqrt{2}} 
\end{bmatrix}\] \\
Aplicamos la relación descrita en las ecuaciones 3.87 de la página 102 del libro:

\[ q_0 = \frac{1}{2} \cdot \sqrt{n_x + o_y + a_z +1} = \frac{1}{2} \cdot \sqrt{1 + \frac{1}{\sqrt{2}} + \frac{1}{\sqrt{2}} + 1} = 0.9239\]
\[ q_1 = signo(o_z - a_y) \cdot \frac{1}{2} \cdot \sqrt{n_x - o_y - a_z +1} = 1 \cdot \frac{1}{2} \cdot \sqrt{1 - \frac{1}{\sqrt{2}} - 0 + 1} = 0.3827\]
\[ q_2 = signo(a_x - n_z) \cdot \frac{1}{2} \cdot \sqrt{-n_x + o_y - a_z +1} = signo(0 - 0) \cdot ... = 0 \cdot ... = 0\]
\[ q_3 = signo(n_y - o_x) \cdot \frac{1}{2} \cdot \sqrt{-n_x - o_y + a_z +1} = signo(0 - 0) \cdot ... = 0 \cdot ... = 0\] \\
Resultado: $Q = (0.9239, 0.3827, 0, 0)$ \\

Si lo resolvemos en Matlab:
\lstinputlisting[]{Matlab/ejercicio16.m}
\lstinputlisting[]{Matlab/salida16.txt}
\vspace{0.3cm}


\item \textbf{¿Qué ángulos de Euler WYW corresponden al siguiente cuaternión?} 
\[Q = (cos(\frac{\pi}{4}), \frac{1}{\sqrt{3}} \cdot \sin(\frac{\pi}{4}),\frac{1}{\sqrt{3}} \cdot \sin(\frac{\pi}{4}), \frac{1}{\sqrt{3}} \cdot \sin(\frac{\pi}{4}))\] \\
Únicamente se resuelve con \textit{Matlab}. No obstante, si se quiere resolver a mano
se proponen dos pasos:
\begin{itemize}
    \item Calcular la matriz de rotación correspondiente al cuaternión.
    \item Calcular los ángulos de Euler WVW correspondientes a la matriz de rotación calculada. 
\end{itemize}
Nota: es importante observar que el vector de rotación es $(1, 1, 1)$, pero se trata de 
un cuaternión unitario donde $||v|| = 1$, por lo que lo que $k_i = \frac{1}{\sqrt{3}}$. \\ 

Resultado: $(W, Y, W) = (Z, Y', Z'') = (-15^\circ; 70.53^\circ; 75^\circ)$. \\

\lstinputlisting[]{Matlab/ejercicio17.m}
\lstinputlisting[]{Matlab/salida17.txt}
\vspace{0.3cm}



\end{enumerate}