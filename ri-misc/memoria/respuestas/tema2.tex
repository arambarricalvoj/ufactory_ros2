\begin{table}[htb]
\centering
\begin{tabular}{|c|c|c|c|c|c|c|c|c|}
\hline
\rowcolor[HTML]{000000} 
\multicolumn{1}{|c|}{\cellcolor[HTML]{000000}{\color[HTML]{FFFFFF} 1}} & \multicolumn{1}{c|}{\cellcolor[HTML]{000000}{\color[HTML]{FFFFFF} 2}} & \multicolumn{1}{c|}{\cellcolor[HTML]{000000}{\color[HTML]{FFFFFF} 3}} & \multicolumn{1}{c|}{\cellcolor[HTML]{000000}{\color[HTML]{FFFFFF} 4}} & \multicolumn{1}{c|}{\cellcolor[HTML]{000000}{\color[HTML]{FFFFFF} 5}} & \multicolumn{1}{c|}{\cellcolor[HTML]{000000}{\color[HTML]{FFFFFF} 6}} & \multicolumn{1}{c|}{\cellcolor[HTML]{000000}{\color[HTML]{FFFFFF} 7}} & \multicolumn{1}{c|}{\cellcolor[HTML]{000000}{\color[HTML]{FFFFFF} 8}} & \multicolumn{1}{c|}{\cellcolor[HTML]{000000}{\color[HTML]{FFFFFF} 9}}\\ \hline
\rowcolor[HTML]{FFFFFF} {\color[HTML]{000000} c}  & {\color[HTML]{000000} b} & {\color[HTML]{000000} b} & {\color[HTML]{000000} a} & {\color[HTML]{000000} c} & {\color[HTML]{000000} a, c, d} & {\color[HTML]{000000} d} & {\color[HTML]{000000} a} & {\color[HTML]{000000} a}\\ \hline
\end{tabular}
\caption{Tabla de respuestas de las cuestiones del segundo tema}\label{table:tema1}
\end{table}

\begin{enumerate}
\item \textbf{¿Qué elementos componen el robot?} (p. 31)

Corresponde con el primer párrafo de la página 31 del libro: \\
``\textcolor{gray}{\textit{Un robot está formado por los siguientes elementos: estructura mecánica, transmisiones, sistema de accionamiento, sistema sensorial, sistema de potencia y control, y elementos terminales.}}`` \\

\item \textbf{¿Qué constituye la estructura mecánica de un robot y qué es?} (p. 32)

Corresponde con el último párrafo de la página 32 del libro: \\
``\textcolor{gray}{\textit{Una cadena cinemática, es una serie de eslabones o barras unidas por articulaciones. La estructura mecánica de un robot manipulador constituye una cadena cinemática.}}``\\

Las respuestas a) y d) son incorrectas porque los términos ``cadena de articulaciones`` y ``cadena dinámica`` no existen en este campo de estudio.\\
La respuesta \textit{c)} es incorrecta porque las articulaciones también constituyen la estructura mecánica de un robot. De hecho, los eslabones están encadenados mediante articulaciones.\\

\item \textbf{¿Cuántos GDL se necesitan para posicionar?} (p. 35)
\item \textbf{¿Cuántos GDL se necesitan para orientar?} (p. 35)

Corresponden con el siguiente párrafo de la página 35 del libro: \\
``\textcolor{gray}{\textit{Puesto que para posicionar y orientar un cuerpo de cualquier manera en el espacio son necesarios seis parámetros, \textbf{tres para definir la posición y tres para la orientación}, si se pretende que un robot posicione y oriente su extremo (y con él la pieza o herramienta manipulada) de cualquier modo en el espacio, se precisarán al menos seis GDL.}}`` \\

\item \textbf{Un robot es redundante cuando…} (p. 36)

Corresponde con el siguiente párrafo de la página 36 del libro: \\
``\textcolor{gray}{\textit{Existen también casos opuestos, en los que se precisan más de seis GDL para que el robot pueda tener acceso a todos los puntos de su entorno. Así, si se trabaja en un entorno con obstáculos, el dotar al robot de grados de libertad adicionales le permite acceder a posiciones y orientaciones de su extremo a las que, como consecuencia de los obstáculos, no hubiera llegado con seis GDL. Otra situación frecuente es la de dotar al robot de un GDL adicional que le permita desplazarse a lo largo de un carril, aumentando así el volumen del espacio al que puede acceder. \textbf{Cuando el número de grados de libertad del robot es mayor que los necesarios para realizar una determinada tarea se dice que el robot es redundante.}}}`` \\

\item \textbf{Un robot redundante tendrá…} (p. 36, 3 respuestas) \\
Corresponde con la figura 2.7 de la página 36 del libro, donde se indica: \\
``\textcolor{gray}{\textit{Robot plano con 3 GDL para aumentar su maniobrabilidad}}`` \\
``\textcolor{gray}{\textit{Robot plano con 3 GDL para aumentar su volumen de trabajo}}`` \\

Es importante remarcar que, como se observa en dicha figura, un robot plano es un robot en dos dimensiones y por tanto, únicamente necesita 2 GDL. \\
La respuesta \textit{b)} es incorrecta porque la redundancia de un robot no está relacionada con la cantidad de actuadores. Se pueden incluir grados de libertad redundantes accionados por los mismos actuadores.\\

\item \textbf{Por lo general, ¿dónde se colocan los actuadores y por qué?} (p. 37) \\
Corresponde con el siguiente párrafo de la página 37 del libro: \\
``\textcolor{gray}{\textit{Dado que un robot mueve su extremo con aceleraciones elevadas, es de gran importancia reducir al máximo su momento de inercia. Del mismo modo, los pares estáticos que deben vencer los actuadores dependen directamente de la distancia de las masas al actuador. Por estos motivos se procura que \textbf{los actuadores, por lo general pesados, estén lo más cerca posible de la base del robot. Esta circunstancia obliga a utilizar sistemas de transmisión que trasladen el movimiento hasta las articulaciones}, especialmente a las situadas en el extremo del robot. Asimismo, las transmisiones pueden ser utilizadas para convertir movimiento circular en lineal o viceversa, lo que en ocasiones puede ser necesario.}}`` \\

La respuesta \textit{a)} hace referencia al accionamiento directo, que no suele ser habitual debido al peso de los motores y a la necesidad de proporcionar un par elevado a bajas revoluciones sin reductores.\\

\item \textbf{¿Qué elementos se utilizan para obtener información de presencia, posición y velocidad en robots industriales?} (p. 51) \\
Corresponde con el siguiente párrafo de la sección 2.4 (sensores internos) de la página 51 del libro: \\
``\textcolor{gray}{\textit{La información que la unidad de control del robot puede obtener sobre el estado de su estructura mecánica es, fundamentalmente, la relativa a su posición y velocidad. En la Tabla 2.5 se resumen los sensores más comúnmente empleados \textbf{para obtener información de presencia, posición y velocidad en robots industriales}...}}`` \\

La respuesta \textit{b)} es incorrecta ya que la información de presencia, posición y velocidad de robots industriales es una información interna referente a los elementos que forman el robot.\\
Las respuestas \textit{c)} y \textit{d)} son incorrectas porque los actuadores generar acciones y movimientos, pero no proporcionan información. \\

\item \textbf{¿Qué elemento o elementos interaccionan directamente con el entorno del robot?} (p. 58) \\
Corresponde con la sección 2.5 (elementos terminales) de la página 58 del libro: \\
``\textcolor{gray}{\textit{Los elementos terminales, también llamados efectores finales (end effector) son los encargados de \textbf{interaccionar directamente con el entorno del robot}.}}`` \\

\end{enumerate}