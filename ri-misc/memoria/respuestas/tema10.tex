\begin{table}[htb]
\centering
\begin{tabular}{|c|c|c|c|c|c|c|c|c|c|}
\hline
\rowcolor[HTML]{000000} 
{\color[HTML]{FFFFFF} 1} & {\color[HTML]{FFFFFF} 2} & {\color[HTML]{FFFFFF} 3} & {\color[HTML]{FFFFFF} 4} & {\color[HTML]{FFFFFF} 5} & {\color[HTML]{FFFFFF} 6} & {\color[HTML]{FFFFFF} 7} & {\color[HTML]{FFFFFF} 8} & {\color[HTML]{FFFFFF} 9} & {\color[HTML]{FFFFFF} 10} \\
\hline
\rowcolor[HTML]{FFFFFF} 
a & b, c, d & c & d & a & d & b & c & c & a \\
\hline
\rowcolor[HTML]{000000} 
{\color[HTML]{FFFFFF} 11} & {\color[HTML]{FFFFFF} 12} & {\color[HTML]{FFFFFF} 13} & {\color[HTML]{FFFFFF} 14} & {\color[HTML]{FFFFFF}} & {\color[HTML]{FFFFFF} } & {\color[HTML]{FFFFFF} } & {\color[HTML]{FFFFFF} } & {\color[HTML]{FFFFFF} } & {\color[HTML]{FFFFFF} } \\
\hline
\rowcolor[HTML]{FFFFFF} 
b & d & a, c & c &  &  &  &  &  &  \\
\hline
\end{tabular}
\caption{Tabla de respuestas de las cuestiones del décimo tema}
\label{table:tema10}
\end{table}


\begin{enumerate}
\item \textbf{¿Qué publica anualmente la Federación Internacional de la Robótica (IFR)?} (p. 440)

La respuesta correcta es la a), que se ha extraído de la página 440 del libro. \\

\item \textbf{¿En qué trabajos es común implantar robots industriales? } (p. 441) (3 respuestas)

Las respuestas correctas son la b), c) y d), que corresponden con las secciones del apartado 10.2 de la página 441 del libro. \\

\item \textbf{¿Cuál fue el primer proceso industrial robotizado y en qué año?} (p. 441)

La respuesta correcta es la c), que se ha extraído de la página 441 del libro. \\

\item \textbf{¿Qué industria emplea la mayor parte de los robots industriales instalados?} (p. 442)

La respuesta correcta es la d), que se ha extraído de la página 442 del libro: \\
``\textcolor{gray}{\textit{La industria automovilística ha sido la gran impulsora de la robótica industrial, empleando la
mayor parte de los robots hoy en día instalados.}}`` \\

\item \textbf{¿Los procesos de pintura son interesantes desde un punto de vista de la robotización?} (p. 446)

La respuesta correcta es la a), que se ha extraído de la página 446 del libro: \\
``\textcolor{gray}{\textit{... el entorno en el que se realiza la pintura es sumamente desagradable y 
peligroso. En él se tiene simultáneamente un reducido espacio, una atmósfera tóxica, un alto nivel de ruido,
un riesgo de incendio, etc. Estas circunstancias han hecho de la pintura y operaciones afines, 
un proceso de interesante robotización. Con el empleo del robot se eliminan
los inconvenientes ambientales y se gana en cuanto a homogeneidad en la calidad del acabado, ahorro de pintura y productividad.}}`` \\

\item \textbf{¿Es la robotización una solución ventajosa para la alimentación de máquinas?} (p. 448)

La respuesta correcta es la d), que se ha extraído de la página 448 del libro: \\
``\textcolor{gray}{\textit{En la industria metalúrgica se usan prensas para conformar los metales en frío o, 
para mediante estampación y embutido, obtener piezas de complicadas formas a partir de planchas de
metal. En ocasiones la misma pieza pasa consecutivamente por varias prensas de estampación
hasta conseguir su forma definitiva. La carga y descarga de estas máquinas se realiza tradicionalmente a mano,
con el elevado riesgo que esto conlleva para el operario, al que una pequeña distracción puede costarle un serio accidente.
Estas circunstancias, junto con la superior precisión de posicionamiento que puede conseguir el robot,
y la capacidad de éste de controlar automáticamente el funcionamiento de la
máquina y dispositivos auxiliares, han hecho que el robot sea una solución ventajosa para estos procesos (Figura 10.8).
Por otra parte, los robots}}`` \\

\item \textbf{¿Qué requieren los robots de ensamblaje y qué tipos son los más adecuados?} (p. 454)

La respuesta correcta es la b), que se ha extraído de la página 454 del libro: \\
``\textcolor{gray}{\textit{Los robots empleados en el ensamblaje requieren, en cualquier caso, una gran precisión 
y repetitividad, no siendo preciso que manejen grandes cargas. El tipo SCARA ha alcanzado gran popularidad en este
tipo de tareas por su bajo coste y buenas características.
Éstas se consiguen por su adaptabilidad selectiva, presentando facilidad para desviarse,
por una fuerza externa, en el plano horizontal y una gran rigidez para hacerlo en el eje
vertical. También son adecuados los robots cartesianos por su elevada precisión y, en general,
los robots articulares que pueden resolver muchas de estas aplicaciones con suficiente efectividad.}}`` \\

\item \textbf{Generalmente, ¿qué implican los procesos de paletización?} (p. 454)

La respuesta correcta es la c), que se ha extraído de la página 454 del libro: \\
``\textcolor{gray}{\textit{Generalmente, las tareas de paletización implican el manejo de grandes cargas, de peso y
dimensiones elevadas. Por este motivo, los robots empleados en este tipo de aplicaciones
acostumbran a ser robots de gran tamaño, con una capacidad de carga de 10 a 100 kg.
Dado que el paletizado se realiza en un espacio de la tarea de 4 grados de libertad (X,
Y, Z, q) en el que los productos son recogidos y depositados siempre en la dirección vertical,
cabe asumir que para el paletizado sea suficiente con utilizar robots de 4 gdl.}}`` \\

\item \textbf{¿Qué tipo de robot es más adecuado para las tareas de control de calidad?} (p. 457)

La respuesta correcta es la c), que se ha extraído de la página 457 del libro: \\
``\textcolor{gray}{\textit{No existe, en este caso, un tipo concreto de robot más adecuado para estas tareas. En el
control dimensional suelen usarse robots cartesianos por la precisión de éstos pero, en general,
son igualmente válidos robots articulares.}}`` \\

\item \textbf{¿Qué caracteriza a la robótica de servicios?} (p. 460)

La respuesta correcta es la a), que se ha extraído del apartado 10.3 de la página 460 del libro. \\

\item \textbf{¿Qué es un robot de servicio según la Federación Internacional de Robótica \textit{(IFR)}?} (p. 460)

La respuesta correcta es la b), que corresponde con la definición expuesta en la paǵina 460 del libro. \\

\newpage
\item \textbf{¿Qué tipos de robots son más comunes en las industrias nuclear y médica?} (p. 467)

La respuesta correcta es la d), que se ha extraído del apartado 10.3.4 de la página 467 del libro. \\

\item \textbf{¿Qué aplicaciones son más aptas para robot aéreos y submarinos?} (p. 470) (2 respuestas)

Las respuestas correcta son la a) y c), que se han extraído de la página 470 del libro: \\
``\textcolor{gray}{\textit{Las aplicaciones de los robots aéreos y submarinos son amplias. El mantenimiento de
infraestructuras de larga longitud y de difícil acceso, como gaseoductos, líneas de energía eléctrica o
líneas de comunicación submarina, que pueden implicar instalaciones de varios cientos de kilómetros,
precisa de una periódica revisión de todo su recorrido. Estas revisiones precisan para su
realización medios económicamente costosos y que implican ciertos riesgos, como helicópteros para el caso de
instalaciones terrestres, o de minisubmarinos unidos a barcos mediante un cable
para el caso de instalaciones submarinas. La utilización de vehículos aéreos autónomos (UAV - Unmanned Aerial Vehicles)
(Figura 10.24) para el primer caso o de submarinos autónomos (AUV - Autonomous Underwater Vehicle) (Figura 10.25)
para el segundo, permite disminuir los riesgos y los costes y con ello aumentar la periodicidad de la inspección.
El uso de estos sistemas puede ser igualmente útil en la inspección de obras civiles, como presas, acueductos o diques.}}`` \\

\item \textbf{¿Cuál es el principal producto de robótica asistencial?} (p. 472) 

La respuesta correcta es la c), que se ha extraído del apartado 10.3.6 de la página 472 del libro. \\

\end{enumerate}

