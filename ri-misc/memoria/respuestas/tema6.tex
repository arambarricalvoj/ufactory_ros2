\begin{table}[htb]
\centering
\begin{tabular}{|c|c|c|c|c|c|c|c|c|c|}
\hline
\rowcolor[HTML]{000000} 
{\color[HTML]{FFFFFF} 1} & {\color[HTML]{FFFFFF} 2} & {\color[HTML]{FFFFFF} 3} & {\color[HTML]{FFFFFF} 4} & {\color[HTML]{FFFFFF} 5} & {\color[HTML]{FFFFFF} 6} & {\color[HTML]{FFFFFF} 7} & {\color[HTML]{FFFFFF} 8} & {\color[HTML]{FFFFFF} 9} & {\color[HTML]{FFFFFF} 10} \\
\hline
\rowcolor[HTML]{FFFFFF} 
c & b & a & a & c & b & d & d & b & a \\
\hline
\rowcolor[HTML]{000000} 
{\color[HTML]{FFFFFF} 11} & {\color[HTML]{FFFFFF} 12} & {\color[HTML]{FFFFFF} 13} & {\color[HTML]{FFFFFF} 14} & {\color[HTML]{FFFFFF} } & {\color[HTML]{FFFFFF} } & {\color[HTML]{FFFFFF} } & {\color[HTML]{FFFFFF} } & {\color[HTML]{FFFFFF} } & {\color[HTML]{FFFFFF} } \\
\hline
\rowcolor[HTML]{FFFFFF} 
d & b & c & - &  &  &  &  &  &  \\
\hline
\end{tabular}
\caption{Tabla de respuestas de las cuestiones del sexto tema}
\label{table:tema6}
\end{table}


\begin{enumerate}
\item \textbf{¿Qué se debe conocer para abordar el control cinemático de un robot?} (p. 281)

La respuesta correcta es la c), que corresponde con el siguiente texto de la página 281 del libro \cite{barrientos2007fundamentos}: \\
``\textcolor{gray}{\textit{para abordar el control cinemático de un robot se deberá, fundamentalmente,
conocer qué tipo de trayectorias se puede pretender que realice el robot [...] así
como qué tipo de interpoladores pueden ser eficaces para unir los puntos articulares por los
que se quiere pasar}}`` \\

\item \textbf{¿Qué problema presenta el movimiento simultáneo de ejes no coordinado en las trayectorias punto a punto?} (p. 282)

La respuesta correcta es la b), que corresponde con el siguiente texto de la página 282 del libro: \\
``\textcolor{gray}{\textit{El movimiento del robot no acabará hasta que se alcance definitivamente el punto final, lo
que se producirá cuando concluya su movimiento el eje que más tarde. De esta manera, el
tiempo total invertido en el movimiento coincidirá con el del eje que más tiempo emplee en
realizar su movimiento particular, pudiéndose dar la circunstancia de que el resto de los actuadores hayan forzado su
movimiento a una velocidad y aceleración elevada, viéndose
obligados finalmente a esperar a la articulación más lenta}}`` \\

\item \textbf{¿Qué caracteriza a las trayectorias punto a punto con movimiento simultáneo y coordinado de ejes?} (p. 283)

La respuesta correcta es la a), que corresponde con el siguiente texto de la página 283 del libro: \\
``\textcolor{gray}{\textit{un cálculo previo, averiguando cuál es esta articulación y qué tiempo invertirá. Se ralentizará
entonces el movimiento del resto de los ejes para que inviertan el mismo tiempo en
su movimiento, acabando todos ellos simultáneamente. Se tiene así que todas las articulaciones se coordinan comenzando y
acabando su movimiento a la vez, adaptándose todas a la más lenta [...].El tiempo total invertido en el movimiento es el
menor posible y no se demandan aceleraciones y velocidades elevadas a los actuadores de manera inútil.}}`` \\

\item \textbf{¿En qué se diferencian las trayectorias punto a punto y continuas?} (pp. 282 y 283)

La respuesta correcta es la a), que corresponde con las definciones de ambas trayectorias en las páginas 282 y 283 del libro:  \\

Trayectoria punto a punto: \\
``\textcolor{gray}{\textit{En este tipo de trayectorias cada articulación evoluciona desde su posición inicial a la final sin
realizar consideración alguna sobre el estado o evolución de las demás articulaciones. Normalmente, cada actuador trata de llevar
a su articulación al punto de destino en el menor tiempo posible, pudiéndose distinguir dos casos: 
movimiento eje a eje y movimiento simultáneode ejes. A su vez, dentro del segundo tipo, los ejes pueden moverse de manera independiente o coordinada}}`` \\

Trayectoria continua: \\
``\textcolor{gray}{\textit{Cuando se pretende que la trayectoria que sigue el extremo del robot sea conocida por el
usuario (trayectoria en el espacio cartesiano o de la tarea), es preciso calcular y controlar de
manera continua las trayectorias articulares, que, por lo general, presentarán una compleja
evolución temporal.
Típicamente, las trayectorias que el usuario pretende que el robot describa son trayectorias
en línea recta o en arco de círculo. Para conseguirlo habrá que seguir los pasos CC-1 a CC-5
indicados en el Epígrafe 6.1. El resultado será que cada articulación sigue un movimiento
aparentemente caótico con posibles cambios de dirección y velocidad y sin coordinación con
el resto de las articulaciones. Sin embargo, el resultado conjunto será que el extremo del robot describirá la trayectoria deseada (Figura 6.3d).}}`` \\

\item \textbf{¿Qué son los interpoladores y cuál es su función?} (pp. 287 y 288):

La respuesta correcta es la c), que corresponde con el siguiente texto de las páginas 287 y 288 del libro: \\
``\textcolor{gray}{\textit{unir esta sucesión de puntos articulares garantizando, junto con las condiciones de configuración-tiempo de
paso, que se cumplan las restricciones de velocidad y aceleración máximas asociadas a los
actuadores del robot, de manera que se consiga una trayectoria realizable y suficientemente suave.
Para ello deberá seleccionarse algún tipo de función (frecuentemente polinómica) cuyos
parámetros o coeficientes se ajustarán al imponer las condiciones de contorno: posiciones, velocidades y aceleraciones.}}`` \\

\item \textbf{¿Qué son las condiciones de contorno y cuántas son necesarias?} (pp. 287 y 288)

La respuesta correcta es la b), que corresponde con el siguiente texto de la página 288 del libro: \\
``\textcolor{gray}{\textit{si se desea pasar por n puntos, con la condición de que se parta y se llegue
al reposo (velocidad nula), podría considerarse como primera opción el utilizar un polinomio de
grado n + 1 cuyos n + 2 coeficientes se calcularían para garantizar las n+2 condiciones de con-
torno (n puntos de paso y 2 velocidades).}}`` \\

\newpage
\item \textbf{Si tenemos tres condiciones de contorno, ¿cuál es la interpolación más sencilla?} (pp. 287 y 288)

La respuesta correcta es la d), ya que la parabólica (polinomio de grado dos) solo necesita tres condiciones de contorno. \\
Recordar que la interpolación trapezoidal necesita seis parámetros, no tres. \\

\item \textbf{¿Por qué no es recomendable emplear la interpolación lineal entre 2 localizaciones sucesivas?} (p. 285)

La respuesta correcta es la d), que corresponde con el siguiente texto de la página 285 del libro: \\
``\textcolor{gray}{\textit{Este tipo de evolución origina evidentes discontinuidades en la velocidad y aceleración en
los puntos de paso, por lo que su uso no es recomendable.}}`` \\

\item \textbf{¿Con qué herramienta se representa la evolución más natural e intuitiva de la orientación?} (p. 286)

La respuesta correcta es la b), que corresponde con el siguiente texto de la página 286 del libro: \\
``\textcolor{gray}{\textit{La evolución más natural desde una orientación inicial hasta otra final, sería aquella que
hace girar de manera progresiva al efector final (u objeto manipulado por el robot) desde su
orientación inicial hasta la final, en torno a un eje de giro fijo. Por este motivo, la utilización
del par de rotación, o su equivalente, los cuaternios, es la manera más adecuada para generar
la trayectoria cartesiana de orientación. [...] para que el extremo del robot evolucione desde la orientación inicial hasta la final,
se podrá buscar cuál es el par (k, θ) que relaciona los sistemas de coordenadas ortonormales asociados a ambas orientaciones, y realizar
la evolución temporal mediante un giro en torno al eje k.}}`` \\

\item \textbf{¿Qué estrategia se utiliza para evitar solucionar funciones interpoladoras con un 
elevado grado polinómico?} (p. 288)

La respuesta correcta es la a), que corresponde con el siguiente texto de la página 288 del libro: \\
``\textcolor{gray}{\textit{si se desea pasar por n puntos, con la condición de que se parta y se llegue
al reposo (velocidad nula), podría considerarse como primera opción el utilizar un polinomio de
grado n + 1 cuyos n + 2 coeficientes se calcularían para garantizar las n+2 condiciones de contorno
(n puntos de paso y 2 velocidades). Sin embargo, el uso de un polinomio de grado elevado
presenta desde el punto de vista computacional múltiples inconvenientes. Precisa por ejemplo
de resolver un sistema de ecuaciones con n + 2 incógnitas. Asimismo, la presencia de potencias
elevadas (t n+1), acarreará errores que pueden ser significativos en los cálculos.
Por estos motivos, en lugar de usar una sola función interpoladora que una todos los puntos de la trayectoria,
se utilizan conjuntos de funciones más simples, que interpolan localmente la secuencia de puntos,
esto es, unen sólo unos pocos puntos consecutivos de un intervalo, solapándose unas con otras para garantizar la continuidad.}}`` \\

\newpage
\item \textbf{Se desea rotar un objeto desde la orientación inicial $R_i$ hasta la orientación final $R_f$ 
en el intervalo de tiempo $[t_i = 0, t_f = 1]$, mediante un giro en torno a un eje fijo $\mathbf{k}$ 
con ángulo total $\theta$. El cuaternión correspondiente al giro es $Q = (\cos(\frac{\pi}{4}),\; 0,\; 0,\; \sin(\frac{\pi}{4}))$.
Calcular la orientación en el instante $t=0.5$ y $t=0.7$.} (p. 286)

La respuesta correcta es la d). \\
Del cuaternión se sabe que el ángulo total de giro es:
\[\frac{\theta}{2} = \frac{\pi}{4} \rightarrow \theta = \frac{\pi}{2}\]

La solución se obtiene directamente de la aplicación de la fórmula 6.3 en la página 286 del libro: \\
\[\theta (t) = \theta \cdot \frac{t - t_i}{t_f - t_i} \] \\
\[\theta (0.5) = \frac{\pi}{2} \cdot \frac{0.5 - 0}{1 - 0} = \frac{\pi}{4}\] \\
\[\theta (0.7) = \frac{\pi}{2} \cdot \frac{0.7 - 0}{1 - 0} = \frac{7 \cdot \pi}{20}\] \\

Si resolvemos en \textit{Matlab} \cite{matlabonline2025} utilizando la \textit{toolbox} \cite{corke_robotics_toolbox}:
\lstinputlisting[]{Matlab/tema6/ejer9.m}
\lstinputlisting[]{Matlab/tema6/ejer9_salida.txt}


\item \textbf{Se desea rotar un objeto desde la orientación inicial $R_i = I$ hasta la orientación final $R_f$ 
en el intervalo de tiempo $[t_i = 0, t_f = 1]$, mediante un giro representado por los ángulos de Euler XYZ
$(\frac{\pi}{3},\; \frac{\pi}{5},\; \frac{\pi}{4})$. Calcular la orientación XYZ en el instante $t=0.2$ y $t=0.6$.} (p. 286)

La respuesta correcta es la b). La solución se obtiene directamente de la aplicación de las fórmulas 6.2 en la página 286 del libro: \\
\[\phi (t) = (\phi_f - \phi_i) \cdot \frac{t - t_i}{t_f - t_i} + \phi_i \]
\[\theta (t) = (\theta_f - \theta_i) \cdot \frac{t - t_i}{t_f - t_i} + \theta_i \]
\[\psi (t) = (\psi_f - \psi_i) \cdot \frac{t - t_i}{t_f - t_i} + \psi_i \] \\

\[\phi (0.2) = (\frac{\pi}{3} - 0) \cdot \frac{0.2 - 0}{1 - 0} + 0 = \frac{\pi}{15}\]
\[\theta (0.2) = (\frac{\pi}{5} - 0) \cdot \frac{0.2 - 0}{1 - 0} + 0 = \frac{\pi}{25}\]
\[\psi (0.2) = (\frac{\pi}{4} - 0) \cdot \frac{0.2 - 0}{1 - 0} + 0 = \frac{\pi}{20}\] 

\[XYZ(0.2) = (\frac{\pi}{15},\; \frac{\pi}{25},\; \frac{\pi}{20})\] \\

\[\phi (0.6) = (\frac{\pi}{3} - 0) \cdot \frac{0.6 - 0}{1 - 0} + 0 = \frac{\pi}{5}\]
\[\theta (0.6) = (\frac{\pi}{5} - 0) \cdot \frac{0.6 - 0}{1 - 0} + 0 = \frac{3 \cdot \pi}{25}\]
\[\psi (0.6) = (\frac{\pi}{4} - 0) \cdot \frac{0.6 - 0}{1 - 0} + 0 = \frac{3 \cdot \pi}{20}\]

\[XYZ(0.6) = (\frac{\pi}{5},\; \frac{3\cdot \pi}{25},\; \frac{3 \cdot \pi}{20}) \] \\

\item \textbf{Obtener el valor de $q(0.5)$ mediante la interpolación lineal de un robot con los siguientes datos:} (p. 285)
\[
q(0)=0.2\ \text{rad}, \quad q(1)=0.6\ \text{rad}, \quad \dot{q}(0)=0.1\ \text{rad/s}, \quad \dot{q}(1)=0.3\ \text{rad/s}.
\]

La solución se obtiene directamente de la aplicación de la fórmula 6.1 en la página 285: \\
\[
q(t) = q_i + \frac{t - t_i}{t_f - t_i}\,\big(q_f - q_i\big)
\]

\[
q(0.5) = 0.2 + \frac{0.5 - 0}{1 - 0}\,\big(0.6 - 0.2\big) = 0.4
\]

\item \textbf{Dado el robot de la imagen con sus datos y velocidades nulas en los extremos, articulaciones rotacionales $q_1$ y $q_3$ y articulaciones
prismáticas $q_2$ y $q_4$, responder a la siguientes preguntas:}
\begin{center}
  %\includegraphics[scale=0.2, angle=90]{Images/tema4/dh/9.jpg}
  \includegraphics[scale=0.51, angle=0]{Images/tema6/robot.png}
\end{center}
\[q_1(0) = 90^\circ,\; q_1(1) = 0^\circ,\; q_2(0) = 1,\; q_2(1) = 4,\; q_3(0) = 0^\circ,\; q_3(1) = 45^\circ,\; q_4(0) = 4,\; q_4(1) = 2\]

\[t_0 = 0, \quad t_1 = 1\]

a) Obtener el interpolador splin cúbico para cada eslabón.

Para cada articulación $q_j(t)$ del robot (rotacional o prismática), se define un polinomio cúbico por tramo entre dos instantes $t^i$ y $t^{i+1}$ (fórmula 6.5 de la página 290 del libro):

\[
q_j(t) = a^i_j + b^i_j\,(t - t^i) + c^i_j\,(t - t^i)^2 + d^i_j\,(t - t^i)^3, \quad t^i < t < t^{i+1}
\]


Los coeficientes se calculan como (fórmula 6.5 de la página 290 del libro):

\[
a^i_j = q_j^i, \qquad b^i_j = \dot{q}_j^i, \qquad T^i = t^{i+1} - t^i
\]

\[
c^i_j = \frac{3}{(T^i)^2}(q_j^{i+1} - q_j^i) - \frac{1}{T^i}(\dot{q}_j^{i+1} + 2\dot{q}_j^i)
\]

\[
d^i_j = -\frac{2}{(T^i)^3}(q_j^{i+1} - q_j^i) + \frac{1}{(T^i)^2}(\dot{q}_j^{i+1} + \dot{q}_j^i)
\] \\


Considerando las condiciones de contorno, se obtienen las siguientes interpolaciones por cada eslabón:

\[
  q_1(t) = 180t^3 - 270t^2 + 90, \qquad   q_2(t) = -6t^3 + 9t^2 + 1
  \]
  
\[
  q_3(t) = -90t^3 + 135t^2, \qquad q_4(t) = 4t^3 - 6t^2 + 4
  \] \\

b) Obtener el interpolador splin cúbico para cada eslabón con la siguiente variación de datos:
  \[ \dot{q}_1(1) = \dot{q}_3(1) = 5 \; grados/s, \quad \dot{q}_2(1) = \dot{q}_4(1) = 0.2 m/s\]
  \[ q_j(1) = \frac{q_j(i) + q_j(f)}{2}\]
  \[ t_i = 0, \quad t_f = 2, \quad t_{intermedio} = 1\] \\

Se definen los nodos: $t^{i-1}=0$,\; $t^{i}=1$,\; $t^{i+1}=2$.
\[
\begin{aligned}
&q_1(0)=90^\circ,\quad q_1(1)=45^\circ,\quad q_1(2)=0^\circ\\
&q_2(0)=1,\quad q_2(1)=2,\quad q_2(2)=4\\
&q_3(0)=0^\circ,\quad q_3(1)=22.5^\circ,\quad q_3(2)=45^\circ\\
&q_4(0)=4,\quad q_4(1)=3,\quad q_4(2)=2
\end{aligned}
\] \\

Para cada articulación $q_j$ en el instante $i$, se puede estimar la velocidad de paso como (criterio de Craig, fórmula 6.6 de la página 290 del libro):
\[
\dot{q}_j^i =
\begin{cases}
0, & \text{si } \operatorname{signo}(q_j^i - q_j^{i-1}) \neq \operatorname{signo}(q_j^{i+1} - q_j^i) \

\\[6pt]
\dfrac{1}{2}\left[\dfrac{q_j^{i+1} - q_j^i}{t^{i+1} - t^i} + \dfrac{q_j^i - q_j^{i-1}}{t^i - t^{i-1}}\right], &
\text{si } \operatorname{signo}(q_j^i - q_j^{i-1}) = \operatorname{signo}(q_j^{i+1} - q_j^i)
\end{cases}
\]

\[
\dot q_1(1)=\tfrac{1}{2}\left[\tfrac{0-45}{2-1}+\tfrac{45-90}{1-0}\right]
=\tfrac{1}{2}(-45-45)=-45^\circ/\mathrm{s}\]

\[
\dot q_2(1)=\tfrac{1}{2}\left[\tfrac{4-2}{1}+\tfrac{2-1}{1}\right]
=\tfrac{1}{2}(2+1)=1.5~\mathrm{m/s}\]

\[
\dot q_3(1)=\tfrac{1}{2}\left[\tfrac{45-22.5}{1}+\tfrac{22.5-0}{1}\right]
=\tfrac{1}{2}(22.5+22.5)=22.5^\circ/\mathrm{s}\]

\[
\dot q_4(1)=\tfrac{1}{2}\left[\tfrac{2-3}{1}+\tfrac{3-4}{1}\right]
=\tfrac{1}{2}(-1-1)= -1~\mathrm{m/s}
\] \\

\newpage
Considerando las condiciones de contorno, tras calcular los coeficientes se obtienen las siguientes interpolaciones por cada eslabón en cada tramo:

\[
\boxed{q_1(t)=90 - 90\,t^2 + 45\,t^3,\quad t\in[0,1]}
\]

\[
\boxed{q_1(t)=45 - 45\,\tau - 45\,\tau^2 + 45\,\tau^3,\quad \tau=t-1,\ t\in[1,2]}
\] \\


\[
\boxed{q_2(t)=1 + 1.5\,t^2 - 0.5\,t^3,\quad t\in[0,1]}
\]

\[
\boxed{q_2(t)=2 + 1.5\,\tau + 3\,\tau^2 - 2.5\,\tau^3,\quad \tau=t-1,\ t\in[1,2]}
\] \\


\[
\boxed{q_3(t)=45\,t^2 - 22.5\,t^3,\quad t\in[0,1]}
\]

\[
\boxed{q_3(t)=22.5 + 22.5\,\tau + 22.5\,\tau^2 - 22.5\,\tau^3,\quad \tau=t-1,\ t\in[1,2]}
\] \\

\[
\boxed{q_4(t)=4 - 2\,t^2 + 1\,t^3,\quad t\in[0,1]}
\]

\[
\boxed{q_4(t)=3 - \tau - \tau^2 + \tau^3,\quad \tau=t-1,\ t\in[1,2]}
\] \\

Si revolvemos en \textit{Matlab} programando el método manualmente podemos obtener las gráficas de las interpolaciones (figura \ref{fig:cuatroimagenes}):
\lstinputlisting[]{Matlab/tema6/spline3.m}

\begin{figure}[htbp]
    \centering
    % fila superior
    \begin{minipage}{0.45\textwidth}
        \centering
        \includegraphics[width=\linewidth]{Matlab/tema6/spline3_1.png}
    \end{minipage}
    \hfill
    \begin{minipage}{0.45\textwidth}
        \centering
        \includegraphics[width=\linewidth]{Matlab/tema6/spline3_2.png}
    \end{minipage}

    % fila inferior
    \vskip\baselineskip
    \begin{minipage}{0.45\textwidth}
        \centering
        \includegraphics[width=\linewidth]{Matlab/tema6/spline3_3.png}
    \end{minipage}
    \hfill
    \begin{minipage}{0.45\textwidth}
        \centering
        \includegraphics[width=\linewidth]{Matlab/tema6/spline3_4.png}
    \end{minipage}

    \caption{Gráficas de las interpolaciones cúbicas del ejercicio 12.}
    \label{fig:cuatroimagenes}
\end{figure}

\newpage

c) Obtener el interpolador splin quíntico para cada eslabón.\\

En este caso, aplicamos la fórmula 6.9 de la página 291 del libro:

\[
q(t) = a + b \tau + c \tau^2 + d \tau^3 + e \tau^4 + f \tau^5,
\quad \tau = t-t_i, \quad T = t_{i+1} - t_i \text{ (duración del tramo)}
\]

\[
\begin{aligned}
a &= q_i, \\
b &= \dot q_i T, \\
c &= \tfrac{1}{2}\ddot q_i T^2, \\
d &= 10(q_f - q_i) - (6\dot q_i + 4\dot q_f)T - \tfrac{1}{2}(3\ddot q_i - \ddot q_f)T^2, \\
e &= -15(q_f - q_i) + (8\dot q_i + 7\dot q_f)T + (3\ddot q_i - 2\ddot q_f)T^2, \\
f &= 6(q_f - q_i) - (3\dot q_i + 3\dot q_f)T - (\ddot q_i - \ddot q_f)T^2.
\end{aligned}
\] \\


Primer eslabón:
    \[
q_i = 90^\circ,\quad q_f = 0^\circ,\quad \Delta = -90^\circ
\]

\[
a=90,\quad b=0,\quad c=0,\quad d=10 \cdot (-90)=-900,\quad e=-15 \cdot (-90)=1350,\]
\[\quad f=6 \cdot(-90)=-540
\]




\[
\boxed{\,q_1(t) = 90 - 900\,t^3 + 1350\,t^4 - 540\,t^5\,}
\]

Segundo eslabón:
\[
q_i = 1,\quad q_f = 4,\quad \Delta = 3
\]




\[
a=1,\quad b=0,\quad c=0,\quad d=10 \cdot (3)=30,\quad e=-15\cdot (3)=-45,\quad f=6\cdot (3)=18
\]




\[
\boxed{\,q_2(t) = 1 + 30\,t^3 - 45\,t^4 + 18\,t^5\,}
\]

Tercer eslabón:
\[
q_i = 0^\circ,\quad q_f = 45^\circ,\quad \Delta = 45^\circ
\]


\[
a=0,\quad b=0,\quad c=0,\quad d=10 \cdot (45)=450,\quad e=-15\cdot (45)=-675,\quad f=6\cdot (45)=270
\]




\[
\boxed{\,q_3(t) = 450\,t^3 - 675\,t^4 + 270\,t^5\,}
\]

Cuarto eslabón:
\[
q_i = 4,\quad q_f = 2,\quad \Delta = -2
\]




\[
a=4,\quad b=0,\quad c=0,\quad d=10\cdot (-2)=-20,\quad e=-15 \cdot(-2)=30,\quad f=6 \cdot(-2)=-12
\]




\[
\boxed{\,q_4(t) = 4 - 20\,t^3 + 30\,t^4 - 12\,t^5\,}
\] \\

Si revolvemos en \textit{Matlab} utilizando la \textit{toolbox} podemos obtener las gráficas de las interpolaciones en la posición, velocidad y aceleración
(figuras \ref{fig:spline5pos}, \ref{fig:spline5vel}, \ref{fig:spline5acc}):
\lstinputlisting[]{Matlab/tema6/spline5_toolbox.m}

\begin{figure}[htbp]
    \centering
    \includegraphics[scale=0.57]{Matlab/tema6/spline5_pos.png}
    \caption{Gráfica de la interpolación quíntica en la posición.} \label{fig:spline5pos}
\end{figure}

\begin{figure}[htbp]
    \centering
    \includegraphics[scale=0.57]{Matlab/tema6/spline5_vel.png}
    \caption{Gráfica de la interpolación quíntica en la velocidad.} \label{fig:spline5vel}
\end{figure}

\begin{figure}[htbp]
    \centering
    \includegraphics[scale=0.57]{Matlab/tema6/spline5_acc.png}
    \caption{Gráfica de la interpolación quíntica en la velocidad.} \label{fig:spline5acc}
\end{figure}

\clearpage
d) Obtener los perfiles o gráficas de posición, velocidad y aceleración para un splin quíntico mediante \textit{Matlab}, 
  con las condiciones de contorno del apartado b).

\lstinputlisting[]{Matlab/tema6/spline5_toolbox2.m}

\begin{figure}[htbp]
    \centering
    \includegraphics[scale=0.65]{Matlab/tema6/spline5t_pos.png}
    \caption{Gráfica de la interpolación quíntica por tramos en la posición.} \label{fig:spline5tpos}
\end{figure}

\begin{figure}[htbp]
    \centering
    \includegraphics[scale=0.65]{Matlab/tema6/spline5t_vel.png}
    \caption{Gráfica de la interpolación quíntica por tramos en la velocidad.} \label{fig:spline5tvel}
\end{figure}

\begin{figure}[htbp]
    \centering
    \includegraphics[scale=0.65]{Matlab/tema6/spline5t_acc.png}
    \caption{Gráfica de la interpolación quíntica por tramos en la velocidad.} \label{fig:spline5tacc}
\end{figure}

e) Se han realizado interpolaciones cúbicas en los eslabones con un punto intermedio. Para las articulaciones prismáticas
  el primer tramo es $q_{p1} = 1 + 2t²+3t³$ y el segundo tramo $q_{p2} = 2 + t + 4t²-3t³$. Para las rotacionales el primer tramo 
  es $q_{r1} = 5 + 25t² - 15t³$ y el segundo tramo $q_{r2} = 15 + 5t + 5t² - 5t³$. Si se suponen las velocidades inicial y final cero,
  ¿cuáles son las coordenadas cartesianas de la posición final del extremo del segundo eslabón?

  Nota: las posiciones del extremo del segundo eslabón vienen dadas por:
  \[ x = L_1\cdot \cos(q_1), \qquad y = L_1\cdot \sin(q_1), \qquad z = q_2\]

Las interpolaciones cúbicas se definen en el apartado a) y en la fórmula 6.5 de la página 290 del libro.

De $q_{p2} = 2 + t + 4t²-3t³$ sabemos:
\[ a_{p2} = 2, \quad b_{p2} = 1\]

De $q_{p1} = 1 + 2t²+3t³$ sabemos:
\[ a_{p1} = 1, \quad b_{p1} = 0, \quad c_{p1} = 2 = \frac{3 \cdot (a_{p2} - 1)}{t²_f} - \frac{0 + 2 \cdot b_{p2}}{t_f}, \quad d_{p1} = 3\]

Despejamos $t_f$ y evaluamos la interpolación del tramo segundo de la primera articulación rotacional y de la primera prismática para obtener las coordenadas:
\[c_{p1} = 2 = \frac{3 \cdot (2 - 1)}{t²_f} - \frac{0 + 2 \cdot 1}{t_f} \rightarrow t_{f} = 1\]

\[ q_{p2} (1) = 2 + 1 + 4\cdot 1²-3 \cdot 1³ = 4\]
\[ q_{r2} (1) = 15 + 5\cdot 1 + 5\cdot 1²-5 \cdot 1³ = 20\]

Por tanto:
\[ q_1 = 20 \text{ grados},\; q_2 = 4, \qquad x = L_1\cdot \cos(20),\; y = L_1\cdot \sin(20),\; z = 4\]


\end{enumerate}