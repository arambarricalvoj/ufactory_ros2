\datos{color}{}{}

El \textbf{control borroso}, \textbf{difuso} o \textbf{\textit{fuzzy}} es un algoritmo que traduce \textbf{estrategias de control expresadas en lenguaje natural} en sistemas automáticos, imitando el razonamiento humano. A diferencia de los modelos estadísticos o deterministas, este método se apoya en la lógica difusa para \underline{manejar la incertidumbre} inherente al lenguaje natural y la \underline{información imprecisa}, empleando funciones de pertenencia que establecen aproximaciones cualitativas entre una instancia y un conjunto borroso, lo que lo convierte en una técnica de control \underline{heurística}. \\

La lógica borrosa, propuesta por Zadeh en 1965, se basa en la idea de \textbf{conjuntos difusos} que presentan una transición gradual en la pertenencia de sus elementos, permitiendo que una misma instancia tenga \underline{distintos grados de asociación con varios conjuntos}. Esta extensión de la teoría clásica de conjuntos asigna valores entre cero y uno para representar la pertenencia parcial, configurando una lógica multivaluada que no cumple las leyes tradicionales de contradicción ni de distribución. Su aplicación en el control de sistemas resulta ventajosa, pues \underline{no requiere modelos matemáticos precisos} y puede \underline{operar de manera eficiente ante el ruido} o la \underline{inexactitud de los sensores} en sistemas complejos. Una extensión de la lógica difusa describe la física cuántica y por tanto, las bases de la computación cuántica.\\

Los conjuntos borrosos se describen mediante \textbf{modificadores lingüísticos o etiquetas} como alto, muy alto o poco alto, que determinan los distintos grados, es decir, el \underline{nivel de granularidad}, de una \textbf{variable lingüística}. Dado que es difícil diferenciar entre matices como alta, bastante alta o muy alta, se recomiendan entre tres y once etiquetas como máximo para conservar una resolución adecuada. Para cada modificador se aplica el álgebra de conjuntos borrosos, que modela su interpretación semántica y permite definir las \textbf{funciones de pertenencia}. En el Cuadro \ref{table:algebraDifusa} se muestra un ejemplo de dicho modelado, y considerando la altura como variable y el álgebra del Cuadro \ref{table:algebraDifusa}, se define una función de pertenencia de tipo \textit{gamma}, \ref{eq:alto}. Para una altura de 1,75 metros se obtienen los valores de pertenencia del Cuadro \ref{table:valores}. Este proceso se define como \textit{fuzzificación}.\\

Finalmente, se establecen las \textbf{reglas de producción} que permiten inferir resultados enlazando antecedentes con las premisas mediante la generalización de los operadores lógicos de conjunción, disyunción, intersección, y negación. Se trata de un enfoque basado en la lógica de predicados y constituye el paradigma de la lógica borrosa, ya que es posible \underline{lanzar o activar las reglas sin disponer de todo el conocimiento} de los antecedentes. \\

Entre las reglas más utilizadas destacan las \underline{“si–entonces”} de tipo \textbf{\textit{Mamdani}}, cuya salida puede obtenerse por diversos métodos como izquierda, media o derecha del máximo, bisectriz, promediado de pesos o centroide del área. En la Tabla \ref{table:reglas} se muestra un ejemplo de las reglas y su aplicación, que determinan que una persona de 1,75 metros de altura debe agacharse poco. Este proceso se conoce como \textit{defuzzificación}. \\ 

\begin{table}[htb]
\centering
\begin{tabular}{|c|c|}
\hline
\rowcolor[HTML]{000000} 
\multicolumn{1}{|l|}{\cellcolor[HTML]{000000}{\color[HTML]{FFFFFF} MODIFICADORES LINGÜÍSTICOS}} & \multicolumn{1}{l|}{\cellcolor[HTML]{000000}{\color[HTML]{FFFFFF} ÁLGEBRA DE CONJUNTOS BORROSOS}} \\ \hline
\rowcolor[HTML]{FFFFFF} 
{\color[HTML]{000000} Alto}                                                                     & {\color[HTML]{000000} $x$}                                                                          \\ \hline
Muy alto                                                                                        & $x²$                                                                                                \\ \hline
\rowcolor[HTML]{FFFFFF} 
{\color[HTML]{000000} Poco alto}                                                                & {\color[HTML]{000000} $\sqrt{x}$}                                                                    \\ \hline
\end{tabular}
\caption{Ejemplo del álgebra de conjuntos borrosos.}\label{table:algebraDifusa}
\end{table}

\begin{equation}
\mu_{\text{alto}}(x; 1,6; 1,9) =
\begin{cases}
0 & \text{si } x \leq 1,6 \\
\frac{x - 1,6}{1,9 - 1,6} & \text{si } 1,6 < x < 1,9 \\
1 & \text{si } x \geq 1,9
\end{cases}
\tag{Ec. 1} \label{eq:alto}
\end{equation}

\vspace{-0.2cm}
\begin{table}[htb]
\centering
\begin{tabular}{|l|l|l|}
\hline
\rowcolor[HTML]{000000} 
{\color[HTML]{FFFFFF} $\mu_{\text{alto}}$}                                           & {\color[HTML]{FFFFFF} $\mu_{\text{muy alto}}$}                                     & {\color[HTML]{FFFFFF} $\mu_{\text{poco alto}}$} \\ \hline
\multicolumn{1}{|c|}{\cellcolor[HTML]{FFFFFF}{\color[HTML]{000000} 0.5}} & \multicolumn{1}{c|}{\cellcolor[HTML]{FFFFFF}{\color[HTML]{000000} 0.25}} & \multicolumn{1}{c|}{0.71} \\ \hline
\end{tabular}
\caption{Ejemplo del cálculo de función de pertenencia.}\label{table:valores}
\end{table}

\vspace{-0.3cm}
\begin{table}[htb]
\centering
\begin{tabular}{|c|c|c|}
\hline
\rowcolor[HTML]{000000} 
{\color[HTML]{FFFFFF} REGLA} & {\color[HTML]{FFFFFF} APLICACIÓN} & {\color[HTML]{FFFFFF} RESULTADO} \\ \hline
\rowcolor[HTML]{FFFFFF} 
Si es muy alto, entonces debe agacharse & Si 0.25, entonces agachar $0.25·2$ & 0.5 \\ \hline
Si es alto, entonces podría agacharse & Si 0.5, entonces agachar 0.5 & 0.5 \\ \hline
Si es poco alto, entonces podría agacharse & Si 0.71, entonces agachar $1-0.71$ & 0.29 \\ \hline
\rowcolor[HTML]{DDDDDD} 
\multicolumn{2}{|c|}{\textbf{SALIDA calculada mediante el promedio o media}} & \textbf{0.43} \\ \hline
\end{tabular}
\caption{Ejemplo de reglas difusas, aplicación y salida.} \label{table:reglas}
\end{table}



Este tipo de control ha demostrado su eficacia en diferentes campos, por ejemplo, en la robótica autónoma. En \cite{futbol} se propone un \textbf{controlador difuso para plataformas autónomas en fútbol robótico que esquiven obstáculos y mantengan la trayectoria al punto objetivo}. En este caso, implementan dos controladores difusos encadenados mediante ponderación que determina el peso de cada uno en función de la distancia al punto de destino. El primero de ellos busca y esquiva obstáculos, mientras que el segundo conduce al robot al destino deseado. \\

Los \textbf{obstáculos} se detectan mediante dos sensores y su distancia relativa al robot se modela en los conjuntos borrosos de \underline{muy cerca, cerca, media y lejos}. La distancia a la \textbf{posición a alcanzar} se modela con los conjuntos difusos de \underline{cerca y lejos}. La \textbf{velocidad} de salida por cada uno de los motores se define en los conjuntos de \underline{negativa rápida, negativa media, negativa lenta, nula, lenta, media y rápida}. Observando la Figura 24 en la página 11 del documento digital del artículo, se deduce que para todos esos conjuntos se han utilizado \textbf{funciones triángulares} de pertenencia, definidas correctamente ya que nigún valor del dominio presentado queda sin asignación. \\

En cuanto a las \textbf{reglas}, los autores definen un total de \underline{32 de tipo \textit{Mamdani}}. Reciben como \textbf{entradas} las distancias a los obstáculos detectadas por ambos sensores y la distancia al punto objetivo. La \textbf{salida}, como se intuye del párrafo anterior, son las velocidades de los motores izquierdo y derecho. Estas reglas se presentan en la Tabla 4 en la página 12. \\

Tras haber estudiado el tema me he dado cuenta de que una \textbf{alternativa} posible \textbf{a controladores PID} son los sistemas de control \textit{fuzzy}. En mi Trabajo de Fin de Grado \cite{arambarricalvoj_percepcion-control-ros2-tfg} implementé un sistema PID para que un \textit{rover} pudiera desplazarse autónomamente por un laberinto. Debido a la cinemática del robot el cálculo de las ganancias de control era muy complejo, por lo que obtuve una aproximación mediante ensayo y error analizando el comportamiento de cada una de ellas. Una implementación de control borroso habría sido \underline{más simple y adecuada}.\\

Además, es posible crear un sintonizador automático de PID utilizando un control \textit{fuzzy} basado en el método de ensayo y error como hice manualmente. Una de mis ilusiones era crearlo mediante \textit{Deep} o \textit{Reinforcement Learning}, pero \textbf{al tratar con falta de información el enfoque basado en lógica difusa sería más apropiado y simple}. \\
